% Copyright © 2012 Martin Ueding <dev@martin-ueding.de>
%
\documentclass[11pt, ngerman, fleqn]{article}

\usepackage[a4paper, left=3cm, right=2cm, top=2cm, bottom=2cm]{geometry}
\usepackage[activate]{pdfcprot}
\usepackage[cdot, squaren]{SIunits}
\usepackage[iso]{isodate}
\usepackage[parfill]{parskip}
\usepackage[T1]{fontenc}
\usepackage[utf8]{inputenc}
\usepackage{amsmath}
\usepackage{amssymb}
\usepackage{amsthm}
\usepackage{babel}
\usepackage{color}
\usepackage{commath}
\usepackage{epstopdf}
\usepackage{fancyhdr}
\usepackage{graphicx}
\usepackage{hyperref}
\usepackage{lastpage}
\usepackage{setspace}
\usepackage{tikz}

\usepackage[charter, greekuppercase=italicized]{mathdesign}

\definecolor{darkblue}{rgb}{0,0,.5}
\definecolor{darkgreen}{rgb}{0,.5,0}

\hypersetup{
	breaklinks=false,
	citecolor=darkgreen,
	colorlinks=true,
	linkcolor=black,
	menucolor=black,
	urlcolor=darkblue,
}

\setlength{\columnsep}{2cm}

\DeclareMathOperator{\arcsinh}{arsinh}
\DeclareMathOperator{\arsinh}{arsinh}
\DeclareMathOperator{\asinh}{arsinh}
\DeclareMathOperator{\card}{card}
\DeclareMathOperator{\diam}{diam}

\newcommand{\dalambert}{\mathop{{}\Box}\nolimits}
\newcommand{\divergence}[1]{\inner{\vnabla}{#1}}
\newcommand{\ee}{\mathrm e}
\newcommand{\emesswert}{\del{\messwert \pm \messwert}}
\newcommand{\e}[1]{\cdot 10^{#1}}
\newcommand{\fehlt}{\textcolor{red}{Hier fehlen noch Inhalte.}}
\newcommand{\half}{\frac 12}
\newcommand{\ii}{\mathrm i}
\newcommand{\inner}[2]{\left\langle #1, #2 \right\rangle}
\newcommand{\laplace}{\mathop{{}\Deltaup}\nolimits}
\newcommand{\messwert}{\textcolor{blue}{\square}}
\newcommand{\punkte}{\textcolor{white}{xxxxx}}
\newcommand{\tens}[1]{\boldsymbol{#1}}
\newcommand{\vnabla}{\vec \nabla}
\renewcommand{\vec}[1]{\boldsymbol{#1}}

\newcommand{\themodul}{math340}
\newcommand{\thegruppe}{Gruppe 16 -- Malte Lackmann}
\newcommand{\theuebung}{3}

\pagestyle{fancy}

\fancyfoot[C]{\footnotesize{\thegruppe}}
\fancyfoot[L]{\footnotesize{Ueding, Manz, Lemmer}}
\fancyfoot[R]{\footnotesize{Seite \thepage\ / \pageref{LastPage}}}
\fancyhead[L]{\themodul{} -- Übung \theuebung}

\setcounter{section}{0}

\def\thesection{\theuebung.\arabic{section}}
\def\thesubsection{\thesection\alph{subsection}}

\title{\themodul{} -- Übung \theuebung \\ \vspace{0.5cm} \large{\thegruppe}}

\author{Martin Ueding \\ \small{\href{mailto:mu@uni-bonn.de}{mu@uni-bonn.de}} \and Paul Manz \and Lino Lemmer}

\begin{document}

\maketitle

\begin{table}[h]
	\centering
	\begin{tabular}{l|c|c|c|c}
		Aufgabe & \ref{sec:1} & \ref{sec:2} & \ref{sec:3} & $\sum$   \\
		\hline
		Punkte & \punkte / 2 & \punkte / 6 & \punkte / 6 & \punkte / 14
	\end{tabular}
\end{table}

%%%%%%%%%%%%%%%%%%%%%%%%%%%%%%%%%%%%%%%%%%%%%%%%%%%%%%%%%%%%%%%%%%%%%%%%%%%%%%%
%                          Orthogonalitätsrelation                           %
%%%%%%%%%%%%%%%%%%%%%%%%%%%%%%%%%%%%%%%%%%%%%%%%%%%%%%%%%%%%%%%%%%%%%%%%%%%%%%%

\section{Orthogonalitätsrelation}
\label{sec:1}

Es soll gezeigt werden, dass die Basisvektoren des $L^2$ linear unabhängig sind.

\paragraph{Sinus}

Für den Fall $m \neq n$:
\begin{align*}
	\int_{-\pi}^\pi \dif x \sin\del{nx} \sin\del{mx}
	&= \half \int_{-\pi}^\pi \dif x \del{\cos\del{\del{n-m} x} - \cos\del{\del{n+m} x}} \\
	&= \half \sbr{\frac{1}{n-m} \sin\del{\del{n-m} x} - \frac{1}{n+m} \sin\del{\del{n+m} x}}_{-\pi}^\pi  \\
	&= 0
\end{align*}

Für den Fall $m = n$:
\begin{align*}
	\int_{-\pi}^\pi \dif x \sin^2\del{mx}
	&= \int_{-\pi}^\pi \dif x \frac{1-\cos\del{2nx}}{2} \\
	&= \sbr{\half x - \frac{1}{2n} \sin\del{2nx}}_{-\pi}^\pi \\
	&= \pi
\end{align*}

\paragraph{Sinus und Kosinus}

Für den Fall $m \neq n$:
{
	\renewcommand\d\dif
	\begin{align*}
		\int_{-\pi}^{\pi}\d x\sin\left(mx\right)\cos\left(nx\right)&=\frac 12\int_{-\pi}^{\pi}\d x\sin\left(\left(m-n\right)x\right)+\sin\left(\left(m+n\right)x\right)\\
															 &=-\frac 12 \left[\frac 1{m-n}\cos\left(\left(m-n\right)x\right)+\frac 1{m+n} \cos\left(\left(m+n\right)x\right) \right]_{-\pi}^{\pi}\\
				&=0
	\end{align*}
}

Für den Fall $m = n$ muss auf den ersten Summanden verzichtet werden. Der
zweite Summand ist allerdings auch eigenständig $0$, so dass die Relation auch
dann gilt.

\paragraph{Kosinus}
{
	\renewcommand\d\dif
	F"ur den Fall $m\neq n$:
	\begin{align*}
	\int_{-\pi}^{\pi}\d x\cos\left(mx\right)\cos\left(nx\right)&=\frac 12\int_{-\pi}^{\pi}\d x\cos\left(\left(m-n\right)x\right)+\cos\left(\left(m+n\right)x\right)\\
	&=\frac 12 \left[\frac 1{m-n}\sin\left(\left(m-n\right)x\right)+\frac 1{m+n} \sin\left(\left(m+n\right)x\right) \right]_{-\pi}^{\pi}\\
	&=0
	\end{align*}
Für den Fall $m=n$:
	\begin{align*}
	\int_{-\pi}^{\pi}\d x\cos\left(mx\right)\cos\left(nx\right)&=\frac 12\int_{-\pi}^{\pi}\d x \,1+\cos\left(2mx\right)\\
	&=\frac 12 \left[x+\frac 1{2m} \sin\left(2mx\right) \right]_{-\pi}^{\pi}\\
	&=\pi
	\end{align*}
}

%%%%%%%%%%%%%%%%%%%%%%%%%%%%%%%%%%%%%%%%%%%%%%%%%%%%%%%%%%%%%%%%%%%%%%%%%%%%%%%
%                            duhamelsches Prinzip                             %
%%%%%%%%%%%%%%%%%%%%%%%%%%%%%%%%%%%%%%%%%%%%%%%%%%%%%%%%%%%%%%%%%%%%%%%%%%%%%%%

\stepcounter{section}
\section{duhamelsches Prinzip}
\label{sec:2}

Für ein Anfangswertproblem mit Störfunktion $f$ ist die Lösung:
\[
	u(x, t) = \frac a2 \int_0^t \dif s \int_{x-a(t-s)}^{x+a(t-s)} \dif y f(y, s)
\]

\stepcounter{subsection}
\subsection{Cauchy-Probleme}

\paragraph{die erste}

Gelöst werden soll:
\[
	- \dpd[2]ux + \frac{1}{a^2} \dpd[2]ut = \frac{1}{a^2} \sin(\omega x)
	,\quad
	u(x, 0) = 0
	,\quad
	\dpd ut(x, 0) = 0
\]

Dazu setze ich die Störfunktion ein und rechne das Integral aus:
\begin{align*}
	u(x, t) &= \frac a2 \int_0^t \dif s \int_{x-a(t-s)}^{x+a(t-s)} \dif y \frac{1}{a^2} \sin(\omega y) \\
   &= \frac{1}{2 a^2 \omega} \del{\sin\del{x+at} + \sin\del{x-at} - 2 \sin(x)}
\end{align*}

Die Bedingungen $u(x, 0) = 0$ und $\pd ut(x, 0) = 0$ sind erfüllt.

\paragraph{die zweite}

Ich benutze $u'' := \pd[2]ux$ und $\ddot u := \pd[2]ut$ als Abkürzungen.

Gelöst werden soll:
\[
	- u'' + \frac{1}{2^2} \ddot u = \frac 1{2^2} \sin(x)
	,\quad
	u(x, 0) = \sin(x)
	,\quad
	\dot u(x, 0) = 0
\]

Dazu löse ich die homogene Gleichung mit den Anfangsbedingungen mit der Formel aus \cite[Seite 41]{john-1910}. Die Formel ist mit Anfangsbedingungen $u(x, 0) =: u_0(x)$ und $\dot u(x, 0) =: u_1(x)$:
\[
	u(x, t) = \half \del{
		u_0(x+ct) + u_u(x-ct)
	}
	+ \frac{1}{2a} \int_{x-at}^{x+at} \dif \xi u_1(\xi)
\]

Die Lösung ist dann:
\[
	u_h(x, t) = \frac 18 \del{\sin(x - 2t) + \sin(x + 2t)}
\]

Nun löse ich noch die inhomogene Gleichung mit $u(x, 0) = 0 \wedge \dot u(x, 0) = 0$. Die Summe dieser Lösung und $u_h$ erfüllen sowohl die Differentialgleichung (da diese linear ist) als auch die Anfangsbedingungen. Nach dem duhamelschen Prinzip ist die inhomogene Lösung:
\[
	u_p(x, t)
	= \frac 14 \int_0^t \dif s \int_{x-a(t-s)}^{x+a(t-s)} \dif y \sin(y)
\]

Dies ist aufgelöst:
\[
	u_p(x, t)
	= \frac 18 \del{
		\sin(x-4t) - \sin(x-2t) - \sin(x) + \sin(x+2t)
	}
\]

Die Lösung ist die Summe, also:
\[
	u(x, t)
	= \frac 18 \del{
		\sin(x-4t) - \sin(x) + 2 \sin(x+2t)
	}
\]

Diese Funktion erfüllt die Anfangsbedingungen, allerdings kommt in die
Differentialgleichung eingesetzt nicht ganz $\sin(x)$ heraus, sondern:
\[
	\sin(x) + \half \del{
		- 5 \sin(x-4t) - \frac 94 \sin(x+2t)
	}
\]

\paragraph{die zweite zweite}

Gelöst werden soll:
\[
	- u'' + \frac 19 \ddot u = \frac 19 \sin(t)
	,\quad
	u(x, 0) = 1
	,\quad
	\dot u(x, 0) = 1
\]

Die Lösung des homogen Anfangswertproblem ist:
\[
	u_h(x, t) = 1 + t
\]

Die Lösung der inhomogenen Gleichung mit $u(x, 0) = 0 \wedge \dot u(x, 0)$ ist:
\[
	u_p(x, t) = t \del{
		1 - \cos(t)
	} +1 + t
\]

Als Summe ergibt sich:
\[
	u(x, t) = 1 + 2t - t \cos(t)
\]

Dies erfüllt die Anfangsbedingungen. In die Differentialgleichung eingesetzt
ergibt sich allerdings wie oben etwas anderes:
\[
	\sin(t) + t \cos(t) + \sin(t) \neq \sin(t)
\]

%%%%%%%%%%%%%%%%%%%%%%%%%%%%%%%%%%%%%%%%%%%%%%%%%%%%%%%%%%%%%%%%%%%%%%%%%%%%%%%
%                           Randanfangswertaufgaben                           %
%%%%%%%%%%%%%%%%%%%%%%%%%%%%%%%%%%%%%%%%%%%%%%%%%%%%%%%%%%%%%%%%%%%%%%%%%%%%%%%

\section{Randanfangswertaufgaben}
\label{sec:3}

\subsection{die Erste}

In \cite[Seite 43]{john-1910} wird dieser Lösungsansatz gegeben. Allerdings wird dort als
Grenze $\pi$ und nicht $1$ genommen. Daher habe ich hier das $\pi$ in der
Funktion entfernt. Man kann das sicher \emph{irgendwie} wieder reinbasteln ...

Als Ansatz wähle ich:
\[
	u(x, t) = \sum_n a_n(t) \sin(nx)
\]

Dabei muss für jeden $a_n$ in der Reihe auch die Differentialgleichung erfüllt
sein. Somit gilt die gleiche, homogene, DGL ohne Randbedingungen auch für die
Koeffizienten. Diese Differentialgleichung lässt sich zur eine
Linearkombination von Kosinus und Sinus lösen. Somit sind die Koeffizienten:
\[
	a_n(t) = c_n \cos(nat) + d_n \sin(nat)
\]

Die Koeffizenten kann ich nun durch Entwicklung der Anfangsbedingung $u_0(x) :=
u(x, 0)$ nach Kosinus ermitteln:
\[
	c_n = \frac 2\pi \int_0^\pi \dif x u_0(x) \cos(nx) = 0
\]

Die Sinuskoeffizienten analog, allerdings $u_1(x) := \dot u(x, 0)$:
\[
	d_n = \frac 2\pi \int_0^\pi \dif x \sin(2x) \sin(nx) = - \frac{4 \sin(n \pi)}{cn \del{-4 + n^2} \pi}
\]

Die $d_n$ sind 0 für alle $n$, außer für $n = 2$, dann ist es allerdings ein
Grenzfall. Ich wende l'Hospital für $n$ an. Dann erhalte ich den recht einfachen Ausdruck:
\[
	d_2 = - \frac{1}{2c}
\]

Somit ist die Lösungsfunktion:
\[
	u(x, t) = - \frac{1}{2c} \sin(2at) \sin(2x)
\]

Die Funktion erfüllt die Differentialgleichung und die Anfangsbedingungen und
die Randbedingungen.

\subsection{die Zweite}

\bibliography{../../zentrale_BibTeX/Central}
\bibliographystyle{plain}

\end{document}
