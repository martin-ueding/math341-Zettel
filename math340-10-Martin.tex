% Copyright © 2012 Martin Ueding <dev@martin-ueding.de>
%
\input{header.tex}

\usepackage{scrpage2}
\usepackage{tikz}

\newcommand{\themodul}{math340}
\newcommand{\thegruppe}{Gruppe 16 -- Malte Lackmann}
\newcommand{\theuebung}{10}

\pagestyle{scrheadings}

\automark{section}
\cfoot{\footnotesize{\thegruppe}}
\chead{}
\ifoot{\footnotesize{Ueding, Manz, Lemmer}}
\ihead{\themodul{} -- Übung \theuebung}
\ofoot{\footnotesize{Seite \thepage\ / \pageref{LastPage}}}
\ohead{\rightmark}
\setheadsepline{.4pt}


\def\thesection{\theuebung.\arabic{section}}
\def\thesubsection{\thesection\alph{subsection}}

\title{\themodul{} -- Übung \theuebung \\ \vspace{0.5cm} \large{\thegruppe}}

\author{
	Martin Ueding \\ \small{\href{mailto:mu@uni-bonn.de}{mu@uni-bonn.de}}
	\and
	Paul Manz
	\and
	Lino Lemmer
}

\begin{document}

\maketitle

\begin{table}[h]
	\centering
	\begin{tabular}{l|c|c|c|c|c}
		Aufgabe & \ref 1 & \ref 2 & \ref 3 & $\sum$   \\
		\hline
		Punkte & \punkte / 4 & \punkte / 4 & \punkte / 8 & \punkte / 16
	\end{tabular}
\end{table}

\stepcounter{section}

%%%%%%%%%%%%%%%%%%%%%%%%%%%%%%%%%%%%%%%%%%%%%%%%%%%%%%%%%%%%%%%%%%%%%%%%%%%%%%%
%                           Real- und Imaginärteil                           %
%%%%%%%%%%%%%%%%%%%%%%%%%%%%%%%%%%%%%%%%%%%%%%%%%%%%%%%%%%%%%%%%%%%%%%%%%%%%%%%

\section{Real- und Imaginärteil}
\label 1

\subsection{}
\begin{align*}
	z=\frac{1+\ii}{1-\del{1+\ii}^2}&=-\frac{1}{5}+\frac{3}{5}\ii\\
								\Re\del{z}&=-\frac{1}{5}\\
					 \Im\del z&=\frac{3}{5}\\
					 \abs z&=\sqrt{\frac{2}{5}}\\
			\arg \del{z}&=\arctan\del{-3}
\end{align*}
\subsection{}
\begin{align*}
	z=\del{\frac{1+\ii}{1-\ii}}^{99}&=-\ii\\
								  \Re\del z&=0\\
					  \Im\del z&=-1\\
					  \abs z&=1\\
			   \arg\del z&=\frac{3}{2}\pi
\end{align*}
\subsection{}
\begin{align*}
	z=\frac{2-\ii}{3\ii+\del{1-\ii}^2}&=-\frac{1}{5}-\frac{2}{5}\ii\\
								   \Re\del z&=-\frac{1}{5}\\
					   \Im\del z&=-\frac{2}{5}\\
					   \abs z&=\frac{1}{\sqrt{5}}\\
				\arg\del z&=\arctan\del{2}
\end{align*}
\subsection{}
\begin{align*}
	z=\del{\frac{2\ii}{1-\ii}}^9&=-16+16\ii\\
							  \Re\del z&=-16\\
					 \Im\del z&=16\\
					 \abs z&=\sqrt{2}\cdot 16\\
			  \arg\del z&=\frac 34\pi
\end{align*}

%%%%%%%%%%%%%%%%%%%%%%%%%%%%%%%%%%%%%%%%%%%%%%%%%%%%%%%%%%%%%%%%%%%%%%%%%%%%%%%
%                            komplexe Gleichungen                             %
%%%%%%%%%%%%%%%%%%%%%%%%%%%%%%%%%%%%%%%%%%%%%%%%%%%%%%%%%%%%%%%%%%%%%%%%%%%%%%%

\section{komplexe Gleichungen}
\label 2

\subsection{}

\begin{align*}
	z^7 + 4 &= 0 \\
		z^7 &= -4 \\
	\del{r \ee^{\ii \phi}}^7 &= -4 \\
		r^7 \ee^{7 \ii \phi} &= -4 \\
	\intertext{%
		$r$ muss reell sein, also ist $r = \sqrt[7]{4}$. Nun muss die
		Exponentialfunktion noch die Phase richtig auf $-1$ bekommen.
	}
	\ee^{7 \ii \phi} &= -1 \\
			  7 \phi &= \del{n + \half} \pi \\
		  \phi &= \frac{2n + 1}{14} \pi
\end{align*}

Die Lösungen sind alle $n = 0, \ldots, 6$, also 7 Lösungen, mit $r =
\sqrt[7]{4}$:
\[
	\mathbb L
	= \set{
		r \angle \frac{1}{14} \pi,
		r \angle \frac{3}{14} \pi,
		r \angle \frac{5}{14} \pi,
		r \angle \frac{7}{14} \pi,
		r \angle \frac{9}{14} \pi,
		r \angle \frac{11}{14} \pi,
		r \angle \frac{13}{14} \pi
	}
\]

\subsection{}

\begin{align*}
	z^6 + 64 &= 0 \\
		 r^6 \ee^{6 \ii \phi} &= -64 \\
	\intertext{%
		Wir wählen $r = \sqrt[6]{64}$.
	}
	\ee^{6 \ii \phi} &= -1 \\
							 6 \phi &= \frac{2n+1}{2} \pi \\
							\phi &= \frac{2n+1}{12} \pi
\end{align*}

Eindeutige Lösungen erhalten wir für $n = 0, \ldots, 5$, also 6 Lösungen. Mit $r = \sqrt[6]{64}$.
\[
	\mathbb L
	= \set{
		r \angle \frac{ 1}{12} \pi,
		r \angle \frac{ 3}{12} \pi,
		r \angle \frac{ 5}{12} \pi,
		r \angle \frac{ 7}{12} \pi,
		r \angle \frac{ 9}{12} \pi,
		r \angle \frac{11}{12} \pi,
	}
\]

%%%%%%%%%%%%%%%%%%%%%%%%%%%%%%%%%%%%%%%%%%%%%%%%%%%%%%%%%%%%%%%%%%%%%%%%%%%%%%%
%                                Trigonometrie                                %
%%%%%%%%%%%%%%%%%%%%%%%%%%%%%%%%%%%%%%%%%%%%%%%%%%%%%%%%%%%%%%%%%%%%%%%%%%%%%%%

\stepcounter{section}
\section{Trigonometrie}
\label 3

\subsection{Identitäten}

Die Definitionen der Funktionen sind:
\begin{align*}
	\cosh(z) &:= \frac 1{2} \del{\ee^{z} + \ee^{-z}} \\
	\sinh(z) &:= \frac 1{2} \del{\ee^{z} - \ee^{-z}} \\
	\cos(z) &:= \frac 1{2} \del{\ee^{\ii z} + \ee^{-\ii z}} \\
	\sin(z) &:= \frac 1{2 \ii} \del{\ee^{\ii z} - \ee^{-\ii z}}
\end{align*}

Zusammen mit $1/\ii = - \ii$ ist zu sehen, dass $\cosh(z) = \cos(\ii z)$,
$\cos(z) = \cosh(\ii z)$ sowie $\sinh(z) = - \ii \sin(\ii z)$ und $\sin(z) = -
\ii \sinh(\ii z)$ gilt.

Der trigonometrische Pythagoras:
\begin{align*}
	\sin^2(z) + \cos^2(z)
	&= \del{\frac 1{2 \ii} \del{\ee^{\ii z} - \ee^{-\ii z}}}^2 + \del{\frac 1{2} \del{\ee^{\ii z} + \ee^{-\ii z}}}^2 \\
	&= - \frac 14 \del{\ee^{\ii z} - \ee^{-\ii z}}^2 + \frac 14 \del{\ee^{\ii z} + \ee^{-\ii z}}^2 \\
	&= - \frac 14 \del{\ee^{2 \ii z} + 2 - \ee^{-2 \ii z}} + \frac 14 \del{\ee^{2 \ii z} + 2 + \ee^{-2 \ii z}} \\
	&= - \frac 14 \ee^{2 \ii z} + \frac 12 - \frac 14 \ee^{-2 \ii z} + \frac 14 \ee^{2 \ii z} + \frac 12 + \frac 14 \ee^{-2 \ii z} \\
	&= 1
\end{align*}

\subsection{berechnen}

\begin{gather*}
	\sin(\pi + 3 \ii)
	= \frac 1{2 \ii} \del{\ee^{\ii \del{\pi + 3 \ii}} - \ee^{-\ii \del{\pi + 3 \ii}}}
	= \frac 1{2 \ii} \del{\ee^{3} - \ee^{-3}}
	= - \ii \sinh(3) \\
	%
	\cosh(\ii)
	= \frac 1{2} \del{\ee^{\ii} + \ee^{-\ii}}
	= \cos(1)
\end{gather*}

\subsection{Sinus}

\begin{align*}
	\sin(z) &= 1000 \\
	\frac 1{2 \ii} \del{\ee^{\ii z} - \ee^{-\ii z}} &= 1000 \\
	\ee^{\ii z} - \ee^{-\ii z} &= 2000 \ii \\
	\ee^{\ii \del{a + b \ii}} - \ee^{-\ii \del{a + b \ii}} &= 2000 \ii \\
	\ee^{\ii a - b} - \ee^{-\ii a + b} &= 2000 \ii \\
	\ee^{\ii a - b} - \ee^{-\ii a + b} &= \ee^{\ln(2000) + \half \pi \ii} \\
	\intertext{
		Wir setzen $a = \half \pi$.
	}
	\ee^{\half \pi \ii - b} - \ee^{-\half \pi \ii + b} &= \ee^{\ln(2000) + \half \pi \ii} \\
	\ii \ee^{- b} + \ii \ee^{+ b} &= 2000 \ii \\
	\half \del{\ee^{- b} + \ee^{+ b}} &= 1000 \\
	\cosh(b) &= 1000 \\
		   b &\approx 7.6009
\end{align*}

Die Lösung ist $z = \half \ii \pi + \acosh(1000) \ii \approx \half \ii \pi +
7.6009 \ii$.

\subsection{Gleichungen}

Erste Gleichung:
\[
	\ee^x = 3
	\iff
	x = \ln(3) + 2 n \pi \ii, n \in \Z \\
\]

Zweite Gleichung:
\begin{align*}
	\ee^{x} &= 2 + 3 \ii
	= \exp\del{\ln\del{\sqrt{2^2 + 3^2}} + \arctan\del{\frac 32} \pi \ii} \\
	x &= \ln\del{\sqrt{15}} + \arctan\del{\frac 32} \pi \ii + 2 n \pi \ii, n \in \Z
\end{align*}

Dritte Gleichung:
\begin{align*}
	\ee^x &= \ee + 2 \pi \ii
 = \exp\del{\ln\del{\sqrt{\ee^2 + 4 \pi^2}} + \arctan\del{\frac{2 \pi}{\ee}} \pi \ii} \\
 x &= \ln\del{\sqrt{\ee^2 + 4 \pi^2}} + \arctan\del{\frac{2 \pi}{\ee}} \pi \ii + 2 n \pi \ii, n \in \Z
\end{align*}

%\bibliography{../../zentrale_BibTeX/Central}
%\bibliographystyle{plain}

\end{document}

% vim: spell spelllang=de
