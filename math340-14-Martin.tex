% Copyright © 2012-2013 Martin Ueding <dev@martin-ueding.de>
%
% Copyright © 2012-2013 Martin Ueding <dev@martin-ueding.de>
%
\documentclass[11pt, ngerman, fleqn, DIV=15]{scrartcl}

\usepackage{graphicx}

%%%%%%%%%%%%%%%%%%%%%%%%%%%%%%%%%%%%%%%%%%%%%%%%%%%%%%%%%%%%%%%%%%%%%%%%%%%%%%%
%                                Locale, date                                 %
%%%%%%%%%%%%%%%%%%%%%%%%%%%%%%%%%%%%%%%%%%%%%%%%%%%%%%%%%%%%%%%%%%%%%%%%%%%%%%%

\usepackage{babel}
\usepackage[iso]{isodate}

%%%%%%%%%%%%%%%%%%%%%%%%%%%%%%%%%%%%%%%%%%%%%%%%%%%%%%%%%%%%%%%%%%%%%%%%%%%%%%%
%                          Margins and other spacing                          %
%%%%%%%%%%%%%%%%%%%%%%%%%%%%%%%%%%%%%%%%%%%%%%%%%%%%%%%%%%%%%%%%%%%%%%%%%%%%%%%

\usepackage[activate]{pdfcprot}
\usepackage[parfill]{parskip}
\usepackage{setspace}

\setlength{\columnsep}{2cm}

%%%%%%%%%%%%%%%%%%%%%%%%%%%%%%%%%%%%%%%%%%%%%%%%%%%%%%%%%%%%%%%%%%%%%%%%%%%%%%%
%                                    Color                                    %
%%%%%%%%%%%%%%%%%%%%%%%%%%%%%%%%%%%%%%%%%%%%%%%%%%%%%%%%%%%%%%%%%%%%%%%%%%%%%%%

\usepackage{color}

\definecolor{darkblue}{rgb}{0,0,.5}
\definecolor{darkgreen}{rgb}{0,.5,0}
\definecolor{darkred}{rgb}{.7,0,0}

%%%%%%%%%%%%%%%%%%%%%%%%%%%%%%%%%%%%%%%%%%%%%%%%%%%%%%%%%%%%%%%%%%%%%%%%%%%%%%%
%                         Font and font like settings                         %
%%%%%%%%%%%%%%%%%%%%%%%%%%%%%%%%%%%%%%%%%%%%%%%%%%%%%%%%%%%%%%%%%%%%%%%%%%%%%%%

\usepackage[charter, greekuppercase=italicized]{mathdesign}
\usepackage{beramono}
\usepackage{berasans}

% Style of vectors and tensors.
\newcommand{\tens}[1]{\boldsymbol{\mathsf{#1}}}
\renewcommand{\vec}[1]{\boldsymbol{#1}}

%%%%%%%%%%%%%%%%%%%%%%%%%%%%%%%%%%%%%%%%%%%%%%%%%%%%%%%%%%%%%%%%%%%%%%%%%%%%%%%
%                               Input encoding                                %
%%%%%%%%%%%%%%%%%%%%%%%%%%%%%%%%%%%%%%%%%%%%%%%%%%%%%%%%%%%%%%%%%%%%%%%%%%%%%%%

\usepackage[T1]{fontenc}
\usepackage[utf8]{inputenc}

%%%%%%%%%%%%%%%%%%%%%%%%%%%%%%%%%%%%%%%%%%%%%%%%%%%%%%%%%%%%%%%%%%%%%%%%%%%%%%%
%                         Hyperrefs and PDF metadata                          %
%%%%%%%%%%%%%%%%%%%%%%%%%%%%%%%%%%%%%%%%%%%%%%%%%%%%%%%%%%%%%%%%%%%%%%%%%%%%%%%

\usepackage{hyperref}
\usepackage{lastpage}

\hypersetup{
	breaklinks=false,
	citecolor=darkgreen,
	colorlinks=true,
	linkcolor=black,
	menucolor=black,
	pdfauthor={Martin Ueding},
	urlcolor=darkblue,
}

%%%%%%%%%%%%%%%%%%%%%%%%%%%%%%%%%%%%%%%%%%%%%%%%%%%%%%%%%%%%%%%%%%%%%%%%%%%%%%%
%                               Math Operators                                %
%%%%%%%%%%%%%%%%%%%%%%%%%%%%%%%%%%%%%%%%%%%%%%%%%%%%%%%%%%%%%%%%%%%%%%%%%%%%%%%

\usepackage[thinspace, squaren]{SIunits}
\usepackage{amsmath}
\usepackage{amsthm}
\usepackage{commath}

% Word like operators.
\DeclareMathOperator{\acosh}{arcosh}
\DeclareMathOperator{\arcosh}{arcosh}
\DeclareMathOperator{\arcsinh}{arsinh}
\DeclareMathOperator{\arsinh}{arsinh}
\DeclareMathOperator{\asinh}{arsinh}
\DeclareMathOperator{\card}{card}
\DeclareMathOperator{\diam}{diam}
\renewcommand{\Im}{\mathop{{}\mathrm{Im}}\nolimits}
\renewcommand{\Re}{\mathop{{}\mathrm{Re}}\nolimits}

% Special single letters.
\DeclareMathOperator{\fourier}{\mathcal{F}}
\newcommand{\C}{\ensuremath{\mathbb C}}
\newcommand{\ee}{\mathrm e}
\newcommand{\ii}{\mathrm i}
\newcommand{\N}{\ensuremath{\mathbb N}}
\newcommand{\R}{\ensuremath{\mathbb R}}
\newcommand{\Z}{\ensuremath{\mathbb Z}}

% Shape like operators.
\DeclareMathOperator{\dalambert}{\Box}
\DeclareMathOperator{\laplace}{\bigtriangleup}
\newcommand{\curl}{\vnabla \times}
\newcommand{\divergence}[1]{\inner{\vnabla}{#1}}
\newcommand{\vnabla}{\vec \nabla}

% Shortcuts
\newcommand{\ev}{\hat{\vec e}}
\newcommand{\e}[1]{\cdot 10^{#1}}
\newcommand{\half}{\frac 12}
\newcommand{\inner}[2]{\left\langle #1, #2 \right\rangle}

% Placeholders.
\newcommand{\emesswert}{\del{\messwert \pm \messwert}}
\newcommand{\fehlt}{\textcolor{darkred}{Hier fehlen noch Inhalte.}}
\newcommand{\messwert}{\textcolor{blue}{\square}}
\newcommand{\punkte}{\textcolor{white}{xxxxx}}

% Separator for equations on a single line.
\newcommand{\eqnsep}{,\quad}

%%%%%%%%%%%%%%%%%%%%%%%%%%%%%%%%%%%%%%%%%%%%%%%%%%%%%%%%%%%%%%%%%%%%%%%%%%%%%%%
%                                  Headings                                   %
%%%%%%%%%%%%%%%%%%%%%%%%%%%%%%%%%%%%%%%%%%%%%%%%%%%%%%%%%%%%%%%%%%%%%%%%%%%%%%%

\usepackage{scrpage2}

\pagestyle{scrheadings}

\automark{section}
\cfoot{\footnotesize{Seite \thepage\ / \pageref{LastPage}}}
\chead{}
\ihead{}
\ohead{\rightmark}
\setheadsepline{.4pt}

%%%%%%%%%%%%%%%%%%%%%%%%%%%%%%%%%%%%%%%%%%%%%%%%%%%%%%%%%%%%%%%%%%%%%%%%%%%%%%%
%                            Bibliography (BibTeX)                            %
%%%%%%%%%%%%%%%%%%%%%%%%%%%%%%%%%%%%%%%%%%%%%%%%%%%%%%%%%%%%%%%%%%%%%%%%%%%%%%%

\newcommand{\bibliographyfile}{../../zentrale_BibTeX/Central}


\usepackage{scrpage2}
\usepackage{tikz}
\usetikzlibrary{decorations.markings}

\newcommand{\themodul}{math340}
\newcommand{\thegruppe}{Gruppe 16 -- Malte Lackmann}
\newcommand{\theuebung}{14}

\pagestyle{scrheadings}

\automark{section}
\cfoot{\footnotesize{\thegruppe}}
\chead{}
\ifoot{\footnotesize{Ueding, Manz, Lemmer}}
\ihead{\themodul{} -- Übung \theuebung}
\ofoot{\footnotesize{Seite \thepage\ / \pageref{LastPage}}}
\ohead{\rightmark}
\setheadsepline{.4pt}


\def\thesection{\theuebung.\arabic{section}}
\def\thesubsection{\thesection\alph{subsection}}

\title{\themodul{} -- Übung \theuebung \\ \vspace{0.5cm} \large{\thegruppe}}

\author{
	Martin Ueding \\ \small{\href{mailto:mu@uni-bonn.de}{mu@uni-bonn.de}}
	\and
	Paul Manz
	\and
	Lino Lemmer
}

\begin{document}

\maketitle

\begin{table}[h]
	\centering
	\begin{tabular}{l|c|c|c|c|c}
		Aufgabe
		& \ref 1
		& \ref 2
		& \ref 3
		& \ref 4
		& $\sum$ \\
		\hline
		Punkte
		& \punkte / 4
		& \punkte / 4
		& \punkte / 4
		& \punkte / 2
		& \punkte / 14
	\end{tabular}
\end{table}

%%%%%%%%%%%%%%%%%%%%%%%%%%%%%%%%%%%%%%%%%%%%%%%%%%%%%%%%%%%%%%%%%%%%%%%%%%%%%%%
%                         Hauptzweig des Logarithmus                          %
%%%%%%%%%%%%%%%%%%%%%%%%%%%%%%%%%%%%%%%%%%%%%%%%%%%%%%%%%%%%%%%%%%%%%%%%%%%%%%%

\section{Hauptzweig des Logarithmus}
\label 1

\subsection{$z_0 = 1$}

Wir entwickeln um $z_0 = 1$. Die Ableitungen sind, wie im Reellen:
\[
	f^{(n)} (-1)^{n+1} x^{-n}
\]

\textcolor{red}{
	Das sollte so nicht sein, vor allem, weil dann auch im weiteren Verlauf ein
	unendlicher Konvergenzradius vorkommen kann. Da $z = 0$ allerdings eine
	Polstelle ist, und die Funktion nur auf $\C^-$ definiert ist, sollte es
	Probleme geben.
}

Die Reihe ist mit $f^{(n)}(1) = (-1)^{n+1}$:
\[
	f(z) = \sum_{n=0}^\infty \frac{(-1)^{n+1}}{n!} \del{z - z_0}^n
\]

\subsection{$z_0 = -1 + \iup$}

\fehlt

%%%%%%%%%%%%%%%%%%%%%%%%%%%%%%%%%%%%%%%%%%%%%%%%%%%%%%%%%%%%%%%%%%%%%%%%%%%%%%%
%                            holomorphe Funktionen                            %
%%%%%%%%%%%%%%%%%%%%%%%%%%%%%%%%%%%%%%%%%%%%%%%%%%%%%%%%%%%%%%%%%%%%%%%%%%%%%%%

\stepcounter{section}
\section{holomorphe Funktionen}
\label 2

\subsection{erste Funktion}

\fehlt

\addtocounter{subsection}{2}
\subsection{vierte Funktion}

\fehlt

%%%%%%%%%%%%%%%%%%%%%%%%%%%%%%%%%%%%%%%%%%%%%%%%%%%%%%%%%%%%%%%%%%%%%%%%%%%%%%%
%                        Entwicklung in Laurentreihen                         %
%%%%%%%%%%%%%%%%%%%%%%%%%%%%%%%%%%%%%%%%%%%%%%%%%%%%%%%%%%%%%%%%%%%%%%%%%%%%%%%

\section{Entwicklung in Laurentreihen}
\label 3

\subsection{erste Funktion}

Wir führen eine Partialbruchzerlegung durch:
\[
	f(z) = \frac3{(z+1)(z-2)} = -\frac1{z+1} + \frac1{z-2}
\]

Wir beginnen mit dem ersten Summanden und schreiben diesen als geometrische
Reihe:
\[
	-\frac1{z+1}
	=
	- \frac 1z \frac1{1+\frac 1z}
	=
	- \frac 1z \sum_{k=0}^\infty \del{\frac 1z}^k
	- \sum_{k=1}^\infty \del{\frac 1z}^k
\]

Dabei müssen wir als Einschränkung $\abs z > 1$ einfügen, da die Reihe sonst nicht konvergiert.

Den zweiten Summanden schreiben wir in eine normale Taylorreihe, da wir hier
eine Konvergenzobergrenze erhalten möchten. Nur so können wir den
Konvergenzring zwischen 1 und 2 erreichen. Würden wir hier ebenfalls in eine
geometrische Reihe entwickeln\footnote{
\[
	\frac1{z-2}
	=
	\frac 1z \frac1{1-\frac2z}
	=
	\frac 1z \sum_{k=0}^\infty \del{\frac2z}^k
	=
	\frac 12 \sum_{k=1}^\infty \del{\frac2z}^{k}
\]
}, hätten wir im Konvergenzring $B_{2, \infty}(0)$ entwickelt.

Wir entwickeln in eine Reihe:
\[
	g(z) = \frac{1}{z-2}
	\implies
	g^{(n)}(z) = (-1)^n \frac{1}{(z-2)^n}
	\implies
	g^{(n)}(0) = \del{\half}^n
	\implies
	g(z) = \sum_{n=0}^\infty \del{\half}^n z^n
\]

Diese Reihe konvergiert für alle $\abs z < 2$.

Beide Summen schreiben wir zusammen:
\[
	\frac 12 \sum_{k=1}^\infty \del{\frac2z}^{k}
	+ \sum_{n=0}^\infty \del{\half}^n z^n
	=
	\frac 12 \sum_{n=-\infty}^{-1} \del{\half}^{n} z^n
	+ \sum_{n=0}^\infty \del{\half}^n z^n
\]

\subsection{zweite Funktion}

Es soll
\[
	f(z) = \frac{4z-z^2}{\del{z^2-4}\del{z+1}}
\]
entwickelt werden. Wir führen eine Partialbruchzerlegung durch:
\[
	f(z) = - \frac32\frac1{z+2} + \frac13\frac1{z-2} - \frac53\frac1{z+1}
\]

Zu den einzelnen Summanden bilden wir die Maclaurinreihe:
\begin{enumerate}
	\item
		\[
			- \frac32\frac1{z+2}
			= - \frac 32 \sum_{n=0}^\infty \del{-\frac 12}^n z^n
		\]

	\item
		\[
			\frac13\frac1{z-2}
			= \frac 13 \sum_{n=0}^\infty \del{\frac 12}^n z^n
		\]

	\item
		\[
			\frac53\frac1{z+1}
			= - \frac 53 \sum_{n=0}^\infty \del{-1}^n z^n
		\]
\end{enumerate}

Zu jedem der Summanden bilden wir außerdem noch eine Art geometrische Reihe:
\begin{enumerate}
	\item
		\[
			- \frac 32 \frac{1}{z+2}
			= - \frac 32 \frac 1z \frac{1}{1+\frac 2z}
			= - \frac 32 \frac 1z \sum_{n=0}^\infty \del{-\frac 2z}^n
			= 3 \sum_{n=1}^\infty \del{-\frac 2z}^n
			= 3 \sum_{n=-\infty}^{-1} \del{-\frac 12}^n z^n
		\]

	\item
		\[
			\frac 13 \frac{1}{z-2}
			= \frac 13 \frac 1z \sum_{n=0}^{\infty} \del{\frac 2z}^n
			= \frac 16 \frac 1z \sum_{n=-\infty}^{-1} \del{\half}^n z^n
		\]

	\item
		\[
			- \frac 53 \frac{1}{z+1}
			= - \frac 53 \frac 1z \sum_{n=0}^\infty \del{-\frac 1z}^n
			= \frac 53 \sum_{n=1}^\infty \del{-\frac 1z}^n
			= \frac 53 \sum_{n=-\infty}^{-1} (-1)^n z^n
		\]
\end{enumerate}

Nun müssen wir diese Teile geschickt kombinieren, je nach gewünschtem
Konvergenzring. Wir beginnen mit $B_{1,2}(0)$. Dazu brauchen wir die geometrische Reihe des Termes mit Polstelle bei $\abs z = 1$ und die Maclaurinreihen der Terme mit Polstelle bei $\abs z = 2$. Zusammen also:
\[
	f(z) \restriction B_{1,2}(0)
	=
	- \frac 32 \sum_{n=0}^\infty \del{-\frac 12}^n z^n
	+ \frac 13 \sum_{n=0}^\infty \del{\frac 12}^n z^n
	+ \frac 53 \sum_{n=-\infty}^{-1} (-1)^n z^n
\]

Für $B_{2,\infty}(0)$ benötigen wir nur die geometrischen Reihen von allen
Summanden, da nur diese für beliebig große Summanden gelten:
\[
	f(z) \restriction B_{2,\infty}(0)
	=
	3 \sum_{n=-\infty}^{-1} \del{-\frac 12}^n z^n
	+ \frac 16 \frac 1z \sum_{n=-\infty}^{-1} \del{\half}^n z^n
	+ \frac 53 \sum_{n=-\infty}^{-1} (-1)^n z^n
\]

Und für $B_{0,1}(0)$ brauchen wir nur die Maclaurinreihen, da diese nur für
kleine $\abs z$ gültig sind:
\[
	f(z) \restriction B_{0,1}(0)
	=
	- \frac 32 \sum_{n=0}^\infty \del{-\frac 12}^n z^n
	+ \frac 13 \sum_{n=0}^\infty \del{\frac 12}^n z^n
	- \frac 53 \sum_{n=0}^\infty \del{-1}^n z^n
\]

%%%%%%%%%%%%%%%%%%%%%%%%%%%%%%%%%%%%%%%%%%%%%%%%%%%%%%%%%%%%%%%%%%%%%%%%%%%%%%%
%                             Konvergenzbereiche                              %
%%%%%%%%%%%%%%%%%%%%%%%%%%%%%%%%%%%%%%%%%%%%%%%%%%%%%%%%%%%%%%%%%%%%%%%%%%%%%%%

\section{Konvergenzbereiche}
\label 4

\subsection{erste Reihe}

Wir zerlegen die Reihe in positiven und negativen Teil:
\[
	\sum_{k=0}^\infty \frac{z^k}{k!} + \sum_{k=-\infty}^{-1} \frac{z^k}{\abs k!}
\]

Der Konvergenzradius $R$ der ersten Reihe ist $\infty$, da es die Exponentialfunktion ist:
\[
	\frac 1R
	= \limsup_{k\to\infty} \sqrt[k]{\frac1{k!}}
	= 0
\]

Für den zweiten Summanden ist der Konvergenzradius $r = 0$:
\[
	r
	= \limsup_{k\to\infty} \sqrt[k]{\frac1{k!}}
	= 0
\]

Da wir nur innerhalb des Kreisrings sicher Konvergenz haben, muss $0 < \abs z <
\infty$ gelten.

\subsection{zweite Reihe}

Die Koeffizienten der zweiten Reihe sind:
\[
	a_k = 2 \sech(ak)
\]

Wir bestimmen die Konvergenzradien:
\[
	\frac 1R
	= \limsup_{k\to\infty} {\sqrt[k]{\eup^{ak} + \eup^{-ak}}}
	\geq \limsup_{k\to\infty} {\sqrt[k]{\eup^{ak}}}
	= \eup^{a}
\]

Analog für $r$. Wir haben $r \geq \eup^a$ als auch $R \leq \eup^{-a}$. Die $z$, für die die Reihe konvergiert, sind also:
\[
	\eup^a < \abs z < \eup^{-a}
\]

%\bibliography{../../zentrale_BibTeX/Central}
%\bibliographystyle{plain}

\end{document}

% vim: spell spelllang=de
