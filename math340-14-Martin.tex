% Copyright © 2012-2013 Martin Ueding <dev@martin-ueding.de>
%
\input{header.tex}

\usepackage{scrpage2}
\usepackage{tikz}
\usetikzlibrary{decorations.markings}

\newcommand{\themodul}{math340}
\newcommand{\thegruppe}{Gruppe 16 -- Malte Lackmann}
\newcommand{\theuebung}{14}

\pagestyle{scrheadings}

\automark{section}
\cfoot{\footnotesize{\thegruppe}}
\chead{}
\ifoot{\footnotesize{Ueding, Manz, Lemmer}}
\ihead{\themodul{} -- Übung \theuebung}
\ofoot{\footnotesize{Seite \thepage\ / \pageref{LastPage}}}
\ohead{\rightmark}
\setheadsepline{.4pt}


\def\thesection{\theuebung.\arabic{section}}
\def\thesubsection{\thesection\alph{subsection}}

\title{\themodul{} -- Übung \theuebung \\ \vspace{0.5cm} \large{\thegruppe}}

\author{
	Martin Ueding \\ \small{\href{mailto:mu@uni-bonn.de}{mu@uni-bonn.de}}
	\and
	Paul Manz
	\and
	Lino Lemmer
}

\begin{document}

\maketitle

\begin{table}[h]
	\centering
	\begin{tabular}{l|c|c|c|c|c}
		Aufgabe
		& \ref 1
		& \ref 2
		& \ref 3
		& \ref 4
		& $\sum$ \\
		\hline
		Punkte
		& \punkte / 4
		& \punkte / 4
		& \punkte / 4
		& \punkte / 2
		& \punkte / 14
	\end{tabular}
\end{table}

%%%%%%%%%%%%%%%%%%%%%%%%%%%%%%%%%%%%%%%%%%%%%%%%%%%%%%%%%%%%%%%%%%%%%%%%%%%%%%%
%                         Hauptzweig des Logarithmus                          %
%%%%%%%%%%%%%%%%%%%%%%%%%%%%%%%%%%%%%%%%%%%%%%%%%%%%%%%%%%%%%%%%%%%%%%%%%%%%%%%

\section{Hauptzweig des Logarithmus}
\label 1

\subsection{$z_0 = 1$}

Wir entwickeln um $z_0 = 1$. Die Ableitungen sind, wie im Reellen:
\[
	f^{(n)} (-1)^{n+1} x^{-n}
\]

\textcolor{red}{
	Das sollte so nicht sein, vor allem, weil dann auch im weiteren Verlauf ein
	unendlicher Konvergenzradius vorkommen kann. Da $z = 0$ allerdings eine
	Polstelle ist, und die Funktion nur auf $\C^-$ definiert ist, sollte es
	Probleme geben.
}

Die Reihe ist mit $f^{(n)}(1) = (-1)^{n+1}$:
\[
	f(z) = \sum_{n=0}^\infty \frac{(-1)^{n+1}}{n!} \del{z - z_0}^n
\]

\subsection{$z_0 = -1 + \iup$}

\fehlt

%%%%%%%%%%%%%%%%%%%%%%%%%%%%%%%%%%%%%%%%%%%%%%%%%%%%%%%%%%%%%%%%%%%%%%%%%%%%%%%
%                            holomorphe Funktionen                            %
%%%%%%%%%%%%%%%%%%%%%%%%%%%%%%%%%%%%%%%%%%%%%%%%%%%%%%%%%%%%%%%%%%%%%%%%%%%%%%%

\stepcounter{section}
\section{holomorphe Funktionen}
\label 2

\subsection{erste Funktion}

\fehlt

\addtocounter{subsection}{2}
\subsection{vierte Funktion}

\fehlt

%%%%%%%%%%%%%%%%%%%%%%%%%%%%%%%%%%%%%%%%%%%%%%%%%%%%%%%%%%%%%%%%%%%%%%%%%%%%%%%
%                        Entwicklung in Laurentreihen                         %
%%%%%%%%%%%%%%%%%%%%%%%%%%%%%%%%%%%%%%%%%%%%%%%%%%%%%%%%%%%%%%%%%%%%%%%%%%%%%%%

\section{Entwicklung in Laurentreihen}
\label 3

\subsection{erste Funktion}

Wir führen eine Partialbruchzerlegung durch:
\[
	f(z) = \frac3{(z+1)(z-2)} = -\frac1{z+1} + \frac1{z-2}
\]

Wir beginnen mit dem ersten Summanden und schreiben diesen als geometrische
Reihe:
\[
	-\frac1{z+1}
	=
	- \frac 1z \frac1{1+\frac 1z}
	=
	- \frac 1z \sum_{k=0}^\infty \del{\frac 1z}^k
	- \sum_{k=1}^\infty \del{\frac 1z}^k
\]

Dabei müssen wir als Einschränkung $\abs z > 1$ einfügen, da die Reihe sonst nicht konvergiert.

Den zweiten Summanden schreiben wir analog um:
\[
	\frac1{z-2}
	=
	\frac 1z \frac1{1-\frac2z}
	=
	\frac 1z \sum_{k=0}^\infty \del{\frac2z}^k
	=
	\frac 12 \sum_{k=1}^\infty \del{\frac2z}^{k}
\]

Dort erhalten wir $\abs z > 2$, damit die Reihe konvergiert.

Beide Summen schreiben wir zusammen:
\[
	\sum_{k=1}^\infty \del{ \frac 12 \del{\frac2z}^{k} - \del{\frac 1z}^k}
	=
	\sum_{k=-\infty}^{-1} \del{ \frac 12 \del{\frac z2}^{k} - z^k}
	=
	\sum_{k=-\infty}^{-1} \del{\del{\frac 12}^{k+1} - 1} z^k
\]

Da wir allerdings $\abs z > 1$ und $\abs z > 2$ erhalten haben, haben wir wohl
eine Reihe im Kreisring $B_{2, \infty}(0)$ erzeugt. Gefragt war $B_{1, 2}(0)$.

\fehlt

\subsection{zweite Funktion}

\fehlt

%%%%%%%%%%%%%%%%%%%%%%%%%%%%%%%%%%%%%%%%%%%%%%%%%%%%%%%%%%%%%%%%%%%%%%%%%%%%%%%
%                             Konvergenzbereiche                              %
%%%%%%%%%%%%%%%%%%%%%%%%%%%%%%%%%%%%%%%%%%%%%%%%%%%%%%%%%%%%%%%%%%%%%%%%%%%%%%%

\section{Konvergenzbereiche}
\label 4

\subsection{erste Reihe}

Wir zerlegen die Reihe in positiven und negativen Teil:
\[
	\sum_{k=0}^\infty \frac{z^k}{k!} + \sum_{k=-\infty}^{-1} \frac{z^k}{\abs k!}
\]

Der Konvergenzradius $R$ der ersten Reihe ist $\infty$, da es die Exponentialfunktion ist:
\[
	\frac 1R
	= \limsup_{k\to\infty} \frac1{\sqrt[k]{k!}}
	= 0
\]

Für den zweiten Summanden ist der Konvergenzradius $r = 0$:
\[
	r
	= \limsup_{k\to\infty} \frac1{\sqrt[k]{k!}}
	= 0
\]

Da wir nur innerhalb des Kreisrings sicher Konvergenz haben, muss $0 < \abs z <
\infty$ gelten.

\subsection{zweite Reihe}

Die Koeffizienten der zweiten Reihe sind:
\[
	a_k = 2 \sech(ak)
\]

Wir bestimmen die Konvergenzradien:
\[
	\frac 1R
	= \limsup_{k\to\infty} {\sqrt[k]{\eup^{ak} + \eup^{-ak}}}
	\geq \limsup_{k\to\infty} {\sqrt[k]{\eup^{ak}}}
	= \eup^{a}
\]

Analog für $r$. Wir haben $r \geq \eup^a$ als auch $R \leq \eup^{-a}$. Die $z$, für die die Reihe konvergiert, sind also:
\[
	\eup^a < \abs z < \eup^{-a}
\]

%\bibliography{../../zentrale_BibTeX/Central}
%\bibliographystyle{plain}

\end{document}

% vim: spell spelllang=de
