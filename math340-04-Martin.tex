% Copyright © 2012 Martin Ueding <dev@martin-ueding.de>
%
\documentclass[11pt, ngerman, fleqn]{article}

\usepackage[a4paper, left=3cm, right=2cm, top=2cm, bottom=2cm]{geometry}
\usepackage[activate]{pdfcprot}
\usepackage[cdot, squaren]{SIunits}
\usepackage[iso]{isodate}
\usepackage[parfill]{parskip}
\usepackage[T1]{fontenc}
\usepackage[utf8]{inputenc}
\usepackage{amsmath}
\usepackage{amsthm}
\usepackage{babel}
\usepackage{color}
\usepackage{commath}
\usepackage{epstopdf}
\usepackage{fancyhdr}
\usepackage{graphicx}
\usepackage{hyperref}
\usepackage{lastpage}
\usepackage{setspace}
\usepackage{tikz}

\usepackage[charter, greekuppercase=italicized]{mathdesign}

\definecolor{darkblue}{rgb}{0,0,.5}
\definecolor{darkgreen}{rgb}{0,.5,0}

\hypersetup{
	breaklinks=false,
	citecolor=darkgreen,
	colorlinks=true,
	linkcolor=black,
	menucolor=black,
	urlcolor=darkblue,
}

\setlength{\columnsep}{2cm}

\DeclareMathOperator{\arcsinh}{arsinh}
\DeclareMathOperator{\arsinh}{arsinh}
\DeclareMathOperator{\asinh}{arsinh}
\DeclareMathOperator{\card}{card}
\DeclareMathOperator{\diam}{diam}

\newcommand{\dalambert}{\mathop{{}\Box}\nolimits}
\newcommand{\divergence}[1]{\inner{\vnabla}{#1}}
\newcommand{\ee}{\mathrm e}
\newcommand{\emesswert}{\del{\messwert \pm \messwert}}
\newcommand{\e}[1]{\cdot 10^{#1}}
\newcommand{\fehlt}{\textcolor{red}{Hier fehlen noch Inhalte.}}
\newcommand{\half}{\frac 12}
\newcommand{\ii}{\mathrm i}
\newcommand{\inner}[2]{\left\langle #1, #2 \right\rangle}
\newcommand{\laplace}{\mathop{{}\Deltaup}\nolimits}
\newcommand{\messwert}{\textcolor{blue}{\square}}
\newcommand{\punkte}{\textcolor{white}{xxxxx}}
\newcommand{\tens}[1]{\boldsymbol{#1}}
\newcommand{\vnabla}{\vec \nabla}
\renewcommand{\vec}[1]{\boldsymbol{#1}}

\newcommand{\themodul}{math340}
\newcommand{\thegruppe}{Gruppe 16 -- Malte Lackmann}
\newcommand{\theuebung}{4}

\pagestyle{fancy}

\fancyfoot[C]{\footnotesize{\thegruppe}}
\fancyfoot[L]{\footnotesize{Ueding, Manz, Lemmer}}
\fancyfoot[R]{\footnotesize{Seite \thepage\ / \pageref{LastPage}}}
\fancyhead[L]{\themodul{} -- Übung \theuebung}

\setcounter{section}{0}

\def\thesection{\theuebung.\arabic{section}}
\def\thesubsection{\thesection\alph{subsection}}

\title{\themodul{} -- Übung \theuebung \\ \vspace{0.5cm} \large{\thegruppe}}

\author{Martin Ueding \\ \small{\href{mailto:mu@uni-bonn.de}{mu@uni-bonn.de}} \and Paul Manz \and Lino Lemmer}

\begin{document}

\maketitle

\begin{table}[h]
	\centering
	\begin{tabular}{l|c|c|c|c}
		Aufgabe & \ref{1} & \ref{2} & \ref{3} & $\sum$   \\
		\hline
		Punkte & \punkte / 8 & \punkte / 2 & \punkte / 8 & \punkte / 18
	\end{tabular}
\end{table}

Ich schreibe wieder $\ddot w := \pd[2] wt$.

%%%%%%%%%%%%%%%%%%%%%%%%%%%%%%%%%%%%%%%%%%%%%%%%%%%%%%%%%%%%%%%%%%%%%%%%%%%%%%%
%                                    RAWA                                     %
%%%%%%%%%%%%%%%%%%%%%%%%%%%%%%%%%%%%%%%%%%%%%%%%%%%%%%%%%%%%%%%%%%%%%%%%%%%%%%%

\section{RAWA}
\label{1}

Als Ansatz benute ich $u(x, y, t) = v(x, y) w(t)$. Dies setze ich in die Differentialgleichung ein:
\[
	- \dpd[2] vx w - \dpd[2] vy w + \frac{1}{c^2} v \ddot w = 0
\]

Teilen durch $vw$ ergibt:
\[
	\underbrace{- \frac 1v \dpd[2] vx - \frac 1v \dpd[2] vy}_{=: \alpha} + \underbrace{\frac{1}{c^2} \frac 1w \ddot w}_{=: -\alpha} = 0
\]

Daraus folgen die zwei Gleichungen, die unabhängig sind:
\[
	- v \alpha = \dpd[2] vx + \dpd[2] vy
	\quad \wedge \quad
	- \alpha c^2 w = \ddot w
\]

Falls $\alpha$ positiv ist, wird die Differentialgleichung für $w$ durch Kosinus und Sinus gelöst. Ist $w = 0$, ist dies eine lineare Funktion. Falls $w < 0$, wird dies durch eine Exponentialfunktion gelöst. Mit den Randbedingungen kann allerdings $\alpha$ nur positiv sein. Daraus folgt für die Integralbasis für $w$:
\[
	B_w = \set{
		\cos\del{\sqrt{\alpha} c t}, \sin\del{\sqrt{\alpha} c t}
	}
\]

Für $v$ setze ich nun $GH$ ein und erhalte:
\[
	\frac{G''}{G} + \alpha = - \frac{H''}{H} =: \beta
	\quad\Rightarrow\quad
	\frac{G''}{G} = - \del{\alpha - \beta}
	,\quad
	\frac{H''}{H} = - \beta
\]

Somit erhalte ich zwei weitere Gleichungen, die ich einfach unter Benutzung der
gleichen Fallunterscheidung lösen kann. Ich erhalte als Integralbasen:
\[
	B_G = \set{
	\cos\del{\sqrt{\alpha - \beta} x}, \sin\del{\sqrt{\alpha - \beta} t}
	}
	, \quad
	B_H = \set{
		\cos\del{\sqrt{\beta} y}, \sin\del{\sqrt{\beta} y}
	}
\]

Die Funktion $u$ ist nun eine Linearkombination aus Produkten, die jeweils aus
einer Basisfunktion bestehen. Da allerdings die Randbedingungen festlegen, dass
am Rand die Funktion $0$ sein soll, können nur Sinusterme vorkommen.

Außerdem muss dann $\sqrt{\alpha - \beta} a = m \pi$ und $\sqrt{\beta} b = n \pi$ ($m,
n \in \mathbb N$) gelten, damit bei $x=a$, $y=b$ und $t=0$ die Funktion
ebenfalls $0$ ist. Daraus folgt:
\[
	\beta = \frac{n^2 \pi^2}{b^2}
	,\quad
	\alpha = \frac{m^2 \pi^2}{a^2} + \frac{n^2 \pi^2}{b^2}
\]

Somit muss also die Funktion sein:
\[
	u(x, y, t)
	= \sum_{m=1}^\infty \sum_{n=1}^\infty a_m a_n \sin\del{\frac{m\pi}a x}
	\sin\del{\frac{n\pi}b y}
	\sin\del{\sqrt{\frac{m^2\pi^2}{a^2} + \frac{n^2\pi^2}{b^2}} c t}
\]

Diese Funktion erfüllt für beliebige $a_m$ und $a_n$ die Anfangsbedingung
für $u(x, y, 0)$, und auch auch alle Randbedingungen.

Nun muss ich noch die $a_m$ und $a_n$ so wählen, dass die Bedingung für $\dot
u$ erfüllt ist. Für $t=0$ soll für $\dot u$ gelten:
\[
	\dot u(x, y, 0) =
	\sum_{m=1}^\infty \sum_{n=1}^\infty \underbrace{a_m a_n \sqrt{\frac{m^2\pi^2}{a^2} + \frac{n^2\pi^2}{b^2}} c}_{C_{mn}} \sin\del{\frac{m\pi}a x}
	\sin\del{\frac{n\pi}b y}
	= xy(x-a)(y-b)
\]

Das ist letztlich eine zweidimensionale Fourierreihe. Die Koeffizenten $a_m$
und $a_n$ können durch Projektion bestimmt werden.

Aus dem vorherigen Aufgabenblatt wissen wir für die Parabel, die ungerade fortgesetzt wird:
\[
	x(x-a) = \sum_{n=1}^\infty \frac{4a^2}{\pi^3 n^3} \del{(-1)^n -1} \sin\del{\frac{\pi n}a x}
\]

Somit können wir diese doppelte Parabel als Fourier-Reihe schreiben:
\[
	x(x-a)y(y-b)
	= \sum_{m=1}^\infty \sum_{n=1}^\infty \underbrace{\frac{16 a^2 b^2}{\pi^6 m^3 n^3} \del{(-1)^m -1} \del{(-1)^n -1}}_{C_{mn}} \sin\del{\frac{m \pi}{a} x} \sin\del{\frac{n \pi}{b} x}
\]

Jetzt wählen wir die Vorfaktoren $a_m$ und $a_n$ einfach so, dass sie den Koeffizenten in der doppelten Parabel entsprechen.

Die generelle Lösung $u$ ist also:
\[
	u =
	\sum_{m=1}^\infty \sum_{n=1}^\infty
	\frac{16 a^2 b^2}{\pi^6 m^3 n^3} \frac{ \del{(-1)^m -1} \del{(-1)^n -1}}
	{\sqrt{\frac{m^2\pi^2}{a^2} + \frac{n^2\pi^2}{b^2}} c}
	\sin\del{\frac{m \pi}{a} x} \sin\del{\frac{n \pi}{b} x}
	\sin\del{\sqrt{\frac{m^2\pi^2}{a^2} + \frac{n^2\pi^2}{b^2}} c t}
\]

%%%%%%%%%%%%%%%%%%%%%%%%%%%%%%%%%%%%%%%%%%%%%%%%%%%%%%%%%%%%%%%%%%%%%%%%%%%%%%%
%                              Laplacegleichung                               %
%%%%%%%%%%%%%%%%%%%%%%%%%%%%%%%%%%%%%%%%%%%%%%%%%%%%%%%%%%%%%%%%%%%%%%%%%%%%%%%

\stepcounter{section}
\section{Laplacegleichung}
\label{2}

Es soll bestimmt werden, für welche $\vec a$ folgende Gleichung durch $\phi\del{\inner{\vec x}{\vec a} + ct}$ gelöst wird:
\[
	- \laplace u - \frac1{c^2} \dpd[2] ut = 0
\]

Die Ableitungen der Funktion $\phi$ sind:
\[
	\dpd[2]\phi t = c^2 \ddot \phi\del{\inner{\vec x}{\vec a} + ct}
\]

Der Gradient:
\[
	\vnabla \phi = \phi'\del{\inner{\vec x}{\vec a} + ct} \vec a
\]

Laplace:
\[
	\laplace \phi = \phi''\del{\inner{\vec x}{\vec a} + ct} \inner{\vec a}{\vec a}
\]

Eingesetzt in die Differentialgleichung ergibt sich dann die Bedingung:
\[
	a^2 = 1
\]

Die Vektoren $\vec a$ müssen also normiert sein.

%%%%%%%%%%%%%%%%%%%%%%%%%%%%%%%%%%%%%%%%%%%%%%%%%%%%%%%%%%%%%%%%%%%%%%%%%%%%%%%
%                               Cauchy-Probleme                               %
%%%%%%%%%%%%%%%%%%%%%%%%%%%%%%%%%%%%%%%%%%%%%%%%%%%%%%%%%%%%%%%%%%%%%%%%%%%%%%%

\section{Cauchy-Probleme}
\label{3}

\stepcounter{subsection}

\subsection{die Zweite}

Ich benutze die Lösungsformel:
\[
	u\del{\vec x, t} = \dpd{}t \del{
		\frac{1}{2 \pi} \int_{\abs{\vec y - \vec x} \leq t} \dif \vec \tau_y \frac{u_0\del{\vec y}}{\sqrt{t^2 - \abs{\vec y - \vec x}^2}}
	}
	+
	\frac{1}{2 \pi} \int_{\abs{\vec y - \vec x} \leq t} \dif \vec \tau_y \frac{u_1\del{\vec y}}{\sqrt{t^2 - \abs{\vec y - \vec x}^2}}
\]

Das Integral ist auf einer Einheitskreisfläche. Daher gilt: \( \dif \vec \tau_y = \dif y_1 \dif y_2 \). Die Grenzen sind:
\[
	x_1 - t \leq y_1 \leq x_1 + t
	,\quad
	x_2 - \sqrt{t^2-\del{y_1 - x_1}^2} \leq y_2 \leq x_2 + \sqrt{t^2-\del{y_1 - x_1}^2}
\]

Mit diesen Grenzen wird die Lösungsformel zu:
\begin{align*}
	u\del{\vec x, t} &= \dpd{}t \del{
		\frac{1}{2 \pi}
		\int_{x_1 - t}^{x_1 + t} \dif y_1
		\int_{x_2 - \sqrt{t^2-\del{y_1 - x_1}^2}}^{x_2 + \sqrt{t^2-\del{y_1 - x_1}^2}} \dif y_2
		\frac{u_0\del{\vec y}}{\sqrt{t^2 - \abs{\vec y - \vec x}^2}}
	} \\
	&\quad+
	\frac{1}{2 \pi}
	\int_{x_1 - t}^{x_1 + t} \dif y_1
	\int_{x_2 - \sqrt{t^2-\del{y_1 - x_1}^2}}^{x_2 + \sqrt{t^2-\del{y_1 - x_1}^2}} \dif y_2
	\frac{u_1\del{\vec y}}{\sqrt{t^2 - \abs{\vec y - \vec x}^2}}
\end{align*}

Dort setze ich jetzt die Funktion ein:
\begin{align*}
	u\del{\vec x, t} &= \dpd{}t \del{
		\frac{1}{2 \pi}
		\int_{x_1 - t}^{x_1 + t} \dif y_1
		\int_{x_2 - \sqrt{t^2-\del{y_1 - x_1}^2}}^{x_2 + \sqrt{t^2-\del{y_1 - x_1}^2}} \dif y_2
		\frac{y_1^2 + y_2^2}{\sqrt{t^2 - \abs{\vec y - \vec x}^2}}
	} \\
	&\quad+
	\frac{1}{2 \pi}
	\int_{x_1 - t}^{x_1 + t} \dif y_1
	\int_{x_2 - \sqrt{t^2-\del{y_1 - x_1}^2}}^{x_2 + \sqrt{t^2-\del{y_1 - x_1}^2}} \dif y_2
	\frac{1}{\sqrt{t^2 - \abs{\vec y - \vec x}^2}} \\
	&= \frac 43 t^2 + \frac 13 \del{2t^2 + 3 \del{x_1^2 + x_2^2}} + t
\end{align*}

Diese Lösung erfüllt die Differentialgleichung und auch die Anfangsbedingungen.

%\bibliography{../../zentrale_BibTeX/Central}
%\bibliographystyle{plain}

\end{document}
