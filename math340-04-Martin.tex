% Copyright © 2012 Martin Ueding <dev@martin-ueding.de>
%
\documentclass[11pt, ngerman, fleqn]{article}

\usepackage[a4paper, left=3cm, right=2cm, top=2cm, bottom=2cm]{geometry}
\usepackage[activate]{pdfcprot}
\usepackage[cdot, squaren]{SIunits}
\usepackage[iso]{isodate}
\usepackage[parfill]{parskip}
\usepackage[T1]{fontenc}
\usepackage[utf8]{inputenc}
\usepackage{amsmath}
\usepackage{amsthm}
\usepackage{babel}
\usepackage{color}
\usepackage{commath}
\usepackage{epstopdf}
\usepackage{fancyhdr}
\usepackage{graphicx}
\usepackage{hyperref}
\usepackage{lastpage}
\usepackage{setspace}
\usepackage{tikz}

\usepackage[charter, greekuppercase=italicized]{mathdesign}

\definecolor{darkblue}{rgb}{0,0,.5}
\definecolor{darkgreen}{rgb}{0,.5,0}

\hypersetup{
	breaklinks=false,
	citecolor=darkgreen,
	colorlinks=true,
	linkcolor=black,
	menucolor=black,
	urlcolor=darkblue,
}

\setlength{\columnsep}{2cm}

\DeclareMathOperator{\arcsinh}{arsinh}
\DeclareMathOperator{\arsinh}{arsinh}
\DeclareMathOperator{\asinh}{arsinh}
\DeclareMathOperator{\card}{card}
\DeclareMathOperator{\diam}{diam}

\newcommand{\dalambert}{\mathop{{}\Box}\nolimits}
\newcommand{\divergence}[1]{\inner{\vnabla}{#1}}
\newcommand{\ee}{\mathrm e}
\newcommand{\emesswert}{\del{\messwert \pm \messwert}}
\newcommand{\e}[1]{\cdot 10^{#1}}
\newcommand{\fehlt}{\textcolor{red}{Hier fehlen noch Inhalte.}}
\newcommand{\half}{\frac 12}
\newcommand{\ii}{\mathrm i}
\newcommand{\inner}[2]{\left\langle #1, #2 \right\rangle}
\newcommand{\laplace}{\mathop{{}\Deltaup}\nolimits}
\newcommand{\messwert}{\textcolor{blue}{\square}}
\newcommand{\punkte}{\textcolor{white}{xxxxx}}
\newcommand{\tens}[1]{\boldsymbol{#1}}
\newcommand{\vnabla}{\vec \nabla}
\renewcommand{\vec}[1]{\boldsymbol{#1}}

\newcommand{\themodul}{math340}
\newcommand{\thegruppe}{Gruppe 16 -- Malte Lackmann}
\newcommand{\theuebung}{4}

\pagestyle{fancy}

\fancyfoot[C]{\footnotesize{\thegruppe}}
\fancyfoot[L]{\footnotesize{Ueding, Manz, Lemmer}}
\fancyfoot[R]{\footnotesize{Seite \thepage\ / \pageref{LastPage}}}
\fancyhead[L]{\themodul{} -- Übung \theuebung}

\setcounter{section}{0}

\def\thesection{\theuebung.\arabic{section}}
\def\thesubsection{\thesection\alph{subsection}}

\title{\themodul{} -- Übung \theuebung \\ \vspace{0.5cm} \large{\thegruppe}}

\author{Martin Ueding \\ \small{\href{mailto:mu@uni-bonn.de}{mu@uni-bonn.de}} \and Paul Manz \and Lino Lemmer}

\begin{document}

\maketitle

\begin{table}[h]
	\centering
	\begin{tabular}{l|c|c|c|c}
		Aufgabe & \ref{1} & \ref{2} & \ref{3} & $\sum$   \\
		\hline
		Punkte & \punkte / 8 & \punkte / 2 & \punkte / 8 & \punkte / 18
	\end{tabular}
\end{table}

Ich schreibe wieder $\ddot w := \pd[2] wt$.

%%%%%%%%%%%%%%%%%%%%%%%%%%%%%%%%%%%%%%%%%%%%%%%%%%%%%%%%%%%%%%%%%%%%%%%%%%%%%%%
%                                    RAWA                                     %
%%%%%%%%%%%%%%%%%%%%%%%%%%%%%%%%%%%%%%%%%%%%%%%%%%%%%%%%%%%%%%%%%%%%%%%%%%%%%%%

\section{RAWA}
\label{1}

Als Ansatz benute ich $u(x, y, t) = v(x, y) w(t)$. Dies setze ich in die Differentialgleichung ein:
\[
	- \dpd[2] vx w - \dpd[2] vy w + \frac{1}{c^2} v \ddot w = 0
\]

Teilen durch $vw$ ergibt:
\[
	\underbrace{- \frac 1v \dpd[2] vx - \frac 1w \dpd[2] vy}_{=: -\alpha} + \underbrace{\frac{1}{c^2} \frac 1w \ddot w}_{=: \alpha} = 0
\]

Daraus folgen die zwei Gleichungen, die unabhängig sind:
\[
	v \alpha = \dpd[2] vx + \dpd[2] vy
	\quad \wedge \quad
	\alpha c^2 w = \ddot w
\]

Daraus folgt für die Integralbasis für $w$:
\[
	B_w = \set{
		\cos\del{\alpha c^2 t}, \sin\del{\alpha c^2 t}
	}
\]

Für $v$ setze ich nun $GH$ ein und erhalte:
\[
	\frac{G''}{G} - \alpha = \frac{H''}{H} =: \beta
\]

Somit erhalte ich zwei weitere Gleichungen, die ich einfach lösen kann. Ich erhalte als Integralbasen:
\[
	B_G = \set{
		\cos\del{\del{\alpha + \beta} x}, \sin\del{\del{\alpha + \beta} t}
	}
	, \quad
	B_H = \set{
		\cos\del{\beta y}, \sin\del{\beta y}
	}
\]

Die Funktion $u$ ist nun eine Linearkombination aus Produkten, die jeweils aus
einer Basisfunktion bestehen. Da allerdings die Randbedingungen festlegen, dass
am Rand die Funktion $0$ sein soll, können nur Sinusterme vorkommen.

Außerdem muss dann $\del{\alpha + \beta} a = n \pi$ und $\beta b = n \pi$
gelten, damit bei $x=a$, $y=b$ und $t=0$ die Funktion ebenfalls $0$ ist.

Somit muss also die Funktion sein:
\[
	u(x, y, t)
	= \sin\del{\frac{n\pi}a x}
	\sin\del{\frac{n\pi}b y}
	\sin\del{\del{\frac{n\pi}a - \frac{n\pi}b} c^2 t}
\]

Diese Funktion erfüllt zwar die Anfangsbedingung für $u(x, y, 0)$, und auch
auch alle Randbedingungen, allerdings scheitert es an $\dot u$. Die Ableitung
ist einfach nur die beiden Sinusterme und innere Ableitung des Zeitterms. Das
geforderte Polynom erhalte ich jedoch nicht.

%%%%%%%%%%%%%%%%%%%%%%%%%%%%%%%%%%%%%%%%%%%%%%%%%%%%%%%%%%%%%%%%%%%%%%%%%%%%%%%
%                              Laplacegleichung                               %
%%%%%%%%%%%%%%%%%%%%%%%%%%%%%%%%%%%%%%%%%%%%%%%%%%%%%%%%%%%%%%%%%%%%%%%%%%%%%%%

\stepcounter{section}
\section{Laplacegleichung}
\label{2}

Es soll bestimmt werden, für welche $\vec a$ folgende Gleichung durch $\phi\del{\inner{\vec x}{\vec a} + ct}$ gelöst wird:
\[
	- \laplace u - \frac1{c^2} \dpd[2] ut = 0
\]

Die Ableitungen der Funktion $\phi$ sind:
\[
	\dpd[2]\phi t = c^2 \ddot \phi\del{\inner{\vec x}{\vec a} + ct}
\]

Der Gradient:
\[
	\vnabla \phi = \phi'\del{\inner{\vec x}{\vec a} + ct} \vec a
\]

Laplace:
\[
	\laplace \phi = \phi''\del{\inner{\vec x}{\vec a} + ct} \inner{\vec a}{\vec a}
\]

Eingesetzt in die Differentialgleichung ergibt sich dann die Bedingung:
\[
	a^2 = 1
\]

Die Vektoren $\vec a$ müssen also normiert sein.

%%%%%%%%%%%%%%%%%%%%%%%%%%%%%%%%%%%%%%%%%%%%%%%%%%%%%%%%%%%%%%%%%%%%%%%%%%%%%%%
%                               Cauchy-Probleme                               %
%%%%%%%%%%%%%%%%%%%%%%%%%%%%%%%%%%%%%%%%%%%%%%%%%%%%%%%%%%%%%%%%%%%%%%%%%%%%%%%

\section{Cauchy-Probleme}
\label{3}

\subsection{die Erste}

\fehlt

\subsection{die Zweite}

Ich benutze die Lösungsformel:
\[
	u\del{\vec x, t} = \dpd{}t \del{
		\frac{1}{2 \pi} \int_{\abs{\vec y - \vec x} \leq t} \dif \vec \tau_y \frac{u_0\del{\vec y}}{\sqrt{t^2 - \abs{\vec y - \vec x}^2}}
	}
	+
	\frac{1}{2 \pi} \int_{\abs{\vec y - \vec x} \leq t} \dif \vec \tau_y \frac{u_1\del{\vec y}}{\sqrt{t^2 - \abs{\vec y - \vec x}^2}}
\]

Das Integral ist auf einer Einheitskreisfläche. Daher gilt:
\[
	\dif \vec \tau_y = \dif y_1 \dif y_2
\]

Die Grenzen sind:
\[
	x_1 - t \leq y_1 \leq x_1 + t
	,\quad
	x_2 - \sqrt{t^2-\del{y_1 - x_1}^2} \leq y_2 \leq x_2 + \sqrt{t^2-\del{y_1 - x_1}^2}
\]

Mit diesen Grenzen wird die Lösungsformel zu:
\begin{align*}
	u\del{\vec x, t} &= \dpd{}t \del{
		\frac{1}{2 \pi}
		\int_{x_1 - t}^{x_1 + t} \dif y_1
		\int_{x_2 - \sqrt{t^2-\del{y_1 - x_1}^2}}^{x_2 + \sqrt{t^2-\del{y_1 - x_1}^2}} \dif y_2
		\frac{u_0\del{\vec y}}{\sqrt{t^2 - \abs{\vec y - \vec x}^2}}
	} \\
	&\quad+
	\frac{1}{2 \pi}
	\int_{x_1 - t}^{x_1 + t} \dif y_1
	\int_{x_2 - \sqrt{t^2-\del{y_1 - x_1}^2}}^{x_2 + \sqrt{t^2-\del{y_1 - x_1}^2}} \dif y_2
	\frac{u_1\del{\vec y}}{\sqrt{t^2 - \abs{\vec y - \vec x}^2}}
\end{align*}

Dort setze ich jetzt die Funktion ein:
\begin{align*}
	u\del{\vec x, t} &= \dpd{}t \del{
		\frac{1}{2 \pi}
		\int_{x_1 - t}^{x_1 + t} \dif y_1
		\int_{x_2 - \sqrt{t^2-\del{y_1 - x_1}^2}}^{x_2 + \sqrt{t^2-\del{y_1 - x_1}^2}} \dif y_2
		\frac{y_1^2 + y_2^2}{\sqrt{t^2 - \abs{\vec y - \vec x}^2}}
	} \\
	&\quad+
	\frac{1}{2 \pi}
	\int_{x_1 - t}^{x_1 + t} \dif y_1
	\int_{x_2 - \sqrt{t^2-\del{y_1 - x_1}^2}}^{x_2 + \sqrt{t^2-\del{y_1 - x_1}^2}} \dif y_2
	\frac{1}{\sqrt{t^2 - \abs{\vec y - \vec x}^2}} \\
	&= \frac 43 t^2 + \frac 13 \del{2t^2 + 3 \del{x_1^2 + x_2^2}} + t
\end{align*}

Diese Lösung erfüllt die Differentialgleichung und auch die Anfangsbedingungen.

%\bibliography{../../zentrale_BibTeX/Central}
%\bibliographystyle{plain}

\end{document}
