% Copyright © 2012-2013 Martin Ueding <dev@martin-ueding.de>
%
% Copyright © 2012-2013 Martin Ueding <dev@martin-ueding.de>
%
\documentclass[11pt, ngerman, fleqn, DIV=15]{scrartcl}

\usepackage{graphicx}

%%%%%%%%%%%%%%%%%%%%%%%%%%%%%%%%%%%%%%%%%%%%%%%%%%%%%%%%%%%%%%%%%%%%%%%%%%%%%%%
%                                Locale, date                                 %
%%%%%%%%%%%%%%%%%%%%%%%%%%%%%%%%%%%%%%%%%%%%%%%%%%%%%%%%%%%%%%%%%%%%%%%%%%%%%%%

\usepackage{babel}
\usepackage[iso]{isodate}

%%%%%%%%%%%%%%%%%%%%%%%%%%%%%%%%%%%%%%%%%%%%%%%%%%%%%%%%%%%%%%%%%%%%%%%%%%%%%%%
%                          Margins and other spacing                          %
%%%%%%%%%%%%%%%%%%%%%%%%%%%%%%%%%%%%%%%%%%%%%%%%%%%%%%%%%%%%%%%%%%%%%%%%%%%%%%%

\usepackage[activate]{pdfcprot}
\usepackage[parfill]{parskip}
\usepackage{setspace}

\setlength{\columnsep}{2cm}

%%%%%%%%%%%%%%%%%%%%%%%%%%%%%%%%%%%%%%%%%%%%%%%%%%%%%%%%%%%%%%%%%%%%%%%%%%%%%%%
%                                    Color                                    %
%%%%%%%%%%%%%%%%%%%%%%%%%%%%%%%%%%%%%%%%%%%%%%%%%%%%%%%%%%%%%%%%%%%%%%%%%%%%%%%

\usepackage{color}

\definecolor{darkblue}{rgb}{0,0,.5}
\definecolor{darkgreen}{rgb}{0,.5,0}
\definecolor{darkred}{rgb}{.7,0,0}

%%%%%%%%%%%%%%%%%%%%%%%%%%%%%%%%%%%%%%%%%%%%%%%%%%%%%%%%%%%%%%%%%%%%%%%%%%%%%%%
%                         Font and font like settings                         %
%%%%%%%%%%%%%%%%%%%%%%%%%%%%%%%%%%%%%%%%%%%%%%%%%%%%%%%%%%%%%%%%%%%%%%%%%%%%%%%

\usepackage[charter, greekuppercase=italicized]{mathdesign}
\usepackage{beramono}
\usepackage{berasans}

% Style of vectors and tensors.
\newcommand{\tens}[1]{\boldsymbol{\mathsf{#1}}}
\renewcommand{\vec}[1]{\boldsymbol{#1}}

%%%%%%%%%%%%%%%%%%%%%%%%%%%%%%%%%%%%%%%%%%%%%%%%%%%%%%%%%%%%%%%%%%%%%%%%%%%%%%%
%                               Input encoding                                %
%%%%%%%%%%%%%%%%%%%%%%%%%%%%%%%%%%%%%%%%%%%%%%%%%%%%%%%%%%%%%%%%%%%%%%%%%%%%%%%

\usepackage[T1]{fontenc}
\usepackage[utf8]{inputenc}

%%%%%%%%%%%%%%%%%%%%%%%%%%%%%%%%%%%%%%%%%%%%%%%%%%%%%%%%%%%%%%%%%%%%%%%%%%%%%%%
%                         Hyperrefs and PDF metadata                          %
%%%%%%%%%%%%%%%%%%%%%%%%%%%%%%%%%%%%%%%%%%%%%%%%%%%%%%%%%%%%%%%%%%%%%%%%%%%%%%%

\usepackage{hyperref}
\usepackage{lastpage}

\hypersetup{
	breaklinks=false,
	citecolor=darkgreen,
	colorlinks=true,
	linkcolor=black,
	menucolor=black,
	pdfauthor={Martin Ueding},
	urlcolor=darkblue,
}

%%%%%%%%%%%%%%%%%%%%%%%%%%%%%%%%%%%%%%%%%%%%%%%%%%%%%%%%%%%%%%%%%%%%%%%%%%%%%%%
%                               Math Operators                                %
%%%%%%%%%%%%%%%%%%%%%%%%%%%%%%%%%%%%%%%%%%%%%%%%%%%%%%%%%%%%%%%%%%%%%%%%%%%%%%%

\usepackage[thinspace, squaren]{SIunits}
\usepackage{amsmath}
\usepackage{amsthm}
\usepackage{commath}

% Word like operators.
\DeclareMathOperator{\acosh}{arcosh}
\DeclareMathOperator{\arcosh}{arcosh}
\DeclareMathOperator{\arcsinh}{arsinh}
\DeclareMathOperator{\arsinh}{arsinh}
\DeclareMathOperator{\asinh}{arsinh}
\DeclareMathOperator{\card}{card}
\DeclareMathOperator{\diam}{diam}
\renewcommand{\Im}{\mathop{{}\mathrm{Im}}\nolimits}
\renewcommand{\Re}{\mathop{{}\mathrm{Re}}\nolimits}

% Special single letters.
\DeclareMathOperator{\fourier}{\mathcal{F}}
\newcommand{\C}{\ensuremath{\mathbb C}}
\newcommand{\ee}{\mathrm e}
\newcommand{\ii}{\mathrm i}
\newcommand{\N}{\ensuremath{\mathbb N}}
\newcommand{\R}{\ensuremath{\mathbb R}}
\newcommand{\Z}{\ensuremath{\mathbb Z}}

% Shape like operators.
\DeclareMathOperator{\dalambert}{\Box}
\DeclareMathOperator{\laplace}{\bigtriangleup}
\newcommand{\curl}{\vnabla \times}
\newcommand{\divergence}[1]{\inner{\vnabla}{#1}}
\newcommand{\vnabla}{\vec \nabla}

% Shortcuts
\newcommand{\ev}{\hat{\vec e}}
\newcommand{\e}[1]{\cdot 10^{#1}}
\newcommand{\half}{\frac 12}
\newcommand{\inner}[2]{\left\langle #1, #2 \right\rangle}

% Placeholders.
\newcommand{\emesswert}{\del{\messwert \pm \messwert}}
\newcommand{\fehlt}{\textcolor{darkred}{Hier fehlen noch Inhalte.}}
\newcommand{\messwert}{\textcolor{blue}{\square}}
\newcommand{\punkte}{\textcolor{white}{xxxxx}}

% Separator for equations on a single line.
\newcommand{\eqnsep}{,\quad}

%%%%%%%%%%%%%%%%%%%%%%%%%%%%%%%%%%%%%%%%%%%%%%%%%%%%%%%%%%%%%%%%%%%%%%%%%%%%%%%
%                                  Headings                                   %
%%%%%%%%%%%%%%%%%%%%%%%%%%%%%%%%%%%%%%%%%%%%%%%%%%%%%%%%%%%%%%%%%%%%%%%%%%%%%%%

\usepackage{scrpage2}

\pagestyle{scrheadings}

\automark{section}
\cfoot{\footnotesize{Seite \thepage\ / \pageref{LastPage}}}
\chead{}
\ihead{}
\ohead{\rightmark}
\setheadsepline{.4pt}

%%%%%%%%%%%%%%%%%%%%%%%%%%%%%%%%%%%%%%%%%%%%%%%%%%%%%%%%%%%%%%%%%%%%%%%%%%%%%%%
%                            Bibliography (BibTeX)                            %
%%%%%%%%%%%%%%%%%%%%%%%%%%%%%%%%%%%%%%%%%%%%%%%%%%%%%%%%%%%%%%%%%%%%%%%%%%%%%%%

\newcommand{\bibliographyfile}{../../zentrale_BibTeX/Central}


\usepackage{scrpage2}
\usepackage{tikz}

\newcommand{\themodul}{math340}
\newcommand{\thegruppe}{Gruppe 16 -- Malte Lackmann}
\newcommand{\theuebung}{12}

\pagestyle{scrheadings}

\automark{section}
\cfoot{\footnotesize{\thegruppe}}
\chead{}
\ifoot{\footnotesize{Ueding, Manz, Lemmer}}
\ihead{\themodul{} -- Übung \theuebung}
\ofoot{\footnotesize{Seite \thepage\ / \pageref{LastPage}}}
\ohead{\rightmark}
\setheadsepline{.4pt}


\def\thesection{\theuebung.\arabic{section}}
\def\thesubsection{\thesection\alph{subsection}}

\title{\themodul{} -- Übung \theuebung \\ \vspace{0.5cm} \large{\thegruppe}}

\author{
	Martin Ueding \\ \small{\href{mailto:mu@uni-bonn.de}{mu@uni-bonn.de}}
	\and
	Paul Manz
	\and
	Lino Lemmer
}

\begin{document}

\maketitle

\begin{table}[h]
	\centering
	\begin{tabular}{l|c|c|c|c|c|c}
		Aufgabe & \ref 1 & \ref 2 & \ref 3 & \ref 4 & $\sum$   \\
		\hline
		Punkte & \punkte / 4 & \punkte / 3 & \punkte / 5 & \punkte / 3 & \punkte / 15
	\end{tabular}
\end{table}

%%%%%%%%%%%%%%%%%%%%%%%%%%%%%%%%%%%%%%%%%%%%%%%%%%%%%%%%%%%%%%%%%%%%%%%%%%%%%%%
%                               Betragsfunktion                               %
%%%%%%%%%%%%%%%%%%%%%%%%%%%%%%%%%%%%%%%%%%%%%%%%%%%%%%%%%%%%%%%%%%%%%%%%%%%%%%%

\section{Betragsfunktion}
\label 1

\subsection{gerade Strecke}

Wir wählen die Parametrisierung:
\[
	z(t) = (1-t) + \ii t
\]

Damit ist das Integral:
\begin{align*}
	\int_{\gamma_1} \dif z \, f(z)
	&= \int_0^{1} \dif t \, f\del{z(t)} \dot z(t) \\
	&= \int_0^{1} \dif t \, \del{(1-t)+\ii t}\del{(1-t)-\ii t} (-1 + \ii) \\
	&= (-1 + \ii) \int_0^{\pi/2} \dif t \, \del{1 - 2t + 2t^2} \\
	&= (-1 + \ii) \frac 23
\end{align*}

\subsection{Kreisbogen}

Wir wählen als Parametrisierung:
\[
	z(t) = \ee^{\ii t}
\]

Somit ist das Integral:
\begin{align*}
	\int_{\gamma_2} \dif z \, f(z)
	&= \int_0^{\pi/2} \dif t \, \ee^{\ii t} \ee^{-\ii t} \ii \\
	&= \half \pi \ii
\end{align*}

%%%%%%%%%%%%%%%%%%%%%%%%%%%%%%%%%%%%%%%%%%%%%%%%%%%%%%%%%%%%%%%%%%%%%%%%%%%%%%%
%                                 Grenzwerte                                  %
%%%%%%%%%%%%%%%%%%%%%%%%%%%%%%%%%%%%%%%%%%%%%%%%%%%%%%%%%%%%%%%%%%%%%%%%%%%%%%%

\stepcounter{section}
\stepcounter{section}
\section{Grenzwerte}
\label 2

\subsection{erster Grenzwert}

Wir dürfen Integral und Grenzwert vertauschen, weil die Funktion stetig ist. Da
$f$ stetig ist, dürfen wir auch den Grenzwert in das Argument reinziehen. Somit
bleibt:
\[
	\int_0^{2\pi} \dif \phi \, f(0) = 2 \pi f(0)
\]

\subsection{zweiter Grenzwert}

Wir integrieren über den Kreisumfang, der allerdings gegen 0 geht. Daher können
wir die Funktion durch ihre nullte Näherung ersetzen:
\[
	\lim_{r \to 0} \int_{\abs z = r} \dif z \, \frac{f(0)}{z}
\]

Die nun konstante Funktion ziehen wir heraus und erhalten die Integralformel,
die wir bereits in der Vorlesung hatten ($\oint \dif z / z = 2 \pi \ii$). Somit
ist der Ausdruck $f(0) 2 \pi \ii$.

%%%%%%%%%%%%%%%%%%%%%%%%%%%%%%%%%%%%%%%%%%%%%%%%%%%%%%%%%%%%%%%%%%%%%%%%%%%%%%%
%                            Parameterdarstellung                             %
%%%%%%%%%%%%%%%%%%%%%%%%%%%%%%%%%%%%%%%%%%%%%%%%%%%%%%%%%%%%%%%%%%%%%%%%%%%%%%%

\section{Parameterdarstellung}
\label 3

Wir müssen folgende Fälle unterscheiden:

\begin{description}
	\item[Fall $k = 0$]
		Die Funktion ist konstant. Für eine solche Funktion gibt es eine
		Stammfunktion, es gilt also $\oint \ldots = 0$.

	\item[Fall $k > 0$ oder $k < -1$]
		Die Stammfunktion ist $F(z) = \frac{1}{k+1} \del{z - z_0}^{k+1}$. Die
		Riemann'sche Fläche, auf der die Funktion definiert ist, ist einfach
		$\C$, so dass auch hier $\oint \ldots = 0$ gilt.

	\item[Fall $k = -1$]
		Die Stammfunktion ist $F(z) = \ln\del{z - z_0}$, jedoch auf $2 \pi n
		\ii$ unbestimmt. Da die Riemann'sche Fläche die Spirale mit Zentrum
		$z_0$ ist, kommt hier $\oint \ldots = 2 \pi \ii$ heraus, falls $z_0$
		von $\gamma$ umschlossen wird. Ansonsten wechseln wir die Ebenen der
		Spirale nicht und erhalten wieder $\oint \ldots = 0$.
\end{description}

%%%%%%%%%%%%%%%%%%%%%%%%%%%%%%%%%%%%%%%%%%%%%%%%%%%%%%%%%%%%%%%%%%%%%%%%%%%%%%%
%                                 Sterngebiet                                 %
%%%%%%%%%%%%%%%%%%%%%%%%%%%%%%%%%%%%%%%%%%%%%%%%%%%%%%%%%%%%%%%%%%%%%%%%%%%%%%%

\section{Sterngebiet}
\label 4

\subsection{Sterngebiet?}

Nein, $\mathcal G$ ist kein Sterngebiet. Angenommen es wäre doch ein
Sterngebiet mit Mittelpunkt $g$. Dann ist $z(t) = (1-t) g$ für $t \in \intcc{0,
1}$ die Gerade vom Punkt $g$ zu $0$. Der Verbindungslinie von $g = z(0)$ zu
$z(2)$ geht durch 0, und ist somit nicht möglich. $g$ kann nicht der
Mittelpunkt des Gebiets sein. Somit kann $\mathcal G$ nicht sternförmig sein.

\subsection{geschlossene Wege}

Damit das Integral auf allen geschlossenen Wegen gleich 0 ist, muss die
Funktion eine Stammfunktion haben. Diese bestimmen wir mit
Partialbruchzerlegung:
\begin{gather*}
	f(z) = \frac{1}{z(z-1)} = \frac{1}{z-1} - \frac 1z \\
	F(z) = \ln\del{z-1} - \ln\del2
\end{gather*}

Der komplexe Logarithmus ist nur auf ein Vielfaches von $2 \pi \ii$ definiert.
Da wir allerdings die Differenz betrachten und keine Punkte zwischen 0 und 1
zulassen, gibt es nur Wege mit geradem Genus. Das Ergebnis $\oint \dif z / z =
2 \pi \ii$ erhalten wir nur, wenn wir eine Fläche mit Genus 1 umspannen, was
hier allerdings nicht geht.

Somit haben wir eine Stammfunktion und es gilt $\oint \dif z \, f(z) = 0$ für
alle erlaubten Wege.

%\bibliography{../../zentrale_BibTeX/Central}
%\bibliographystyle{plain}

\end{document}

% vim: spell spelllang=de
