% Copyright © 2012-2013 Martin Ueding <dev@martin-ueding.de>
%
\input{header.tex}

\usepackage{scrpage2}
\usepackage{tikz}
\usetikzlibrary{decorations.markings}

\newcommand{\themodul}{math340}
\newcommand{\thegruppe}{Gruppe 16 -- Malte Lackmann}
\newcommand{\theuebung}{13}

\pagestyle{scrheadings}

\automark{section}
\cfoot{\footnotesize{\thegruppe}}
\chead{}
\ifoot{\footnotesize{Ueding, Manz, Lemmer}}
\ihead{\themodul{} -- Übung \theuebung}
\ofoot{\footnotesize{Seite \thepage\ / \pageref{LastPage}}}
\ohead{\rightmark}
\setheadsepline{.4pt}


\def\thesection{\theuebung.\arabic{section}}
\def\thesubsection{\thesection\alph{subsection}}

\title{\themodul{} -- Übung \theuebung \\ \vspace{0.5cm} \large{\thegruppe}}

\author{
	Martin Ueding \\ \small{\href{mailto:mu@uni-bonn.de}{mu@uni-bonn.de}}
	\and
	Paul Manz
	\and
	Lino Lemmer
}

\begin{document}

\maketitle

\begin{table}[h]
	\centering
	\begin{tabular}{l|c|c|c|c|c}
		Aufgabe
		& \ref 1
		& \ref 2
		& \ref 3
		& \ref 4
		& $\sum$ \\
		\hline
		Punkte
		& \punkte / 3
		& \punkte / 6
		& \punkte / 4
		& \punkte / 3
		& \punkte / 16
	\end{tabular}
\end{table}

\section{Integrale}
\label 1

\section{Integrale}
\label 2

\section{Fouriertransformation}
\label 3

\section{Hauptzweig des Logarithmus}
\label 4

Wir wählen folgenden Pfad:

\begin{center}
	\begin{tikzpicture}[scale=1.5]
		\draw[->] (-1, 0) -- (3, 0) node[right] {$\Re$};
		\draw[->] (0, -1) -- (0, 2) node[above] {$\Im$};

		\begin{scope}[decoration={markings, mark=at position 0.5 with {\arrow{>}}}, thick]
			\draw[postaction={decorate}] (1, 0) -- (2, 0) node[below, midway] {$\gamma_1$};
			\draw[postaction={decorate}] (2, 0) arc (0:40:2);
			\draw[postaction={decorate}] (40:2) -- (40:1) node[above, sloped, midway] {$\gamma_3$};
			\draw[postaction={decorate}] (40:1) arc (40:0:1);
		\end{scope}

		\node[right] at (20:2) {$\gamma_2$};
		\node[left] at (20:1) {$\gamma_4$};

		\node[above] at (40:2) {$z$};
		\fill (40:2) circle (.6mm);

		\node[below] at (1, 0) {$1$};
		\fill (1, 0) circle (.6mm);
	\end{tikzpicture}
\end{center}

Dabei sind unsere Parametrisierungen:
\begin{align*}
	\gamma_1 \colon f(t) &= t & t &\in \intcc{1, \abs{z}} \\
	\gamma_2 \colon f(t) &= \abs{z} \ee^{\ii t} & t &\in \intcc{0, \arg\del{z}} \\
	\gamma_3 \colon f(t) &= t \ee^{\ii \arg\del{z}} & t &\in \intcc{\abs{z}, 1} \\
	\gamma_4 \colon f(t) &= \ee^{\ii t} & t &\in \intcc{\arg\del{z}, \abs{z}}
\end{align*}

Wir berechnen die einzelnen Integrale:
\begin{gather*}
	\int_1^{\abs{z}} \dif t \, \frac 1t = \log\del{\abs{z}} \\
	\int_0^{\arg\del{z}} \dif t \, \frac 1{\abs{z}} \ee^{\ii t} \ii \abs{z} \ee^{\ii t} = \ii \arg\del{z} \\
	\int_{\abs{z}}^1 \dif t \, \frac 1t \ee^{-\ii \arg\del{z}} \ee^{\ii \arg\del{z}} = - \log\del{ \abs{z}} \\
	\int_{\arg\del{z}}^0 \dif t \, \ee^{-\ii t} \ii \ee^{\ii t} = - \ii \arg\del{z}
\end{gather*}

Die ersten beiden Integrale zusammen sind gerade:
\[
	\log\del{\abs{z}}
	\ii \arg\del{z}
\]

Dies ist die Definition des komplexen Logarithmus.

Außerdem sind alle vier Integrale zusammen 0. Bei der Parametrisierung haben
wir indirekt benutzt, dass das Gebiet sternförmig ist. Nur dadurch ist die
Parametrisierung eindeutig gewesen. Weil das Ergebnis für jedes $z$ aus $\C^-$
gilt, muss $\log\del z$ in $\C^-$ holomorph sein.

%\bibliography{../../zentrale_BibTeX/Central}
%\bibliographystyle{plain}

\end{document}

% vim: spell spelllang=de
