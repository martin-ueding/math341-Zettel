% Copyright © 2012-2013 Martin Ueding <dev@martin-ueding.de>
%
% Copyright © 2012-2013 Martin Ueding <dev@martin-ueding.de>
%
\documentclass[11pt, ngerman, fleqn, DIV=15]{scrartcl}

\usepackage{graphicx}

%%%%%%%%%%%%%%%%%%%%%%%%%%%%%%%%%%%%%%%%%%%%%%%%%%%%%%%%%%%%%%%%%%%%%%%%%%%%%%%
%                                Locale, date                                 %
%%%%%%%%%%%%%%%%%%%%%%%%%%%%%%%%%%%%%%%%%%%%%%%%%%%%%%%%%%%%%%%%%%%%%%%%%%%%%%%

\usepackage{babel}
\usepackage[iso]{isodate}

%%%%%%%%%%%%%%%%%%%%%%%%%%%%%%%%%%%%%%%%%%%%%%%%%%%%%%%%%%%%%%%%%%%%%%%%%%%%%%%
%                          Margins and other spacing                          %
%%%%%%%%%%%%%%%%%%%%%%%%%%%%%%%%%%%%%%%%%%%%%%%%%%%%%%%%%%%%%%%%%%%%%%%%%%%%%%%

\usepackage[activate]{pdfcprot}
\usepackage[parfill]{parskip}
\usepackage{setspace}

\setlength{\columnsep}{2cm}

%%%%%%%%%%%%%%%%%%%%%%%%%%%%%%%%%%%%%%%%%%%%%%%%%%%%%%%%%%%%%%%%%%%%%%%%%%%%%%%
%                                    Color                                    %
%%%%%%%%%%%%%%%%%%%%%%%%%%%%%%%%%%%%%%%%%%%%%%%%%%%%%%%%%%%%%%%%%%%%%%%%%%%%%%%

\usepackage{color}

\definecolor{darkblue}{rgb}{0,0,.5}
\definecolor{darkgreen}{rgb}{0,.5,0}
\definecolor{darkred}{rgb}{.7,0,0}

%%%%%%%%%%%%%%%%%%%%%%%%%%%%%%%%%%%%%%%%%%%%%%%%%%%%%%%%%%%%%%%%%%%%%%%%%%%%%%%
%                         Font and font like settings                         %
%%%%%%%%%%%%%%%%%%%%%%%%%%%%%%%%%%%%%%%%%%%%%%%%%%%%%%%%%%%%%%%%%%%%%%%%%%%%%%%

\usepackage[charter, greekuppercase=italicized]{mathdesign}
\usepackage{beramono}
\usepackage{berasans}

% Style of vectors and tensors.
\newcommand{\tens}[1]{\boldsymbol{\mathsf{#1}}}
\renewcommand{\vec}[1]{\boldsymbol{#1}}

%%%%%%%%%%%%%%%%%%%%%%%%%%%%%%%%%%%%%%%%%%%%%%%%%%%%%%%%%%%%%%%%%%%%%%%%%%%%%%%
%                               Input encoding                                %
%%%%%%%%%%%%%%%%%%%%%%%%%%%%%%%%%%%%%%%%%%%%%%%%%%%%%%%%%%%%%%%%%%%%%%%%%%%%%%%

\usepackage[T1]{fontenc}
\usepackage[utf8]{inputenc}

%%%%%%%%%%%%%%%%%%%%%%%%%%%%%%%%%%%%%%%%%%%%%%%%%%%%%%%%%%%%%%%%%%%%%%%%%%%%%%%
%                         Hyperrefs and PDF metadata                          %
%%%%%%%%%%%%%%%%%%%%%%%%%%%%%%%%%%%%%%%%%%%%%%%%%%%%%%%%%%%%%%%%%%%%%%%%%%%%%%%

\usepackage{hyperref}
\usepackage{lastpage}

\hypersetup{
	breaklinks=false,
	citecolor=darkgreen,
	colorlinks=true,
	linkcolor=black,
	menucolor=black,
	pdfauthor={Martin Ueding},
	urlcolor=darkblue,
}

%%%%%%%%%%%%%%%%%%%%%%%%%%%%%%%%%%%%%%%%%%%%%%%%%%%%%%%%%%%%%%%%%%%%%%%%%%%%%%%
%                               Math Operators                                %
%%%%%%%%%%%%%%%%%%%%%%%%%%%%%%%%%%%%%%%%%%%%%%%%%%%%%%%%%%%%%%%%%%%%%%%%%%%%%%%

\usepackage[thinspace, squaren]{SIunits}
\usepackage{amsmath}
\usepackage{amsthm}
\usepackage{commath}

% Word like operators.
\DeclareMathOperator{\acosh}{arcosh}
\DeclareMathOperator{\arcosh}{arcosh}
\DeclareMathOperator{\arcsinh}{arsinh}
\DeclareMathOperator{\arsinh}{arsinh}
\DeclareMathOperator{\asinh}{arsinh}
\DeclareMathOperator{\card}{card}
\DeclareMathOperator{\diam}{diam}
\renewcommand{\Im}{\mathop{{}\mathrm{Im}}\nolimits}
\renewcommand{\Re}{\mathop{{}\mathrm{Re}}\nolimits}

% Special single letters.
\DeclareMathOperator{\fourier}{\mathcal{F}}
\newcommand{\C}{\ensuremath{\mathbb C}}
\newcommand{\ee}{\mathrm e}
\newcommand{\ii}{\mathrm i}
\newcommand{\N}{\ensuremath{\mathbb N}}
\newcommand{\R}{\ensuremath{\mathbb R}}
\newcommand{\Z}{\ensuremath{\mathbb Z}}

% Shape like operators.
\DeclareMathOperator{\dalambert}{\Box}
\DeclareMathOperator{\laplace}{\bigtriangleup}
\newcommand{\curl}{\vnabla \times}
\newcommand{\divergence}[1]{\inner{\vnabla}{#1}}
\newcommand{\vnabla}{\vec \nabla}

% Shortcuts
\newcommand{\ev}{\hat{\vec e}}
\newcommand{\e}[1]{\cdot 10^{#1}}
\newcommand{\half}{\frac 12}
\newcommand{\inner}[2]{\left\langle #1, #2 \right\rangle}

% Placeholders.
\newcommand{\emesswert}{\del{\messwert \pm \messwert}}
\newcommand{\fehlt}{\textcolor{darkred}{Hier fehlen noch Inhalte.}}
\newcommand{\messwert}{\textcolor{blue}{\square}}
\newcommand{\punkte}{\textcolor{white}{xxxxx}}

% Separator for equations on a single line.
\newcommand{\eqnsep}{,\quad}

%%%%%%%%%%%%%%%%%%%%%%%%%%%%%%%%%%%%%%%%%%%%%%%%%%%%%%%%%%%%%%%%%%%%%%%%%%%%%%%
%                                  Headings                                   %
%%%%%%%%%%%%%%%%%%%%%%%%%%%%%%%%%%%%%%%%%%%%%%%%%%%%%%%%%%%%%%%%%%%%%%%%%%%%%%%

\usepackage{scrpage2}

\pagestyle{scrheadings}

\automark{section}
\cfoot{\footnotesize{Seite \thepage\ / \pageref{LastPage}}}
\chead{}
\ihead{}
\ohead{\rightmark}
\setheadsepline{.4pt}

%%%%%%%%%%%%%%%%%%%%%%%%%%%%%%%%%%%%%%%%%%%%%%%%%%%%%%%%%%%%%%%%%%%%%%%%%%%%%%%
%                            Bibliography (BibTeX)                            %
%%%%%%%%%%%%%%%%%%%%%%%%%%%%%%%%%%%%%%%%%%%%%%%%%%%%%%%%%%%%%%%%%%%%%%%%%%%%%%%

\newcommand{\bibliographyfile}{../../zentrale_BibTeX/Central}


\usepackage{scrpage2}
\usepackage{tikz}
\usetikzlibrary{decorations.markings}

\newcommand{\themodul}{math340}
\newcommand{\thegruppe}{Gruppe 16 -- Malte Lackmann}
\newcommand{\theuebung}{13}

\pagestyle{scrheadings}

\automark{section}
\cfoot{\footnotesize{\thegruppe}}
\chead{}
\ifoot{\footnotesize{Ueding, Manz, Lemmer}}
\ihead{\themodul{} -- Übung \theuebung}
\ofoot{\footnotesize{Seite \thepage\ / \pageref{LastPage}}}
\ohead{\rightmark}
\setheadsepline{.4pt}


\def\thesection{\theuebung.\arabic{section}}
\def\thesubsection{\thesection\alph{subsection}}

\title{\themodul{} -- Übung \theuebung \\ \vspace{0.5cm} \large{\thegruppe}}

\author{
	Martin Ueding \\ \small{\href{mailto:mu@uni-bonn.de}{mu@uni-bonn.de}}
	\and
	Paul Manz
	\and
	Lino Lemmer
}

\begin{document}

\maketitle

\begin{table}[h]
	\centering
	\begin{tabular}{l|c|c|c|c|c}
		Aufgabe
		& \ref 1
		& \ref 2
		& \ref 3
		& \ref 4
		& $\sum$ \\
		\hline
		Punkte
		& \punkte / 3
		& \punkte / 6
		& \punkte / 4
		& \punkte / 3
		& \punkte / 16
	\end{tabular}
\end{table}

\section{Integrale}
\label 1

\subsection{}

\fehlt

\subsection{}

\fehlt

\subsection{}

\fehlt

\section{Integrale}
\label 2

\subsection{}

Es ist folgendes Integral zu berechnen:
\[
	\int_{\abs z = 2} \frac{\dif z}{1 + z^2}
\]

Wir dürfen den Integralsatz nicht direkt anwenden, weil wir mehr als eine
Singularität mit dem Weg umschlossen haben. Daher müssen wir erst eine
Partialbruchzerlegung durchführen:
\[
	\half \int_{\abs z = 2} \frac{\dif z}{1 + i z}
	+
	\half \int_{\abs z = 2} \frac{\dif z}{1 - i z}
\]

Für diese beiden Integrale können wir den Satz nun anwenden und erhalten pro
Summand $\half 2 \pi \ii$. Als Endergebnis erhalten wir $2 \pi \ii$.

\stepcounter{subsection}
\subsection{}

\fehlt

\section{Fouriertransformation}
\label 3

\stepcounter{subsection}
\subsection{}

\fehlt

\section{Hauptzweig des Logarithmus}
\label 4

Wir wählen folgenden Pfad:

\begin{center}
	\begin{tikzpicture}[scale=1.5]
		\draw[->] (-1, 0) -- (3, 0) node[right] {$\Re$};
		\draw[->] (0, -1) -- (0, 2) node[above] {$\Im$};

		\begin{scope}[decoration={markings, mark=at position 0.5 with {\arrow{>}}}, thick]
			\draw[postaction={decorate}] (1, 0) -- (2, 0) node[below, midway] {$\gamma_1$};
			\draw[postaction={decorate}] (2, 0) arc (0:40:2);
			\draw[postaction={decorate}] (40:2) -- (40:1) node[above, sloped, midway] {$\gamma_3$};
			\draw[postaction={decorate}] (40:1) arc (40:0:1);
		\end{scope}

		\node[right] at (20:2) {$\gamma_2$};
		\node[left] at (20:1) {$\gamma_4$};

		\node[above] at (40:2) {$z$};
		\fill (40:2) circle (.6mm);

		\node[below] at (1, 0) {$1$};
		\fill (1, 0) circle (.6mm);
	\end{tikzpicture}
\end{center}

Dabei sind unsere Parametrisierungen:
\begin{align*}
	\gamma_1 \colon f(t) &= t & t &\in \intcc{1, \abs{z}} \\
	\gamma_2 \colon f(t) &= \abs{z} \ee^{\ii t} & t &\in \intcc{0, \arg\del{z}} \\
	\gamma_3 \colon f(t) &= t \ee^{\ii \arg\del{z}} & t &\in \intcc{\abs{z}, 1} \\
	\gamma_4 \colon f(t) &= \ee^{\ii t} & t &\in \intcc{\arg\del{z}, \abs{z}}
\end{align*}

Wir berechnen die einzelnen Integrale:
\begin{gather*}
	\int_1^{\abs{z}} \dif t \, \frac 1t = \log\del{\abs{z}} \\
	\int_0^{\arg\del{z}} \dif t \, \frac 1{\abs{z}} \ee^{\ii t} \ii \abs{z} \ee^{\ii t} = \ii \arg\del{z} \\
	\int_{\abs{z}}^1 \dif t \, \frac 1t \ee^{-\ii \arg\del{z}} \ee^{\ii \arg\del{z}} = - \log\del{ \abs{z}} \\
	\int_{\arg\del{z}}^0 \dif t \, \ee^{-\ii t} \ii \ee^{\ii t} = - \ii \arg\del{z}
\end{gather*}

Die ersten beiden Integrale zusammen sind gerade:
\[
	\log\del{\abs{z}}
	\ii \arg\del{z}
\]

Dies ist die Definition des komplexen Logarithmus.

Außerdem sind alle vier Integrale zusammen 0. Bei der Parametrisierung haben
wir indirekt benutzt, dass das Gebiet sternförmig ist. Nur dadurch ist die
Parametrisierung eindeutig gewesen. Weil das Ergebnis für jedes $z$ aus $\C^-$
gilt, muss $\log\del z$ in $\C^-$ holomorph sein.

%\bibliography{../../zentrale_BibTeX/Central}
%\bibliographystyle{plain}

\end{document}

% vim: spell spelllang=de
