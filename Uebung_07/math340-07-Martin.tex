% Copyright © 2012 Martin Ueding <dev@martin-ueding.de>
%
\documentclass[11pt, ngerman, fleqn]{article}

\usepackage[a4paper, left=3cm, right=2cm, top=2cm, bottom=2cm]{geometry}
\usepackage[activate]{pdfcprot}
\usepackage[cdot, squaren]{SIunits}
\usepackage[iso]{isodate}
\usepackage[parfill]{parskip}
\usepackage[T1]{fontenc}
\usepackage[utf8]{inputenc}
\usepackage{amsmath}
\usepackage{amsthm}
\usepackage{babel}
\usepackage{color}
\usepackage{commath}
\usepackage{epstopdf}
\usepackage{fancyhdr}
\usepackage{graphicx}
\usepackage{hyperref}
\usepackage{lastpage}
\usepackage{setspace}
\usepackage{tikz}

\usepackage[charter, greekuppercase=italicized]{mathdesign}

\definecolor{darkblue}{rgb}{0,0,.5}
\definecolor{darkgreen}{rgb}{0,.5,0}

\hypersetup{
	breaklinks=false,
	citecolor=darkgreen,
	colorlinks=true,
	linkcolor=black,
	menucolor=black,
	urlcolor=darkblue,
}

\setlength{\columnsep}{2cm}

\DeclareMathOperator{\arcsinh}{arsinh}
\DeclareMathOperator{\arsinh}{arsinh}
\DeclareMathOperator{\asinh}{arsinh}
\DeclareMathOperator{\card}{card}
\DeclareMathOperator{\diam}{diam}

\newcommand{\dalambert}{\mathop{{}\Box}\nolimits}
\newcommand{\divergence}[1]{\inner{\vnabla}{#1}}
\newcommand{\ee}{\mathrm e}
\newcommand{\emesswert}{\del{\messwert \pm \messwert}}
\newcommand{\e}[1]{\cdot 10^{#1}}
\newcommand{\fehlt}{\textcolor{red}{Hier fehlen noch Inhalte.}}
\newcommand{\half}{\frac 12}
\newcommand{\ii}{\mathrm i}
\newcommand{\inner}[2]{\left\langle #1, #2 \right\rangle}
\newcommand{\laplace}{\mathop{{}\bigtriangleup}\nolimits}
\newcommand{\messwert}{\textcolor{blue}{\square}}
\newcommand{\punkte}{\textcolor{white}{xxxxx}}
\newcommand{\tens}[1]{\boldsymbol{#1}}
\newcommand{\vnabla}{\vec \nabla}
\renewcommand{\vec}[1]{\boldsymbol{#1}}

\newcommand{\themodul}{math340}
\newcommand{\thegruppe}{Gruppe 16 -- Malte Lackmann}
\newcommand{\theuebung}{7}

\pagestyle{fancy}

\fancyfoot[C]{\footnotesize{\thegruppe}}
\fancyfoot[L]{\footnotesize{Ueding, Manz, Lemmer}}
\fancyfoot[R]{\footnotesize{Seite \thepage\ / \pageref{LastPage}}}
\fancyhead[L]{\themodul{} -- Übung \theuebung}

\def\thesection{\theuebung.\arabic{section}}
\def\thesubsection{\thesection\alph{subsection}}

\title{\themodul{} -- Übung \theuebung \\ \vspace{0.5cm} \large{\thegruppe}}

\author{
	Martin Ueding \\ \small{\href{mailto:mu@uni-bonn.de}{mu@uni-bonn.de}}
	\and
	Paul Manz
	\and
	Lino Lemmer
}

\begin{document}

\maketitle

\begin{table}[h]
	\centering
	\begin{tabular}{l|c|c|c|c}
		Aufgabe & 7.1 & \ref 2 & \ref 3 & $\sum$   \\
		\hline
		Punkte & \punkte / 6 & \punkte / 6 & \punkte / 8 & \punkte / 20
	\end{tabular}
\end{table}

%%%%%%%%%%%%%%%%%%%%%%%%%%%%%%%%%%%%%%%%%%%%%%%%%%%%%%%%%%%%%%%%%%%%%%%%%%%%%%%
%                         äußeres Dirichletproblem                          %
%%%%%%%%%%%%%%%%%%%%%%%%%%%%%%%%%%%%%%%%%%%%%%%%%%%%%%%%%%%%%%%%%%%%%%%%%%%%%%%

\stepcounter{section}

%%%%%%%%%%%%%%%%%%%%%%%%%%%%%%%%%%%%%%%%%%%%%%%%%%%%%%%%%%%%%%%%%%%%%%%%%%%%%%%
%                              Separationsansatz                              %
%%%%%%%%%%%%%%%%%%%%%%%%%%%%%%%%%%%%%%%%%%%%%%%%%%%%%%%%%%%%%%%%%%%%%%%%%%%%%%%

\section{Separationsansatz}
\label 2

Wir benutzen $u(x, y) = v(x) w(y)$ und erhalten:
\[
	\frac{v''}{v} = \alpha ^2
	\quad\wedge\quad
	\frac{w''}{w} = \alpha ^2
\]

Diese Differentialgleichungen lösen wir durch einen Exponentialansatz. Die
zusammengesetzte Lösung ist:
\[
	u(x, y)
	=
	\sum_{n=0}^\infty
	\del{a_n \ee^{\alpha_n x} + b_n \ee^{-\alpha_n x}}
	\del{c_n \ee^{\ii \alpha_n y} + d_n \ee^{-\ii \alpha_n y}}
\]

Nun müssen wir noch die Randbedingungen erfüllen. Dabei setzen wir schon
$\sin^3(y) = \frac 34 \sin(y) - \frac 14 \sin(3y)$ ein. Die vier Bedingungen
sind:
\begin{align}
	\label{2-1}
	u(x, 0) &=&
	\sum_{n=0}^\infty
	\del{a_n \ee^{\alpha_n x} + b_n \ee^{-\alpha_n x}}
	\del{c_n + d_n }
	&= \sin(x) \\
	%
	\label{2-2}
	u(0, y) &=&
	\sum_{n=0}^\infty
	\del{a_n + b_n }
	\del{c_n \ee^{\ii \alpha_n y} + d_n \ee^{-\ii \alpha_n y}}
	&= \frac 34 \sin(y) - \frac 14 \sin(3y) \\
	%
	\label{2-3}
	u(x, \pi) &=&
	\sum_{n=0}^\infty
	\del{a_n \ee^{\alpha_n x} + b_n \ee^{-\alpha_n x}}
	\del{c_n \ee^{\ii \alpha_n \pi} + d_n \ee^{-\ii \alpha_n \pi}}
	&= 0 \\
	%
	\label{2-4}
	u(\pi, y) &=&
	\sum_{n=0}^\infty
	\del{a_n \ee^{\alpha_n \pi} + b_n \ee^{-\alpha_n \pi}}
	\del{c_n \ee^{\ii \alpha_n y} + d_n \ee^{-\ii \alpha_n y}}
	&= 0
\end{align}

\paragraph{Bedingungen aus \eqref{2-1}}

Für mindestens ein $n$ muss gelten:
\[
	c_n + d_n \neq 0
\]

Ein Summand sollte ausreichen, um das $\sin(x)$ zu erreichen. Dieses sei das
$0$. Element. Es muss gelten:
\[
	\alpha_0 = 1
	\quad\wedge\quad
	a_0 = - b_0
	\quad\wedge\quad
	2 a_0 \del{c_0 + d_0} = 1
\]

\paragraph{Bedingungen aus \eqref{2-2}}

Für mindestens ein $n$ muss gelten:
\[
	a_n + b_n \neq 0
\]

Da die Exponentialfunktionen eine Basis für den $L^2$ bilden, sollten zwei
Summanden ausreichen, um auf die geforderten $\frac 34 \sin(y) - \frac 14
\sin(3y)$ zu kommen. Diese Summanden seien die ersten beiden. Für den ersten
Summanden erhalten wir folgende Bedingungen:
\[
	\del{a_0 + b_0} \frac{c_0}2 = \frac 34
	\quad\wedge\quad
	c_0 = - d_0
	\quad\wedge\quad
	\alpha_0 = 1
\]

Sowie durch den zweiten Summanden:
\[
	\del{a_1 + b_1} \frac{c_1}{2} = - \frac 14
	\quad\wedge\quad
	c_1 = - d_1
	\quad\wedge\quad
	\alpha_1 = 3
\]

\paragraph{Bedingungen aus \eqref{2-3}}

Es muss für alle $n$ gelten, dass:
\[
	\del{c_n \ee^{\ii \alpha_n \pi} + d_n \ee^{-\ii \alpha_n \pi}} = 0
\]

Nur so ist gewährleistet, dass für alle $x$ gerade $0$ herauskommt. Dies können
wir umformen zu:
\[
	- \frac{d_0}{c_0} = \ee^{2 \ii \alpha_0 \pi}
\]

\paragraph{Bedingungen aus \eqref{2-4}}

Analog zum vorherigen Abschnitt muss gelten für alle $n$ gelten:
\[
	- \frac{b_0}{a_0} = \ee^{2 \alpha_0 \pi}
\]

Eventuell lässt sich auf diesem Wege die Koeffizienten so bestimmen, dass eine
Lösung $u$ herauskommt.

%%%%%%%%%%%%%%%%%%%%%%%%%%%%%%%%%%%%%%%%%%%%%%%%%%%%%%%%%%%%%%%%%%%%%%%%%%%%%%%
%                              Randwertaufgaben                               %
%%%%%%%%%%%%%%%%%%%%%%%%%%%%%%%%%%%%%%%%%%%%%%%%%%%%%%%%%%%%%%%%%%%%%%%%%%%%%%%

\section{Randwertaufgaben}
\label 3

\subsection{}

Wir nehmen an, dass die Funktion $u$ nur von $r$ abhängt. Somit vereinfacht
sich die Poissongleichung zu:
\begin{align*}
	\laplace u(r) = \frac{1}{r^2} \dpd {}r \del{r^2 \dpd ur} &= r \\
	\dpd {}r \del{r^2 \dpd ur} &= r^3 \\
	r^2 \dpd ur &= \frac 14 r^4 + c_1 \\
		   \dpd ur &= \frac 14 r^2 + c_1 \frac{1}{r^2} \\
	 u &= \frac 1{12} r^3 + c_2 \frac 1r + c_3
\end{align*}

Diese Funktion ist allerdings bei $r = 0$ nicht differenzierbar, womit $c_2 =
0$ gelten muss. Ansonsten hätten wir auch eine Randbedingung zu wenig. Es
bleibt die Randbedingung:
\[
	u(R) = \frac{1}{12} R^3 + c_3 = \pi R^3
\]

Daraus leiten wir ab, dass $c_3 = \del{\pi - 1/12} R^3$ sein muss. Die Lösung
lautet:
\[
	u(r) = \frac{1}{12} r^3 + \del{\pi - \frac{1}{12}} R^3
\]

Wir machen die Probe und setzen $u$ in die Poissongleichung ein:
\[
\dpd ur = \frac 14 r^2
\overset{r^2}{\quad\leadsto\quad}
\frac 14 r^4
\overset{\dpd {} r}{\quad\leadsto\quad}
r^3
\overset{r^{-2}}{\quad\leadsto\quad}
r
\quad \checkmark
\]

Die Funktion löst also die Poissongleichung.

\subsection{}

Hier können wir analog vorgehen:
\begin{align*}
	\laplace u = \frac{1}{r^2} \dpd{}r \del{r^2 \dpd ur} &= 1 \\
	\dpd{}r \del{r^2 \dpd ur} &= r^2 \\
	r^2 \dpd ur &= \frac 13 r^3 + c_1 \\
		   \dpd ur &= \frac 13 r + c_1 \frac{1}{r^2} \\
		   u &= \frac 16 r^2 + c_2 \frac{1}{r} + c_3
\end{align*}

Hier muss die Funktion nicht bei $r = 0$ differenzierbar sein, so dass $c_2
\neq 0$ zulässig ist. Daher brauchen wir auch zwei Randbedingungen, um beide
Konstanten zu bestimmen. Mit $u(1) = 0$ und $u(4) = 0$ erhalten wir $c_2 =
10/3$ und $c_3 = - 7/2$. Die Lösung ist also:
\[
	u(r) = \frac 16 r^2 + \frac{10}{3} \frac{1}{r} - \frac 72
\]

%\bibliography{../../zentrale_BibTeX/Central}
%\bibliographystyle{plain}

\end{document}

% vim: spell spelllang=de
