% Copyright © 2012 Martin Ueding <dev@martin-ueding.de>
%
% Copyright © 2012-2013 Martin Ueding <dev@martin-ueding.de>
%
\documentclass[11pt, ngerman, fleqn, DIV=15]{scrartcl}

\usepackage{graphicx}

%%%%%%%%%%%%%%%%%%%%%%%%%%%%%%%%%%%%%%%%%%%%%%%%%%%%%%%%%%%%%%%%%%%%%%%%%%%%%%%
%                                Locale, date                                 %
%%%%%%%%%%%%%%%%%%%%%%%%%%%%%%%%%%%%%%%%%%%%%%%%%%%%%%%%%%%%%%%%%%%%%%%%%%%%%%%

\usepackage{babel}
\usepackage[iso]{isodate}

%%%%%%%%%%%%%%%%%%%%%%%%%%%%%%%%%%%%%%%%%%%%%%%%%%%%%%%%%%%%%%%%%%%%%%%%%%%%%%%
%                          Margins and other spacing                          %
%%%%%%%%%%%%%%%%%%%%%%%%%%%%%%%%%%%%%%%%%%%%%%%%%%%%%%%%%%%%%%%%%%%%%%%%%%%%%%%

\usepackage[activate]{pdfcprot}
\usepackage[parfill]{parskip}
\usepackage{setspace}

\setlength{\columnsep}{2cm}

%%%%%%%%%%%%%%%%%%%%%%%%%%%%%%%%%%%%%%%%%%%%%%%%%%%%%%%%%%%%%%%%%%%%%%%%%%%%%%%
%                                    Color                                    %
%%%%%%%%%%%%%%%%%%%%%%%%%%%%%%%%%%%%%%%%%%%%%%%%%%%%%%%%%%%%%%%%%%%%%%%%%%%%%%%

\usepackage{color}

\definecolor{darkblue}{rgb}{0,0,.5}
\definecolor{darkgreen}{rgb}{0,.5,0}
\definecolor{darkred}{rgb}{.7,0,0}

%%%%%%%%%%%%%%%%%%%%%%%%%%%%%%%%%%%%%%%%%%%%%%%%%%%%%%%%%%%%%%%%%%%%%%%%%%%%%%%
%                         Font and font like settings                         %
%%%%%%%%%%%%%%%%%%%%%%%%%%%%%%%%%%%%%%%%%%%%%%%%%%%%%%%%%%%%%%%%%%%%%%%%%%%%%%%

\usepackage[charter, greekuppercase=italicized]{mathdesign}
\usepackage{beramono}
\usepackage{berasans}

% Style of vectors and tensors.
\newcommand{\tens}[1]{\boldsymbol{\mathsf{#1}}}
\renewcommand{\vec}[1]{\boldsymbol{#1}}

%%%%%%%%%%%%%%%%%%%%%%%%%%%%%%%%%%%%%%%%%%%%%%%%%%%%%%%%%%%%%%%%%%%%%%%%%%%%%%%
%                               Input encoding                                %
%%%%%%%%%%%%%%%%%%%%%%%%%%%%%%%%%%%%%%%%%%%%%%%%%%%%%%%%%%%%%%%%%%%%%%%%%%%%%%%

\usepackage[T1]{fontenc}
\usepackage[utf8]{inputenc}

%%%%%%%%%%%%%%%%%%%%%%%%%%%%%%%%%%%%%%%%%%%%%%%%%%%%%%%%%%%%%%%%%%%%%%%%%%%%%%%
%                         Hyperrefs and PDF metadata                          %
%%%%%%%%%%%%%%%%%%%%%%%%%%%%%%%%%%%%%%%%%%%%%%%%%%%%%%%%%%%%%%%%%%%%%%%%%%%%%%%

\usepackage{hyperref}
\usepackage{lastpage}

\hypersetup{
	breaklinks=false,
	citecolor=darkgreen,
	colorlinks=true,
	linkcolor=black,
	menucolor=black,
	pdfauthor={Martin Ueding},
	urlcolor=darkblue,
}

%%%%%%%%%%%%%%%%%%%%%%%%%%%%%%%%%%%%%%%%%%%%%%%%%%%%%%%%%%%%%%%%%%%%%%%%%%%%%%%
%                               Math Operators                                %
%%%%%%%%%%%%%%%%%%%%%%%%%%%%%%%%%%%%%%%%%%%%%%%%%%%%%%%%%%%%%%%%%%%%%%%%%%%%%%%

\usepackage[thinspace, squaren]{SIunits}
\usepackage{amsmath}
\usepackage{amsthm}
\usepackage{commath}

% Word like operators.
\DeclareMathOperator{\acosh}{arcosh}
\DeclareMathOperator{\arcosh}{arcosh}
\DeclareMathOperator{\arcsinh}{arsinh}
\DeclareMathOperator{\arsinh}{arsinh}
\DeclareMathOperator{\asinh}{arsinh}
\DeclareMathOperator{\card}{card}
\DeclareMathOperator{\diam}{diam}
\renewcommand{\Im}{\mathop{{}\mathrm{Im}}\nolimits}
\renewcommand{\Re}{\mathop{{}\mathrm{Re}}\nolimits}

% Special single letters.
\DeclareMathOperator{\fourier}{\mathcal{F}}
\newcommand{\C}{\ensuremath{\mathbb C}}
\newcommand{\ee}{\mathrm e}
\newcommand{\ii}{\mathrm i}
\newcommand{\N}{\ensuremath{\mathbb N}}
\newcommand{\R}{\ensuremath{\mathbb R}}
\newcommand{\Z}{\ensuremath{\mathbb Z}}

% Shape like operators.
\DeclareMathOperator{\dalambert}{\Box}
\DeclareMathOperator{\laplace}{\bigtriangleup}
\newcommand{\curl}{\vnabla \times}
\newcommand{\divergence}[1]{\inner{\vnabla}{#1}}
\newcommand{\vnabla}{\vec \nabla}

% Shortcuts
\newcommand{\ev}{\hat{\vec e}}
\newcommand{\e}[1]{\cdot 10^{#1}}
\newcommand{\half}{\frac 12}
\newcommand{\inner}[2]{\left\langle #1, #2 \right\rangle}

% Placeholders.
\newcommand{\emesswert}{\del{\messwert \pm \messwert}}
\newcommand{\fehlt}{\textcolor{darkred}{Hier fehlen noch Inhalte.}}
\newcommand{\messwert}{\textcolor{blue}{\square}}
\newcommand{\punkte}{\textcolor{white}{xxxxx}}

% Separator for equations on a single line.
\newcommand{\eqnsep}{,\quad}

%%%%%%%%%%%%%%%%%%%%%%%%%%%%%%%%%%%%%%%%%%%%%%%%%%%%%%%%%%%%%%%%%%%%%%%%%%%%%%%
%                                  Headings                                   %
%%%%%%%%%%%%%%%%%%%%%%%%%%%%%%%%%%%%%%%%%%%%%%%%%%%%%%%%%%%%%%%%%%%%%%%%%%%%%%%

\usepackage{scrpage2}

\pagestyle{scrheadings}

\automark{section}
\cfoot{\footnotesize{Seite \thepage\ / \pageref{LastPage}}}
\chead{}
\ihead{}
\ohead{\rightmark}
\setheadsepline{.4pt}

%%%%%%%%%%%%%%%%%%%%%%%%%%%%%%%%%%%%%%%%%%%%%%%%%%%%%%%%%%%%%%%%%%%%%%%%%%%%%%%
%                            Bibliography (BibTeX)                            %
%%%%%%%%%%%%%%%%%%%%%%%%%%%%%%%%%%%%%%%%%%%%%%%%%%%%%%%%%%%%%%%%%%%%%%%%%%%%%%%

\newcommand{\bibliographyfile}{../../zentrale_BibTeX/Central}


\usepackage{fancyhdr}
\usepackage{tikz}

\newcommand{\themodul}{math340}
\newcommand{\thegruppe}{Gruppe 16 -- Malte Lackmann}
\newcommand{\theuebung}{8}

\pagestyle{fancy}

\fancyfoot[C]{\footnotesize{\thegruppe}}
\fancyfoot[L]{\footnotesize{Ueding, Manz, Lemmer}}
\fancyfoot[R]{\footnotesize{Seite \thepage\ / \pageref{LastPage}}}
\fancyhead[L]{\themodul{} -- Übung \theuebung}

\def\thesection{\theuebung.\arabic{section}}
\def\thesubsection{\thesection\alph{subsection}}

\title{\themodul{} -- Übung \theuebung \\ \vspace{0.5cm} \large{\thegruppe}}

\author{
	Martin Ueding \\ \small{\href{mailto:mu@uni-bonn.de}{mu@uni-bonn.de}}
	\and
	Paul Manz
	\and
	Lino Lemmer
}

\begin{document}

\maketitle

\begin{table}[h]
	\centering
	\begin{tabular}{l|c|c|c|c}
		Aufgabe & \ref 1 & \ref 2 & $\sum$   \\
		\hline
		Punkte & \punkte / 10 & \punkte / 5 & \punkte / 15
	\end{tabular}
\end{table}

%%%%%%%%%%%%%%%%%%%%%%%%%%%%%%%%%%%%%%%%%%%%%%%%%%%%%%%%%%%%%%%%%%%%%%%%%%%%%%%
%                                                                             %
%%%%%%%%%%%%%%%%%%%%%%%%%%%%%%%%%%%%%%%%%%%%%%%%%%%%%%%%%%%%%%%%%%%%%%%%%%%%%%%

\section{Eigenschaften der Legendrepolynome}
\label 1

\subsection{$P$ berechnen}

Wir benutzen die Formel \eqref{eq:leibniz} und setzen ein:
\begin{align*}
	P_1(x) &= x \\
	P_2(x) &= \frac{3}{2} x^2-\frac{1}{2} \\
	P_3(x) &= \frac{5}{2} x^3-\frac{3}{2} x \\
	P_4(x) &= \frac{35}{8} x^4-\frac{15}{4}+\frac{3}{8} x^2
\end{align*}

\subsection{$P(\pm 1)$}

In der anderen Aufgabe haben wir bereits hergeleitet, dass gilt
\eqref{eq:leibniz}:
\[
	P_l(x) = \frac{1}{2^l l!} \dod[l]{}x \del{x^2-1}^l
	= \frac{1}{2^l} \sum_{i=0}^l \binom{l}{i} (x-1)^i (x+1)^{l-i}
\]

Wenn wir dort $x = 1$ einsetzen, sind alle Potenten von $(x-1)$ gleich $0$,
außer wenn $i = 0$ ist. Dann ist allerdings $(x+1)^l$ gerade $2^l$. Somit ist
$P_l(1) = 1$.

Für $x = -1$ gilt das gleiche Argument, die verbleibende Klammer ist allerdings
$(-2)^l$, so dass $P_l(-1) = (-1)^l$ bleibt.

\subsection{Orthogonalität}

\fehlt

\subsection{Differentialgleichung erfüllen}

\fehlt

%%%%%%%%%%%%%%%%%%%%%%%%%%%%%%%%%%%%%%%%%%%%%%%%%%%%%%%%%%%%%%%%%%%%%%%%%%%%%%%
%                                                                             %
%%%%%%%%%%%%%%%%%%%%%%%%%%%%%%%%%%%%%%%%%%%%%%%%%%%%%%%%%%%%%%%%%%%%%%%%%%%%%%%

\stepcounter{section}
\section{Dirichtletproblem}
\label 2

Die Randbedingung formen wir um:
\[
	u(x, y, z) = z^4
	\quad\Leftrightarrow\quad
	u(r, \theta, \phi) = r^4 \cos^4 \theta
\]

Wir setzen $r = R$ ein und erhalten als Randbedingung eine Funktion auf der
Kugelfläche:
\[
	u_R(\theta, \phi) = R^4 \cos^4 \theta
\]

Da $\pd{u_R}\phi = 0$, können wir $\phi$ aus den Argumenten streichen. Außerdem
setzen wir $\cos \theta =: x$ um die weiteren Rechnungen übersichtlicher zu
gestalten. Somit erhalten wir als Randbedingung:
\[
	\tilde u(x) = x^4
\]

Diese Funktion entwickeln wir nun nach Kugelflächenfunktionen. Aus der
Symmetrie in $\phi$ müssen die $m = 0$ sein. Somit bleiben die $l$. Die
Koeffizienten $K_l^m$ bestimmen wir über das Skalarprodukt:
\[
\inner{\tilde u}{Y_l^0} = \int_0^{2\pi} \dif \phi \int_{-1}^1 \dif x \, u Y_l^{0*}
\]

Dabei sind die Kugelflächenfunktionen gegeben durch:
\begin{align*}
Y_l^m(\theta, \phi) &= \sqrt{\frac{2l+1}{4\pi} \frac{(l-m)!}{(l+m)!}} P_l^m(\cos \theta) \ee^{\ii m \phi} \\
	\intertext{%
	Allerdings ist hier gerade $m = 0$, so dass sich das ganze noch weiter
	vereinfacht.
	}
									   &= \sqrt{\frac{2l+1}{4\pi}} P_l(\cos \theta) \\
	\intertext{%
	Wir setzen die Formel von Rodriguez ein.
	}
									   &= \sqrt{\frac{2l+1}{4\pi}} \frac{1}{2^l l!} \dod[l]{}x \del{x^2-1}^l
\end{align*}

\subsection{Bestimmung der Koeffizienten}

Wir setzen unsere Funktion $\tilde u$ ein und bestimmen somit die
Koeffizienten. Dabei haben wir $\int \dif \phi$ schon zu $2 \pi$ ausgewertet
und gekürzt.
\[
K_l^0 = \sqrt{\del{2l+1} \pi} \frac{1}{2^l l!} R^4 \int_{-1}^1 \dif x \, x^4 \dod[l]{}x \del{x^2-1}^l
\]

An dieser Stelle müssen wir in $l$ eine Fallunterscheidung machen.

\paragraph{Fall $l = 0$}

Das Integral vereinfacht sich, weil das Polynom entfällt. Somit bleibt ein
Polynom, dessen Integral ist:
\[
\sbr{\frac 15 x^5}_{-1}^1
\]

Der erste Koeffizient ist damit dann:
\[
K_0^0 = \frac 25 \sqrt{\pi} R^4
\]

\paragraph{Fall $l > 0$}

Jetzt müssen wir partielle Integration benutzen, wir erhalten:
\[
K_l^0 = \underbrace{\sbr{x^4 \dod[l-1]{}x \del{x^2-1}^l}_{-1}^1}_{=0} - \int_{-1}^1 \dif x \, 4 x^3 \dod[l-1]{}x \del{x^2-1}^l
\]

Die eckige Klammer wird $0$. Dies können wir dadurch sehen, dass wir die
Produktregel $l-1$ mal anwenden und sich dann ein pascalsches Dreieck aus
Summanden bildet. Diese können wir analog zu den binomischen Formeln schreiben
als:
\begin{equation}
	\label{eq:leibniz}
	\dod[l]{}x \del{x^2-1}^l = l! \sum_{i=0}^l \binom{l}{i} (x-1)^i (x+1)^{l-i}
\end{equation}

Wenn wir dort $-1$ oder $1$ einsetzen, wird immer $0$ herauskommen. Somit sind
alle derartigen Klammern, die noch in den weiteren Fällen auftreten werden, 0
und wir werden sie daher in den nächsten Fällen weglassen.

Für den jetzt verbleibenden Integranden müssen wir erneut eine
Fallunterscheidung machen.

\paragraph{Fall $l = 1$}

In diesem Fall ist der Integrand wieder einfach und wir erhalten als Integral:
\[
\sbr{x^4}_{-1}^1 = 0
\]

Somit ist $R_1^0 = 0$. Alle ungerade Koeffizienten werden 0 sein, weil wir eine
gerade Funktion entwickeln.

\paragraph{Fall $l > 1$}

Wieder führen wir eine partielle Integration aus und erhalten:
\[
K_l^0 = \int_{-1}^1 \dif x \, 12 x^2 \dod[l-2]{}x \del{x^2-1}^l
\]

\paragraph{Fall $l = 2$}

Das Integral ist:
\[
\sbr{4x^3}_{-1}^1 = 8
\]

Daraus folgt für den Koeffizienten:
\[
K_2^0 = \sqrt{5 \pi} R^4
\]

\paragraph{Fall $l > 2$}

Durch partielle Integration erhalten wir:
\[
K_l^0 = - \int_{-1}^1 \dif x \, 24 x \dod[l-3]{}x \del{x^2-1}^l
\]

\paragraph{Fall $l = 3$}

Das Integral ist:
\[
\sbr{12 x^2}_{-1}^1 = 0
\]

\paragraph{Fall $l > 3$}

Partielle Integral liefert:
\[
K_l^0 = \int_{-1}^1 \dif x \, 24 \dod[l-4]{}x \del{x^2-1}^l
\]

\paragraph{Fall $l = 4$}

Das Integral ist:
\[
\sbr{24x}_{-1}^1 = 48
\]

Der Koeffizient ist:
\[
K_4^0 = \frac{3}{8} \sqrt{\pi} R^4
\]

\paragraph{Fall $l > 4$}

Eine weitere partielle Integration wird 0 ergeben. Also sind alle Koeffizienten
nach dem vierten gleich 0.

\subsection{Funktion zusammensetzen}

Wir stellen die interessanten Koeffizienten zusammen:
\[
K_0^0 = \frac 25 \sqrt{\pi} R^4
,\quad
K_2^0 = \sqrt{5 \pi} R^4
,\quad
K_4^0 = \frac{3}{8} \sqrt{\pi} R^4
\]

Somit können wir die Funktion als Linearkombination der Kugelflächenfunktionen
darstellen:
\[
\tilde u(x) = \del{\frac 25 P_0(x) + \sqrt{5} P_2(x) + \frac{3}{8} P_4(x)} \sqrt{\pi} R^4
\]

Der Sinn der ganzen Sache war jetzt der, dass die Kugelflächenfunktionen die
Laplacegleichung lösen. Wir fügen noch ein $\del{r/R}^3$ ein, damit wir eine
räumliche Kugelflächenfunktion erhalten. Somit die Lösung der Laplacegleichung
mit den Randbedingungen:
\[
u(r, \theta, \phi) = \del{\frac rR}^3 \del{\frac 25 P_0(\cos \theta) + \sqrt{5} P_2(\cos \theta) + \frac{3}{8} P_4(\cos \theta)} \sqrt{\pi} R^4
\]

Wir können noch $r = \sqrt{x^2+y^2+z^2}$ sowie $\cos \theta =: \hat z :=
\frac{z}{\sqrt{x^2+y^2+z^2}}$ einsetzen und erhalten so:
\[
	u(x, y, z) = \del{\frac{\sqrt{x^2+y^2+z^2}}R}^3
					   \del{\frac 25 P_0\del{\hat z} + \sqrt{5} P_2\del{\hat z} + \frac{3}{8} P_4\del{\hat z}}
		   \sqrt{\pi} R^4
\]

Außerdem können wir die Legendrepolynome einsetzen und erhalten:
\[
	u(x, y, z) = \sqrt{x^2+y^2+z^2}^3
	\del{\frac 25 + \sqrt{5} \half \del{3 \hat z^2-1} + \frac{3}{8} \frac 18 \del{35 \hat z^4 - 30 \hat z^2 + 3}}
		   \sqrt{\pi} R
\]

Wir fassen die Potenzen zusammen und erhalten:
\[
	u(x, y, z) = \sqrt{x^2+y^2+z^2}^3
		   \frac{\sqrt{\pi} R}{320} \del{173- 160 \sqrt{5} + \del{- 450 + 480 \sqrt{5}} \hat z^2 + 525 \hat z^4}
\]



%\bibliography{../../zentrale_BibTeX/Central}
%\bibliographystyle{plain}

\end{document}

% vim: spell spelllang=de
