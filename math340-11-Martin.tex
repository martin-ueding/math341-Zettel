% Copyright © 2012-2013 Martin Ueding <dev@martin-ueding.de>
%
% Copyright © 2012-2013 Martin Ueding <dev@martin-ueding.de>
%
\documentclass[11pt, ngerman, fleqn, DIV=15]{scrartcl}

\usepackage{graphicx}

%%%%%%%%%%%%%%%%%%%%%%%%%%%%%%%%%%%%%%%%%%%%%%%%%%%%%%%%%%%%%%%%%%%%%%%%%%%%%%%
%                                Locale, date                                 %
%%%%%%%%%%%%%%%%%%%%%%%%%%%%%%%%%%%%%%%%%%%%%%%%%%%%%%%%%%%%%%%%%%%%%%%%%%%%%%%

\usepackage{babel}
\usepackage[iso]{isodate}

%%%%%%%%%%%%%%%%%%%%%%%%%%%%%%%%%%%%%%%%%%%%%%%%%%%%%%%%%%%%%%%%%%%%%%%%%%%%%%%
%                          Margins and other spacing                          %
%%%%%%%%%%%%%%%%%%%%%%%%%%%%%%%%%%%%%%%%%%%%%%%%%%%%%%%%%%%%%%%%%%%%%%%%%%%%%%%

\usepackage[activate]{pdfcprot}
\usepackage[parfill]{parskip}
\usepackage{setspace}

\setlength{\columnsep}{2cm}

%%%%%%%%%%%%%%%%%%%%%%%%%%%%%%%%%%%%%%%%%%%%%%%%%%%%%%%%%%%%%%%%%%%%%%%%%%%%%%%
%                                    Color                                    %
%%%%%%%%%%%%%%%%%%%%%%%%%%%%%%%%%%%%%%%%%%%%%%%%%%%%%%%%%%%%%%%%%%%%%%%%%%%%%%%

\usepackage{color}

\definecolor{darkblue}{rgb}{0,0,.5}
\definecolor{darkgreen}{rgb}{0,.5,0}
\definecolor{darkred}{rgb}{.7,0,0}

%%%%%%%%%%%%%%%%%%%%%%%%%%%%%%%%%%%%%%%%%%%%%%%%%%%%%%%%%%%%%%%%%%%%%%%%%%%%%%%
%                         Font and font like settings                         %
%%%%%%%%%%%%%%%%%%%%%%%%%%%%%%%%%%%%%%%%%%%%%%%%%%%%%%%%%%%%%%%%%%%%%%%%%%%%%%%

\usepackage[charter, greekuppercase=italicized]{mathdesign}
\usepackage{beramono}
\usepackage{berasans}

% Style of vectors and tensors.
\newcommand{\tens}[1]{\boldsymbol{\mathsf{#1}}}
\renewcommand{\vec}[1]{\boldsymbol{#1}}

%%%%%%%%%%%%%%%%%%%%%%%%%%%%%%%%%%%%%%%%%%%%%%%%%%%%%%%%%%%%%%%%%%%%%%%%%%%%%%%
%                               Input encoding                                %
%%%%%%%%%%%%%%%%%%%%%%%%%%%%%%%%%%%%%%%%%%%%%%%%%%%%%%%%%%%%%%%%%%%%%%%%%%%%%%%

\usepackage[T1]{fontenc}
\usepackage[utf8]{inputenc}

%%%%%%%%%%%%%%%%%%%%%%%%%%%%%%%%%%%%%%%%%%%%%%%%%%%%%%%%%%%%%%%%%%%%%%%%%%%%%%%
%                         Hyperrefs and PDF metadata                          %
%%%%%%%%%%%%%%%%%%%%%%%%%%%%%%%%%%%%%%%%%%%%%%%%%%%%%%%%%%%%%%%%%%%%%%%%%%%%%%%

\usepackage{hyperref}
\usepackage{lastpage}

\hypersetup{
	breaklinks=false,
	citecolor=darkgreen,
	colorlinks=true,
	linkcolor=black,
	menucolor=black,
	pdfauthor={Martin Ueding},
	urlcolor=darkblue,
}

%%%%%%%%%%%%%%%%%%%%%%%%%%%%%%%%%%%%%%%%%%%%%%%%%%%%%%%%%%%%%%%%%%%%%%%%%%%%%%%
%                               Math Operators                                %
%%%%%%%%%%%%%%%%%%%%%%%%%%%%%%%%%%%%%%%%%%%%%%%%%%%%%%%%%%%%%%%%%%%%%%%%%%%%%%%

\usepackage[thinspace, squaren]{SIunits}
\usepackage{amsmath}
\usepackage{amsthm}
\usepackage{commath}

% Word like operators.
\DeclareMathOperator{\acosh}{arcosh}
\DeclareMathOperator{\arcosh}{arcosh}
\DeclareMathOperator{\arcsinh}{arsinh}
\DeclareMathOperator{\arsinh}{arsinh}
\DeclareMathOperator{\asinh}{arsinh}
\DeclareMathOperator{\card}{card}
\DeclareMathOperator{\diam}{diam}
\renewcommand{\Im}{\mathop{{}\mathrm{Im}}\nolimits}
\renewcommand{\Re}{\mathop{{}\mathrm{Re}}\nolimits}

% Special single letters.
\DeclareMathOperator{\fourier}{\mathcal{F}}
\newcommand{\C}{\ensuremath{\mathbb C}}
\newcommand{\ee}{\mathrm e}
\newcommand{\ii}{\mathrm i}
\newcommand{\N}{\ensuremath{\mathbb N}}
\newcommand{\R}{\ensuremath{\mathbb R}}
\newcommand{\Z}{\ensuremath{\mathbb Z}}

% Shape like operators.
\DeclareMathOperator{\dalambert}{\Box}
\DeclareMathOperator{\laplace}{\bigtriangleup}
\newcommand{\curl}{\vnabla \times}
\newcommand{\divergence}[1]{\inner{\vnabla}{#1}}
\newcommand{\vnabla}{\vec \nabla}

% Shortcuts
\newcommand{\ev}{\hat{\vec e}}
\newcommand{\e}[1]{\cdot 10^{#1}}
\newcommand{\half}{\frac 12}
\newcommand{\inner}[2]{\left\langle #1, #2 \right\rangle}

% Placeholders.
\newcommand{\emesswert}{\del{\messwert \pm \messwert}}
\newcommand{\fehlt}{\textcolor{darkred}{Hier fehlen noch Inhalte.}}
\newcommand{\messwert}{\textcolor{blue}{\square}}
\newcommand{\punkte}{\textcolor{white}{xxxxx}}

% Separator for equations on a single line.
\newcommand{\eqnsep}{,\quad}

%%%%%%%%%%%%%%%%%%%%%%%%%%%%%%%%%%%%%%%%%%%%%%%%%%%%%%%%%%%%%%%%%%%%%%%%%%%%%%%
%                                  Headings                                   %
%%%%%%%%%%%%%%%%%%%%%%%%%%%%%%%%%%%%%%%%%%%%%%%%%%%%%%%%%%%%%%%%%%%%%%%%%%%%%%%

\usepackage{scrpage2}

\pagestyle{scrheadings}

\automark{section}
\cfoot{\footnotesize{Seite \thepage\ / \pageref{LastPage}}}
\chead{}
\ihead{}
\ohead{\rightmark}
\setheadsepline{.4pt}

%%%%%%%%%%%%%%%%%%%%%%%%%%%%%%%%%%%%%%%%%%%%%%%%%%%%%%%%%%%%%%%%%%%%%%%%%%%%%%%
%                            Bibliography (BibTeX)                            %
%%%%%%%%%%%%%%%%%%%%%%%%%%%%%%%%%%%%%%%%%%%%%%%%%%%%%%%%%%%%%%%%%%%%%%%%%%%%%%%

\newcommand{\bibliographyfile}{../../zentrale_BibTeX/Central}


\usepackage{scrpage2}
\usepackage{tikz}

\newcommand{\themodul}{math340}
\newcommand{\thegruppe}{Gruppe 16 -- Malte Lackmann}
\newcommand{\theuebung}{11}

\pagestyle{scrheadings}

\automark{section}
\cfoot{\footnotesize{\thegruppe}}
\chead{}
\ifoot{\footnotesize{Ueding, Manz, Lemmer}}
\ihead{\themodul{} -- Übung \theuebung}
\ofoot{\footnotesize{Seite \thepage\ / \pageref{LastPage}}}
\ohead{\rightmark}
\setheadsepline{.4pt}


\def\thesection{\theuebung.\arabic{section}}
\def\thesubsection{\thesection\alph{subsection}}

\title{\themodul{} -- Übung \theuebung \\ \vspace{0.5cm} \large{\thegruppe}}

\author{
	Martin Ueding \\ \small{\href{mailto:mu@uni-bonn.de}{mu@uni-bonn.de}}
	\and
	Paul Manz
	\and
	Lino Lemmer
}

\begin{document}

\maketitle

\begin{table}[h]
	\centering
	\begin{tabular}{l|c|c|c|c|c|c}
		Aufgabe & \ref 1 & \ref 2 & \ref 3 & \ref 4 & $\sum$   \\
		\hline
		Punkte & \punkte / 8 & \punkte / 2 & \punkte / 2 & \punkte / 4 & \punkte / 16
	\end{tabular}
\end{table}

%%%%%%%%%%%%%%%%%%%%%%%%%%%%%%%%%%%%%%%%%%%%%%%%%%%%%%%%%%%%%%%%%%%%%%%%%%%%%%%
%                            holomorphe Funktionen                            %
%%%%%%%%%%%%%%%%%%%%%%%%%%%%%%%%%%%%%%%%%%%%%%%%%%%%%%%%%%%%%%%%%%%%%%%%%%%%%%%

\section{holomorphe Funktionen}
\label 1

\subsection{}

\begin{gather*}
	u = xy
	,\quad
	v = xy \\
	\dpd ux = y
	,\quad
	\dpd uy = x
	,\quad
	\dpd vx = y
	,\quad
	\dpd vy = x \\
	\dpd ux = \dpd vy
	\quad\wedge\quad
	\dpd uy = - \dpd vx \\
	y = y
	\quad\wedge\quad
	x = -y \\
	y = -y
	\quad\wedge\quad
	x = -x \\
	x = 0
	\quad\wedge\quad
	y = 0
\end{gather*}

Die Funktion ist nur im Punkt $z = 0$ komplex differenzierbar.

\stepcounter{subsection}
\subsection{}

\begin{gather*}
	u = -6 \cos\del x + 2 y^2 + 15\del{y^2 + 2y} \\
	v = -6 \sin\del x - 2y^3 \\
	\dpd ux = 6 \sin\del x
	,\quad
	\dpd uy = 6 y^2 + 15(2y +2)
	,\quad
	\dpd vx = - 6 \cos\del x
	,\quad
	\dpd vy = - 6 y^2
\end{gather*}

Das Gleichungssystem $\pd ux = \pd vy \wedge \pd uy = - \pd vx$ haben wir in
\emph{Mathematica} Lösen lassen, das Ergebnis passte nicht mehr auf einen
Bildschirm. Genähert erhalten wir folgende Lösungen (mit $n \in \Z$):
\begin{center}
\begin{tabular}{cc}
	$x$ & $y$ \\
	\hline
	$+2.50956 - 3.01879 \ii + 2 n \pi$ & $-1.44323 + 2.85358 \ii$ \\
	$+2.50956 + 3.01879 \ii + 2 n \pi$ & $-1.44323 - 2.85358 \ii$ \\
	$-0.93877 - 0.77286 \ii + 2 n \pi$ & $-1.05677 - 0.23817 \ii$ \\
	$-0.93877 + 0.77286 \ii + 2 n \pi$ & $-1.05677 + 0.23819 \ii$ \\
\end{tabular}
\end{center}

\fehlt

\stepcounter{subsection}
\subsection{}

\begin{gather*}
	f(z) = \sqrt{\abs{\Re z \Im z}} = \sqrt{\abs{xy}} \\
	u = \sqrt{\abs{xy}}
	,\quad
	v = 0
\end{gather*}

Die Ableitungen sind:
\[
	\dpd ux = \begin{cases}
		\frac{y}{2\sqrt{xy}} & xy > 0 \\
		\frac{-y}{2\sqrt{-xy}} & xy < 0 \\
						\infty & x = 0 \\
				 0 & y = 0
	\end{cases}
	,\quad
	\dpd uy = \begin{cases}
		\frac{x}{2\sqrt{xy}} & xy > 0 \\
		\frac{-x}{2\sqrt{-xy}} & xy < 0 \\
						\infty & y = 0 \\
				 0 & x = 0
	\end{cases}
	,\quad
	\dpd vx = 0
	,\quad
	\dpd vy = 0
\]

Die Cauchy-Riemann Differentialgleichung sind allerdings nur erfüllt für:
\[
	\del{y = 0 \wedge x \neq 0} \wedge \del{x = 0 \wedge y \neq 0}
\]

Somit ist die Funktion in keinem Punkt holomorph.
	
\subsection{}

\begin{gather*}
	f(z) = z \Re z = (x+\ii y)x = x^2 + \ii xy \\
	u = x^2
	,\quad
	v = xy \\
	\dpd ux = 2x
	,\quad
	\dpd uy = 0
	,\quad
	\dpd vx = y
	,\quad
	\dpd vy = x \\
	2x = 0
	\wedge
	y = 0
	\iff
	x = 0
	\wedge
	y = 0
\end{gather*}

%%%%%%%%%%%%%%%%%%%%%%%%%%%%%%%%%%%%%%%%%%%%%%%%%%%%%%%%%%%%%%%%%%%%%%%%%%%%%%%
%                                                                             %
%%%%%%%%%%%%%%%%%%%%%%%%%%%%%%%%%%%%%%%%%%%%%%%%%%%%%%%%%%%%%%%%%%%%%%%%%%%%%%%

\section{}
\label 2

\stepcounter{subsection}
\subsection{}

Gegeben ist:
\begin{gather*}
	u = x^2 - y^2 + \ee^{-y} \sin\del x - \ee^y \cos\del x \\
	\dpd ux = 2x + \ee^{-y} \cos\del x + \ee^y \sin\del x \\
	\dpd uy = -2y - \ee^{-y} \sin\del x - \ee^y \cos\del x
\end{gather*}

Es müssen die Cauchy-Riemann Differentialgleichung erfüllt sein:
\[
	\dpd ux = \dpd vy \wedge \dpd uy = - \dpd vx
\]

Unser Gleichungssystem:
\begin{align*}
	2x + \ee^{-y} \cos\del x + \ee^y \sin\del x &= \dpd vy \\
	2y + \ee^{-y} \sin\del x + \ee^y \cos\del x &= \dpd vx
\end{align*}

Wir integrieren die erste Gleichung nach $y$, die zweite nach $x$:
\begin{align*}
	v &= 2xy - \ee^{-y} \cos\del x + \ee^y \sin\del x + c_1(x) \\
	v &= 2xy - \ee^{-y} \cos\del x + \ee^y \sin\del x + c_2(y)
\end{align*}

Dies kann nur erfüllt sein, wenn gilt:
\[
	c_1(x) = c_2(y) \iff c_1(x) = c_2(y) = c_3
\]

Alle Funktionen $v$, so dass $f$ holomorph ist, sind:
\[
	v = 2xy - \ee^{-y} \cos\del x + \ee^y \sin\del x + c_3
\]

%%%%%%%%%%%%%%%%%%%%%%%%%%%%%%%%%%%%%%%%%%%%%%%%%%%%%%%%%%%%%%%%%%%%%%%%%%%%%%%
%                                                                             %
%%%%%%%%%%%%%%%%%%%%%%%%%%%%%%%%%%%%%%%%%%%%%%%%%%%%%%%%%%%%%%%%%%%%%%%%%%%%%%%

\section{}
\label 3

Sei $f(x + \ii y) = u(x, y) + \ii v(x, y)$. Nun gilt:
\begin{align*}
	g(z) &= \overline{f\del{\overline z}} = u(x, -y) - \ii v(x, -y) \\
g'(z) &= \dpd ux(x, -y) - \ii \dpd vx(x, -y) \\
			   &= \overline{f'\del{\overline z}}
\end{align*}

Für die Funktion $h(z)$ gilt ähnlich:
\begin{align*}
	h(z) &= \overline{f\del{z}} = u(x, y) - \ii v(x, y) \\
h'(z) &= \dpd ux(x, y) - \ii \dpd vx(x, y) \\
			   &= \overline{f'\del{z}}
\end{align*}

\stepcounter{section}
\section{}
\label 4

%\bibliography{../../zentrale_BibTeX/Central}
%\bibliographystyle{plain}

\end{document}

% vim: spell spelllang=de
