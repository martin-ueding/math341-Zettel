% Copyright © 2012 Martin Ueding <dev@martin-ueding.de>
%
\documentclass[11pt, ngerman]{article}

\usepackage[a4paper, left=3cm, right=2cm, top=2cm, bottom=2cm]{geometry}
\usepackage[activate]{pdfcprot}
\usepackage[cdot, squaren]{SIunits}
\usepackage[iso]{isodate}
\usepackage[parfill]{parskip}
\usepackage[T1]{fontenc}
\usepackage[utf8]{inputenc}
\usepackage{amsmath}
\usepackage{amssymb}
\usepackage{amsthm}
\usepackage{babel}
\usepackage{color}
\usepackage{commath}
\usepackage{epstopdf}
\usepackage{fancyhdr}
\usepackage{graphicx}
\usepackage{hyperref}
\usepackage{setspace}

\usepackage[charter]{mathdesign}

\definecolor{darkblue}{rgb}{0,0,.5}
\definecolor{darkgreen}{rgb}{0,.5,0}

\hypersetup{
	breaklinks=false,
	citecolor=darkgreen,
	colorlinks=true,
	linkcolor=black,
	menucolor=black,
	urlcolor=darkblue,
}

\setlength{\columnsep}{2cm}

\DeclareMathOperator{\arcsinh}{arsinh}
\DeclareMathOperator{\arsinh}{arsinh}
\DeclareMathOperator{\asinh}{arsinh}
\DeclareMathOperator{\card}{card}
\DeclareMathOperator{\diam}{diam}

\newcommand{\emesswert}{\del{\messwert \pm \messwert}}
\newcommand{\e}[1]{\cdot 10^{#1}}
\newcommand{\fehlt}{\textcolor{red}{Hier fehlen noch Inhalte.}}
\newcommand{\half}{\frac 12}
\newcommand{\laplace}{\vnabla^2}
\newcommand{\messwert}{\textcolor{blue}{\square}}
\newcommand{\punkte}{\textcolor{white}{xxxxx}}
\newcommand{\vnabla}{\vec \nabla}

\newcommand{\themodul}{math340}
\newcommand{\thegruppe}{Gruppe 16 -- Malte Lackmann}
\newcommand{\theuebung}{1}

\pagestyle{fancy}

\fancyfoot[C]{\footnotesize{\thegruppe}}
\fancyfoot[L]{\footnotesize{Martin Ueding}}
\fancyfoot[R]{\footnotesize{Seite \thepage}}
\fancyhead[L]{\themodul{} -- Übung \theuebung}

\setcounter{section}{0}

\def\thesection{\theuebung.\arabic{section}}
\def\thesubsection{\thesection\alph{subsection}}

\title{\themodul{} -- Übung \theuebung \\ \vspace{0.5cm} \large{\thegruppe}}

\author{Martin Ueding \\ \small{\href{mailto:mu@uni-bonn.de}{mu@uni-bonn.de}}}

\begin{document}

\maketitle

\begin{table}[h]
	\centering
	\begin{tabular}{l|c|c|c|c}
		Aufgabe & 1.1 & 1.2 & 1.3 & $\Sigma$   \\
		\hline
		Punkte & \punkte / 4 & \punkte / 4 & \punkte / 2 & \punkte / 10
	\end{tabular}
\end{table}

\section{Differentialgleichungen}

Ich gehe davon aus dass $y'$ die Funktion $\pd yx(x)$ meint.

\subsection{die Einfache}

Dies geht mit der Separation der Variablen:
%
\begin{align*}
	\dod yx &= \exp\del{2x-y} \\
	\dif y \exp\del y &= \exp\del{2x} \dif x \\
	\int \dif y \exp \del y &= \int \dif x \exp\del{2x} \\
	\exp\del y &= \frac 12 \exp\del{2x} + c_1 \\
			 y &= \ln\del{\frac 12 \exp\del{2x} + c_1}
\end{align*}

\stepcounter{subsection}

\subsection{die Schwere}

Es sollen alle Lösungen zu folgender Differentialgleichung bestimmt werden:
\[
	\od[2] yx + 4y = x^2 + 5 \cos\del{2x}
\]

\newcommand\ex{\exp\del{2ix}}

Ich beginne damit, die homogene Differentialgleichung zu lösen. Diese ist
einfach mit einem Exponentialansatz gelöst. Die Integralbasis $B$ für die
homogene Gleichung ist:
\[
	B = \set{\ex, \exp\del{-2ix}}
\]

Nun muss ich noch eine partikuläre Lösung bestimmen. Dazu wähle ich den Ansatz
der Variiation der Konstanten. Die Funktionen in der Basis bezeichne ich mit
$U_1$ und $u_2$. Das Gleichungssystem ist:
\begin{align*}
	\dot c_1 u_1 + \dot c_2 u_2 &= 0 \\
	\dot c_1 \dot u_1 + \dot c_2 \dot u_2 &= x^2 + 5 \cos(2x)
\end{align*}

Dort setze ich die beiden Basisfunktionen ein und löse mit dem Gaußalgorithmus:
\[
	\begin{pmatrix}
		\exp(2ix) & \exp(-2ix) & 0 \\
		2i \exp(2ix) & -2i \exp(-2ix) & x^2 + 5 \cos(2x)
	\end{pmatrix}
	\leadsto
	\begin{pmatrix}
		0 & \exp(-2ix) & - \frac1{4i} \del{ x^2 + 5 \cos(2x)} \\
		\exp(2ix) & 0 & \frac1{4i} \del{ x^2 + 5 \cos(2x)}
	\end{pmatrix}
\]

Jetzt integriere ich nach $x$ um $c_1$ zu erhalten:
\begin{align*}
	\dot c_1 \exp(2ix) &= \frac1{4i} \del{ x^2 + 5 \cos(2x)} \\
	\dot c_1 &= \frac1{4i} \exp(-2ix) \del{ x^2 + 5 \cos(2x)} \\
	c_1 &= \frac1{4i} \int \dif x \exp(-2ix) \del{ x^2 + 5 \cos(2x)} \\
	\intertext{
		Dieses Integral kann ich so lösen, allerdings erhalte ich dann eine
		komplexe Funktion. Da alle Koeffizienten real sind, gehe ich davon aus,
		dass hier am Ende nur der Realteil der Funktion interessant ist. Durch
		das $\frac 1i$ muss ich den Imaginärteil des Integrals betrachten, das
		ist $i \sin(-2x)$ der Exponentialfunktion.
	}
	c_1 &= \frac14 \int \dif x \sin(-2x) \del{ x^2 + 5 \cos(2x)} \\
	\intertext{Dieses Integral lässt sich durch (mehrfache) partielle Integration lösen. Ich erhalte:}
	c_1 &= \frac 1{16} \del{\del{-1+2x^2} \cos(2x) + 5 \cos^2(2x) - 2x \sin(2x)}
\end{align*}

Da der Sinus punktsymmetrisch ist, wird das gleiche Ergebnis auch für $c_2$
herauskommen. Das Minus auf der rechten Seite wird gerade durch das Minus im
Exponenten der Exponentialfunktion ausgeglichen. Es läuft auf das gleiche
Integral heraus.

Dann ist meine partikuläre Lösung allerdings:
\[
	y_p(x) = 
\]

\section{Laplace-Operator}

\subsection{Polarkoordinaten}

Gegeben ist eine Funktion:
\[
	U(r, \phi) := u\del{r \cos(\phi), r \sin(\phi)}
\]

Es ist zu zeigen, dass gilt:
\[
	\laplace u = \pd[2] ux + \pd[2] uy = \pd[2] Ur + \frac 1r \pd Ur + \frac
	1{r^2} \pd[2] U\phi
\]

Ich beginne mit $\pd U r$:
\[
	\pd Ur = \od ur = \pd ux \cos(\phi) + \pd uy \sin(\phi)
\]

Dies leite ich noch einmal nach $r$ ab. Dabei kann ich den Kosinus und Sinus
jeweils vor die Ableitung ziehen, da diese nicht von $r$ abhängen.
\[
	\pd[2] Ur = \od[2] ur = \pd[2] ux \cos^2(\phi) + \pd[2] uy \sin^2(\phi)
\]

Zuletzte bestimme ich $\pd[2] U\phi$:
\[
	\pd[2] U\phi = \od[2] u\phi = \pd[2] ux r^2 \sin^2(\phi) - \pd ux r
	\cos(\phi) + \pd[2] uy r^2 \cos^2(\phi) - r \pd uy \sin(\phi)
\]

Diese Teile setze ich jetzt ein und fasse Terme mit $\cos^2(\phi) +
\sin^2(\phi) = 1$ zusammen:
%
\begin{align*}
	\dpd[2] Ur + \frac 1r \dpd Ur + \frac 1{r^2} \dpd[2] U\phi
	&=\dpd[2] ux \cos^2(\phi) + \dpd[2] uy \sin^2(\phi) \\
	&\quad+ \frac 1r \dpd ux \cos(\phi) + \frac 1r \dpd uy \sin(\phi) \\
	&\quad+ \dpd[2] ux \sin^2(\phi) - \frac 1r \dpd ux r \cos(\phi) + \dpd[2]
	uy \cos^2(\phi) - \frac 1r \dpd uy \sin(\phi) \\
	&= \dpd[2] ux + \dpd[2] uy
\end{align*}

Somit gilt die Relation.

\section{harmonische Funktion}

\subsection{Überprüfung}

Es ist zu zeigen, dass $f = \frac 1r$ eine harmonische Funktion ist.

Ich bestimme die Ableitungen der Funktion $f$:
\[
	\vnabla f = \frac{1}{r} \begin{pmatrix}
		x \\ y \\ z
	\end{pmatrix} = \frac{\vec r}{r}
	,\quad
	\laplace f = \sum_i \del{- \frac12 \frac{2x_i}{r^3} x_i + \frac 1r 1} = 0
\]

$\laplace f = 0$ gilt also.

\end{document}
