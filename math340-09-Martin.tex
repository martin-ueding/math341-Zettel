% Copyright © 2012 Martin Ueding <dev@martin-ueding.de>
%
\input{header.tex}

\usepackage{fancyhdr}
\usepackage{tikz}

\newcommand{\themodul}{math340}
\newcommand{\thegruppe}{Gruppe 16 -- Malte Lackmann}
\newcommand{\theuebung}{9}

\pagestyle{fancy}

\fancyfoot[C]{\footnotesize{\thegruppe}}
\fancyfoot[L]{\footnotesize{Ueding, Manz, Lemmer}}
\fancyfoot[R]{\footnotesize{Seite \thepage\ / \pageref{LastPage}}}
\fancyhead[L]{\themodul{} -- Übung \theuebung}

\def\thesection{\theuebung.\arabic{section}}
\def\thesubsection{\thesection\alph{subsection}}

\title{\themodul{} -- Übung \theuebung \\ \vspace{0.5cm} \large{\thegruppe}}

\author{
	Martin Ueding \\ \small{\href{mailto:mu@uni-bonn.de}{mu@uni-bonn.de}}
	\and
	Paul Manz
	\and
	Lino Lemmer
}

\begin{document}

\maketitle

\begin{table}[h]
	\centering
	\begin{tabular}{l|c|c|c|c|c}
		Aufgabe & \ref 1 & \ref 2 & \ref 3 & $\sum$   \\
		\hline
		Punkte & \punkte / 7 & \punkte / 5 & \punkte 3 & \punkte / 15
	\end{tabular}
\end{table}

%%%%%%%%%%%%%%%%%%%%%%%%%%%%%%%%%%%%%%%%%%%%%%%%%%%%%%%%%%%%%%%%%%%%%%%%%%%%%%%
%                            Fouriertransformierte                            %
%%%%%%%%%%%%%%%%%%%%%%%%%%%%%%%%%%%%%%%%%%%%%%%%%%%%%%%%%%%%%%%%%%%%%%%%%%%%%%%

\section{Fouriertransformierte}
\label 1

\subsection{}

{
	\renewcommand \exp[1]{\mathrm e^{#1}}

	Gesucht ist die Fouriertransformierte von $f_a\del x=\exp{-a\abs x}$ für
	$a>0$. Ich berechne
	\begin{align*}
		\mathcal F\del{f}\del\omega&=\frac{1}{\sqrt{2\pi}}\int_{-\infty}^\infty f_a\del x\cdot \exp{-i\omega x}\dif x\\
												 &=\frac{1}{\sqrt{2\pi}}\int_{-\infty}^\infty\exp{-a\abs x}\cdot \exp{-i\omega x}\dif x\\
						 &=\frac{1}{\sqrt{2\pi}}\int_{-\infty}^\infty\exp{-a\abs x-i\omega x}\dif x\\
				&=\frac{1}{\sqrt{2\pi}}\del{\int_{-\infty}^0\exp{\del{a-i\omega}x}\dif x+\int_0^\infty\exp{-\del{a+i\omega}x}\dif x}\\
			 &=\frac{1}{\sqrt{2\pi}}\del{\left.\frac{1}{a-i\omega}\exp{\del{a-i\omega}x}\right|_{-\infty}^0-\left.\frac{1}{a+i\omega}\exp{-\del{a+i\omega}x}\right|_0^{\infty}}\\
			 &=\frac{1}{\sqrt{2\pi}}\del{\frac{1}{a-i\omega}+\frac{1}{a+i\omega}}\\
			 &=\frac{1}{\sqrt{2\pi}}\frac{2a}{a^2+\omega^2}\\
			 &=\frac{\sqrt{2}}{\pi}\frac{a}{a^2+\omega^2}
		\intertext{Die Gesamtenergie ergibt sich zu}
		\int_{-\infty}^\infty \abs{f_a\del{x}}^2\dif x&=\int_{-\infty}^\infty \del{\exp{-a\abs{x}}}^2\dif x\\
																						  &=\int_{-\infty}^\infty \del{\exp{-2a\abs{x}}}^2\dif x\\
									   &=\int_{-\infty}^0 \exp{2ax}\dif x+\int_{0}^\infty \exp{-2ax}\dif x\\
						&=\left.\frac{1}{2a}\exp{2ax}\right|_{-\infty}^0-\left.\frac{1}{2a}\exp{-2ax}\right|_{0}^\infty\\
			   &=\frac 1a
	\end{align*}
}

\subsection{}

\fehlt

%%%%%%%%%%%%%%%%%%%%%%%%%%%%%%%%%%%%%%%%%%%%%%%%%%%%%%%%%%%%%%%%%%%%%%%%%%%%%%%
%                              Dirichletproblem                               %
%%%%%%%%%%%%%%%%%%%%%%%%%%%%%%%%%%%%%%%%%%%%%%%%%%%%%%%%%%%%%%%%%%%%%%%%%%%%%%%

\section{Dirichletproblem}
\label 2

Gelöst werden soll die Laplacegleichung mit Randbedingung:
\[
	\dpd[2] ux + \dpd[2] uy = 0
	\quad\wedge\quad
	u(x, 0) = f(x)
\]

Wir setzen:
\[
	W(\omega, y) := \del{\fourier_x u}(\omega, y)
	,\quad
	W_0(\omega) := \del{\fourier_x f}(\omega)
	,\quad
	\dpd[2] Wx := \fourier_x\del{\dpd[2] ux} = - \omega^2 W
\]

Wir wenden die Fouriertransformation in $x \to \omega$, $\fourier_x$, auf die
Differentialgleichung an und erhalten (unter Benutzung der Linearität):
\[
	\fourier_x\del{\dpd[2] ux} + \fourier_x \del{\dpd[2] uy} = 0
	\iff
	- \omega^2 W + \dpd[2] Wy = 0
\]

Diese Differentialgleichung können wir durch einen Exponentialansatz lösen:
\[
	W(\omega, y) = \del{c_1 \ee^{\ii \omega y} + c_2 \ee^{-\ii \omega y}} W_0(\omega)
\]

Dies müssen wir rücktransformieren um eine Lösung $u$ zu erhalten:
\begin{align*}
	u(x, y)
	&= \fourier_x^{-1} \del{\del{c_1 \ee^{\ii \omega y} + c_2 \ee^{-\ii \omega y}} W_0(\omega)} \\
	&= \frac 1{\sqrt{2\pi}} \fourier_x^{-1} \del{c_1 \ee^{\ii \omega y} + c_2 \ee^{-\ii \omega y}} * \fourier_x^{-1} W_0(\omega) \\
	&= \frac 1{\sqrt{2\pi}} \fourier_x^{-1} \del{c_1 \ee^{\ii \omega y} + c_2 \ee^{-\ii \omega y}} * f(x) \\
	&= \frac 1{2\pi} \del{
		c_1 \int_{-\infty}^\infty \dif \omega \, \ee^{\ii \omega y} \ee^{\ii \omega x}
		+ c_2 \int_{-\infty}^\infty \dif \omega \, \ee^{-\ii \omega y} \ee^{\ii \omega x}
	} * f(x) \\
	&= \frac 1{2\pi} \del{
		c_1 \int_{-\infty}^\infty \dif \omega \, \ee^{\ii \omega (x+y)}
		+ c_2 \int_{-\infty}^\infty \dif \omega \, \ee^{\ii \omega (x-y)}
	} * f(x) \\
	&= \frac 1{2\pi} \del{
	\frac{c_1}{\ii (x+y)} \sbr{\ee^{\ii \omega (x+y)}}_{-\infty}^\infty
	+ \frac{c_2}{\ii (x-y)} \sbr{\ee^{\ii \omega (x-y)}}_{-\infty}^\infty
	} * f(x) \\
\end{align*}

\fehlt

%%%%%%%%%%%%%%%%%%%%%%%%%%%%%%%%%%%%%%%%%%%%%%%%%%%%%%%%%%%%%%%%%%%%%%%%%%%%%%%
%                                 Zeitsignal                                  %
%%%%%%%%%%%%%%%%%%%%%%%%%%%%%%%%%%%%%%%%%%%%%%%%%%%%%%%%%%%%%%%%%%%%%%%%%%%%%%%

\section{Zeitsignal}
\label 3

Es soll die Gesamtenergie eines Signals mit Spektralfunktion $F$ bestimmt
werden:
\[
	F(\omega) = \frac{\sqrt{\abs \omega}}{1+\omega^2}
\]

Wir berechnen die Gesamtenergie:
\begin{align*}
	W
	&= \int_{-\infty}^{\infty} \dif t \, \abs{f(t)}^2 \\
	\intertext{%
		Praktischerweise können wir hier die Fouriertransformation auf die
		Funktion anwenden, ohne dass sich das Integral ändert (Plancherel).
	}
	&= \int_{-\infty}^{\infty} \dif \omega \, \abs{\del{\fourier f}(\omega)}^2 \\
	\intertext{%
		Dies ist gerade die gegebene Spektralfunktion.
	}
	&= \int_{-\infty}^{\infty} \dif \omega \, \abs{F(\omega)}^2
	= \int_{-\infty}^{\infty} \dif \omega \, \frac{\abs \omega}{\del{1+\omega^2}^2}
	= 2 \int_0^{\infty} \dif \omega \, \frac{\omega}{\del{1+\omega^2}^2} \\
	\intertext{%
		Wir substituieren $z(\omega) := 1 + \omega^2$ mit $\dif z / (2 \omega) = \dif \omega$.
	}
	&= \int_{z(0)}^{z(\infty)} \dif z \, \frac{1}{z^2}
	= \sbr{- \frac{1}{z}}_{z(0)}^{z(\infty)}
	= \sbr{- \frac{1}{1 + \omega^2}}_{0}^{\infty}
	= 1
\end{align*}

%\bibliography{../../zentrale_BibTeX/Central}
%\bibliographystyle{plain}

\end{document}

% vim: spell spelllang=de
