% Copyright © 2012 Martin Ueding <dev@martin-ueding.de>
%
% Copyright © 2012-2013 Martin Ueding <dev@martin-ueding.de>
%
\documentclass[11pt, ngerman, fleqn, DIV=15]{scrartcl}

\usepackage{graphicx}

%%%%%%%%%%%%%%%%%%%%%%%%%%%%%%%%%%%%%%%%%%%%%%%%%%%%%%%%%%%%%%%%%%%%%%%%%%%%%%%
%                                Locale, date                                 %
%%%%%%%%%%%%%%%%%%%%%%%%%%%%%%%%%%%%%%%%%%%%%%%%%%%%%%%%%%%%%%%%%%%%%%%%%%%%%%%

\usepackage{babel}
\usepackage[iso]{isodate}

%%%%%%%%%%%%%%%%%%%%%%%%%%%%%%%%%%%%%%%%%%%%%%%%%%%%%%%%%%%%%%%%%%%%%%%%%%%%%%%
%                          Margins and other spacing                          %
%%%%%%%%%%%%%%%%%%%%%%%%%%%%%%%%%%%%%%%%%%%%%%%%%%%%%%%%%%%%%%%%%%%%%%%%%%%%%%%

\usepackage[activate]{pdfcprot}
\usepackage[parfill]{parskip}
\usepackage{setspace}

\setlength{\columnsep}{2cm}

%%%%%%%%%%%%%%%%%%%%%%%%%%%%%%%%%%%%%%%%%%%%%%%%%%%%%%%%%%%%%%%%%%%%%%%%%%%%%%%
%                                    Color                                    %
%%%%%%%%%%%%%%%%%%%%%%%%%%%%%%%%%%%%%%%%%%%%%%%%%%%%%%%%%%%%%%%%%%%%%%%%%%%%%%%

\usepackage{color}

\definecolor{darkblue}{rgb}{0,0,.5}
\definecolor{darkgreen}{rgb}{0,.5,0}
\definecolor{darkred}{rgb}{.7,0,0}

%%%%%%%%%%%%%%%%%%%%%%%%%%%%%%%%%%%%%%%%%%%%%%%%%%%%%%%%%%%%%%%%%%%%%%%%%%%%%%%
%                         Font and font like settings                         %
%%%%%%%%%%%%%%%%%%%%%%%%%%%%%%%%%%%%%%%%%%%%%%%%%%%%%%%%%%%%%%%%%%%%%%%%%%%%%%%

\usepackage[charter, greekuppercase=italicized]{mathdesign}
\usepackage{beramono}
\usepackage{berasans}

% Style of vectors and tensors.
\newcommand{\tens}[1]{\boldsymbol{\mathsf{#1}}}
\renewcommand{\vec}[1]{\boldsymbol{#1}}

%%%%%%%%%%%%%%%%%%%%%%%%%%%%%%%%%%%%%%%%%%%%%%%%%%%%%%%%%%%%%%%%%%%%%%%%%%%%%%%
%                               Input encoding                                %
%%%%%%%%%%%%%%%%%%%%%%%%%%%%%%%%%%%%%%%%%%%%%%%%%%%%%%%%%%%%%%%%%%%%%%%%%%%%%%%

\usepackage[T1]{fontenc}
\usepackage[utf8]{inputenc}

%%%%%%%%%%%%%%%%%%%%%%%%%%%%%%%%%%%%%%%%%%%%%%%%%%%%%%%%%%%%%%%%%%%%%%%%%%%%%%%
%                         Hyperrefs and PDF metadata                          %
%%%%%%%%%%%%%%%%%%%%%%%%%%%%%%%%%%%%%%%%%%%%%%%%%%%%%%%%%%%%%%%%%%%%%%%%%%%%%%%

\usepackage{hyperref}
\usepackage{lastpage}

\hypersetup{
	breaklinks=false,
	citecolor=darkgreen,
	colorlinks=true,
	linkcolor=black,
	menucolor=black,
	pdfauthor={Martin Ueding},
	urlcolor=darkblue,
}

%%%%%%%%%%%%%%%%%%%%%%%%%%%%%%%%%%%%%%%%%%%%%%%%%%%%%%%%%%%%%%%%%%%%%%%%%%%%%%%
%                               Math Operators                                %
%%%%%%%%%%%%%%%%%%%%%%%%%%%%%%%%%%%%%%%%%%%%%%%%%%%%%%%%%%%%%%%%%%%%%%%%%%%%%%%

\usepackage[thinspace, squaren]{SIunits}
\usepackage{amsmath}
\usepackage{amsthm}
\usepackage{commath}

% Word like operators.
\DeclareMathOperator{\acosh}{arcosh}
\DeclareMathOperator{\arcosh}{arcosh}
\DeclareMathOperator{\arcsinh}{arsinh}
\DeclareMathOperator{\arsinh}{arsinh}
\DeclareMathOperator{\asinh}{arsinh}
\DeclareMathOperator{\card}{card}
\DeclareMathOperator{\diam}{diam}
\renewcommand{\Im}{\mathop{{}\mathrm{Im}}\nolimits}
\renewcommand{\Re}{\mathop{{}\mathrm{Re}}\nolimits}

% Special single letters.
\DeclareMathOperator{\fourier}{\mathcal{F}}
\newcommand{\C}{\ensuremath{\mathbb C}}
\newcommand{\ee}{\mathrm e}
\newcommand{\ii}{\mathrm i}
\newcommand{\N}{\ensuremath{\mathbb N}}
\newcommand{\R}{\ensuremath{\mathbb R}}
\newcommand{\Z}{\ensuremath{\mathbb Z}}

% Shape like operators.
\DeclareMathOperator{\dalambert}{\Box}
\DeclareMathOperator{\laplace}{\bigtriangleup}
\newcommand{\curl}{\vnabla \times}
\newcommand{\divergence}[1]{\inner{\vnabla}{#1}}
\newcommand{\vnabla}{\vec \nabla}

% Shortcuts
\newcommand{\ev}{\hat{\vec e}}
\newcommand{\e}[1]{\cdot 10^{#1}}
\newcommand{\half}{\frac 12}
\newcommand{\inner}[2]{\left\langle #1, #2 \right\rangle}

% Placeholders.
\newcommand{\emesswert}{\del{\messwert \pm \messwert}}
\newcommand{\fehlt}{\textcolor{darkred}{Hier fehlen noch Inhalte.}}
\newcommand{\messwert}{\textcolor{blue}{\square}}
\newcommand{\punkte}{\textcolor{white}{xxxxx}}

% Separator for equations on a single line.
\newcommand{\eqnsep}{,\quad}

%%%%%%%%%%%%%%%%%%%%%%%%%%%%%%%%%%%%%%%%%%%%%%%%%%%%%%%%%%%%%%%%%%%%%%%%%%%%%%%
%                                  Headings                                   %
%%%%%%%%%%%%%%%%%%%%%%%%%%%%%%%%%%%%%%%%%%%%%%%%%%%%%%%%%%%%%%%%%%%%%%%%%%%%%%%

\usepackage{scrpage2}

\pagestyle{scrheadings}

\automark{section}
\cfoot{\footnotesize{Seite \thepage\ / \pageref{LastPage}}}
\chead{}
\ihead{}
\ohead{\rightmark}
\setheadsepline{.4pt}

%%%%%%%%%%%%%%%%%%%%%%%%%%%%%%%%%%%%%%%%%%%%%%%%%%%%%%%%%%%%%%%%%%%%%%%%%%%%%%%
%                            Bibliography (BibTeX)                            %
%%%%%%%%%%%%%%%%%%%%%%%%%%%%%%%%%%%%%%%%%%%%%%%%%%%%%%%%%%%%%%%%%%%%%%%%%%%%%%%

\newcommand{\bibliographyfile}{../../zentrale_BibTeX/Central}


\usepackage{fancyhdr}
\usepackage{tikz}

\newcommand{\themodul}{math340}
\newcommand{\thegruppe}{Gruppe 16 -- Malte Lackmann}
\newcommand{\theuebung}{9}

\pagestyle{fancy}

\fancyfoot[C]{\footnotesize{\thegruppe}}
\fancyfoot[L]{\footnotesize{Ueding, Manz, Lemmer}}
\fancyfoot[R]{\footnotesize{Seite \thepage\ / \pageref{LastPage}}}
\fancyhead[L]{\themodul{} -- Übung \theuebung}

\def\thesection{\theuebung.\arabic{section}}
\def\thesubsection{\thesection\alph{subsection}}

\title{\themodul{} -- Übung \theuebung \\ \vspace{0.5cm} \large{\thegruppe}}

\author{
	Martin Ueding \\ \small{\href{mailto:mu@uni-bonn.de}{mu@uni-bonn.de}}
	\and
	Paul Manz
	\and
	Lino Lemmer
}

\begin{document}

\maketitle

\begin{table}[h]
	\centering
	\begin{tabular}{l|c|c|c|c|c}
		Aufgabe & \ref 1 & \ref 2 & \ref 3 & $\sum$   \\
		\hline
		Punkte & \punkte / 7 & \punkte / 5 & \punkte 3 & \punkte / 15
	\end{tabular}
\end{table}

%%%%%%%%%%%%%%%%%%%%%%%%%%%%%%%%%%%%%%%%%%%%%%%%%%%%%%%%%%%%%%%%%%%%%%%%%%%%%%%
%                            Fouriertransformierte                            %
%%%%%%%%%%%%%%%%%%%%%%%%%%%%%%%%%%%%%%%%%%%%%%%%%%%%%%%%%%%%%%%%%%%%%%%%%%%%%%%

\section{Fouriertransformierte}
\label 1

\subsection{}

\fehlt

\subsection{}

\fehlt

%%%%%%%%%%%%%%%%%%%%%%%%%%%%%%%%%%%%%%%%%%%%%%%%%%%%%%%%%%%%%%%%%%%%%%%%%%%%%%%
%                              Dirichletproblem                               %
%%%%%%%%%%%%%%%%%%%%%%%%%%%%%%%%%%%%%%%%%%%%%%%%%%%%%%%%%%%%%%%%%%%%%%%%%%%%%%%

\section{Dirichletproblem}
\label 2

Gelöst werden soll die Laplacegleichung mit Randbedingung:
\[
	\dpd[2] ux + \dpd[2] uy = 0
	\quad\wedge\quad
	u(x, 0) = f(x)
\]

Wir setzen:
\[
	W(\omega, y) := \fourier_x(u)(\omega, y)
	,\quad
	W_0(\omega) := \fourier_x\del{f}(\omega)
	,\quad
	\dpd[2] Wx := \fourier_x\del{\dpd[2] ux} = - \omega^2 W
\]

Wir wenden die Fouriertransformation in $x \to \omega$, $\fourier_x$, auf die
Differentialgleichung an und erhalten (unter Benutzung der Linearität):
\[
	\fourier_x\del{\dpd[2] ux} + \fourier_x \del{\dpd[2] uy} = 0
	\iff
	- \omega^2 W + \dpd[2] Wy = 0
\]

Diese Differentialgleichung können wir durch einen Exponentialansatz lösen:
\[
	W(\omega, y) = \del{c_1 \ee^{\ii \omega y} + c_2 \ee^{-\ii \omega y}} W_0(\omega)
\]

Dies müssen wir rücktransformieren um eine Lösung $u$ zu erhalten:
\begin{align*}
	u(x, y)
	&= \fourier_x^{-1} \del{\del{c_1 \ee^{\ii \omega y} + c_2 \ee^{-\ii \omega y}} W_0(\omega)} \\
	&= \frac 1{\sqrt{2\pi}} \fourier_x^{-1} \del{c_1 \ee^{\ii \omega y} + c_2 \ee^{-\ii \omega y}} * \fourier_x^{-1} W_0(\omega) \\
	&= \frac 1{\sqrt{2\pi}} \fourier_x^{-1} \del{c_1 \ee^{\ii \omega y} + c_2 \ee^{-\ii \omega y}} * f(x) \\
	&= \frac 1{2\pi} \del{
		c_1 \int_{-\infty}^\infty \dif \omega \, \ee^{\ii \omega y} \ee^{\ii \omega x}
		+ c_2 \int_{-\infty}^\infty \dif \omega \, \ee^{-\ii \omega y} \ee^{\ii \omega x}
	} * f(x) \\
	&= \frac 1{2\pi} \del{
		c_1 \int_{-\infty}^\infty \dif \omega \, \ee^{\ii \omega (x+y)}
		+ c_2 \int_{-\infty}^\infty \dif \omega \, \ee^{\ii \omega (x-y)}
	} * f(x) \\
	&= \frac 1{2\pi} \del{
	\frac{c_1}{\ii (x+y)} \sbr{\ee^{\ii \omega (x+y)}}_{-\infty}^\infty
	+ \frac{c_2}{\ii (x-y)} \sbr{\ee^{\ii \omega (x-y)}}_{-\infty}^\infty
	} * f(x) \\
\end{align*}

\fehlt

%%%%%%%%%%%%%%%%%%%%%%%%%%%%%%%%%%%%%%%%%%%%%%%%%%%%%%%%%%%%%%%%%%%%%%%%%%%%%%%
%                                 Zeitsignal                                  %
%%%%%%%%%%%%%%%%%%%%%%%%%%%%%%%%%%%%%%%%%%%%%%%%%%%%%%%%%%%%%%%%%%%%%%%%%%%%%%%

\section{Zeitsignal}
\label 3

Es soll die Gesamtenergie eines Signals mit Spektralfunktion $F$ bestimmt
werden:
\[
	F(\omega) = \frac{\sqrt{\abs \omega}}{1+\omega^2}
\]

Wir berechnen die Gesamtenergie:
\begin{align*}
	W
	&= \int_{-\infty}^{\infty} \dif t \, \abs{f(t)}^2 \\
	\intertext{%
		Praktischerweise können wir hier die Fouriertransformation auf die
		Funktion anwenden, ohne dass sich das Integral ändert.
	}
	&= \int_{-\infty}^{\infty} \dif \omega \, \abs{\del{\fourier f}(\omega)}^2 \\
	\intertext{%
		Dies ist gerade die gegebene Spektralfunktion.
	}
	&= \int_{-\infty}^{\infty} \dif \omega \, \abs{F(\omega)}^2
	= \int_{-\infty}^{\infty} \dif \omega \, \frac{\abs \omega}{\del{1+\omega^2}^2}
	= 2 \int_0^{\infty} \dif \omega \, \frac{\omega}{\del{1+\omega^2}^2} \\
	\intertext{%
		Wir substituieren $z(\omega) := 1 + \omega^2$ mit $\dif z / (2 \omega) = \dif \omega$.
	}
	&= \int_{z(0)}^{z(\infty)} \dif z \, \frac{1}{z^2}
	= \sbr{- \frac{1}{z}}_{z(0)}^{z(\infty)}
	= \sbr{- \frac{1}{1 + \omega^2}}_{0}^{\infty}
	= 1
\end{align*}

%\bibliography{../../zentrale_BibTeX/Central}
%\bibliographystyle{plain}

\end{document}

% vim: spell spelllang=de
