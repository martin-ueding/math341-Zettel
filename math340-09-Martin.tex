% Copyright © 2012 Martin Ueding <dev@martin-ueding.de>
%
\input{header.tex}

\usepackage{fancyhdr}
\usepackage{tikz}

\newcommand{\themodul}{math340}
\newcommand{\thegruppe}{Gruppe 16 -- Malte Lackmann}
\newcommand{\theuebung}{9}

\pagestyle{fancy}

\fancyfoot[C]{\footnotesize{\thegruppe}}
\fancyfoot[L]{\footnotesize{Ueding, Manz, Lemmer}}
\fancyfoot[R]{\footnotesize{Seite \thepage\ / \pageref{LastPage}}}
\fancyhead[L]{\themodul{} -- Übung \theuebung}

\def\thesection{\theuebung.\arabic{section}}
\def\thesubsection{\thesection\alph{subsection}}

\title{\themodul{} -- Übung \theuebung \\ \vspace{0.5cm} \large{\thegruppe}}

\author{
	Martin Ueding \\ \small{\href{mailto:mu@uni-bonn.de}{mu@uni-bonn.de}}
	\and
	Paul Manz
	\and
	Lino Lemmer
}

\begin{document}

\maketitle

\begin{table}[h]
	\centering
	\begin{tabular}{l|c|c|c|c|c}
		Aufgabe & \ref 1 & \ref 2 & \ref 3 & $\sum$   \\
		\hline
		Punkte & \punkte / 7 & \punkte / 5 & \punkte 3 & \punkte / 15
	\end{tabular}
\end{table}

%%%%%%%%%%%%%%%%%%%%%%%%%%%%%%%%%%%%%%%%%%%%%%%%%%%%%%%%%%%%%%%%%%%%%%%%%%%%%%%
%                            Fouriertransformierte                            %
%%%%%%%%%%%%%%%%%%%%%%%%%%%%%%%%%%%%%%%%%%%%%%%%%%%%%%%%%%%%%%%%%%%%%%%%%%%%%%%

\section{Fouriertransformierte}
\label 1

\subsection{}

\fehlt

\subsection{}

\fehlt

%%%%%%%%%%%%%%%%%%%%%%%%%%%%%%%%%%%%%%%%%%%%%%%%%%%%%%%%%%%%%%%%%%%%%%%%%%%%%%%
%                              Dirichletproblem                               %
%%%%%%%%%%%%%%%%%%%%%%%%%%%%%%%%%%%%%%%%%%%%%%%%%%%%%%%%%%%%%%%%%%%%%%%%%%%%%%%

\section{Dirichletproblem}
\label 2

Gelöst werden soll die Laplacegleichung mit Randbedingung:
\[
	\dpd[2] ux + \dpd[2] uy = 0
	\quad\wedge\quad
	u(x, 0) = f(x)
\]

Wir setzen:
\[
	W(\omega, y) := \fourier_x(u)(\omega, y)
	,\quad
	W_0(\omega) := \fourier_x\del{f}(\omega)
	,\quad
	\dpd[2] Wx := \fourier_x\del{\dpd[2] ux} = - \omega^2 W
\]

Wir wenden die Fouriertransformation in $x \to \omega$, $\fourier_x$, auf die
Differentialgleichung an und erhalten (unter Benutzung der Linearität):
\[
	\fourier_x\del{\dpd[2] ux} + \fourier_x \del{\dpd[2] uy} = 0
	\iff
	- \omega^2 W + \dpd[2] Wy = 0
\]

Diese Differentialgleichung können wir durch einen Exponentialansatz lösen:
\[
	W(\omega, y) = \del{c_1 \ee^{\ii \omega y} + c_2 \ee^{-\ii \omega y}} W_0(\omega)
\]

%%%%%%%%%%%%%%%%%%%%%%%%%%%%%%%%%%%%%%%%%%%%%%%%%%%%%%%%%%%%%%%%%%%%%%%%%%%%%%%
%                                 Zeitsignal                                  %
%%%%%%%%%%%%%%%%%%%%%%%%%%%%%%%%%%%%%%%%%%%%%%%%%%%%%%%%%%%%%%%%%%%%%%%%%%%%%%%

\section{Zeitsignal}
\label 3

\fehlt

%\bibliography{../../zentrale_BibTeX/Central}
%\bibliographystyle{plain}

\end{document}

% vim: spell spelllang=de
