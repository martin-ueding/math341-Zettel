% Copyright © 2012 Martin Ueding <dev@martin-ueding.de>
%
\documentclass[11pt, ngerman, fleqn]{article}

\usepackage[a4paper, left=3cm, right=2cm, top=2cm, bottom=2cm]{geometry}
\usepackage[activate]{pdfcprot}
\usepackage[cdot, squaren]{SIunits}
\usepackage[iso]{isodate}
\usepackage[parfill]{parskip}
\usepackage[T1]{fontenc}
\usepackage[utf8]{inputenc}
\usepackage{amsmath}
\usepackage{amsthm}
\usepackage{babel}
\usepackage{color}
\usepackage{commath}
\usepackage{epstopdf}
\usepackage{fancyhdr}
\usepackage{graphicx}
\usepackage{hyperref}
\usepackage{lastpage}
\usepackage{setspace}
\usepackage{tikz}

\usepackage[charter, greekuppercase=italicized]{mathdesign}

\definecolor{darkblue}{rgb}{0,0,.5}
\definecolor{darkgreen}{rgb}{0,.5,0}

\hypersetup{
	breaklinks=false,
	citecolor=darkgreen,
	colorlinks=true,
	linkcolor=black,
	menucolor=black,
	urlcolor=darkblue,
}

\setlength{\columnsep}{2cm}

\DeclareMathOperator{\arcsinh}{arsinh}
\DeclareMathOperator{\arsinh}{arsinh}
\DeclareMathOperator{\asinh}{arsinh}
\DeclareMathOperator{\card}{card}
\DeclareMathOperator{\diam}{diam}

\newcommand{\dalambert}{\mathop{{}\Box}\nolimits}
\newcommand{\divergence}[1]{\inner{\vnabla}{#1}}
\newcommand{\ee}{\mathrm e}
\newcommand{\emesswert}{\del{\messwert \pm \messwert}}
\newcommand{\e}[1]{\cdot 10^{#1}}
\newcommand{\fehlt}{\textcolor{red}{Hier fehlen noch Inhalte.}}
\newcommand{\half}{\frac 12}
\newcommand{\ii}{\mathrm i}
\newcommand{\inner}[2]{\left\langle #1, #2 \right\rangle}
\newcommand{\laplace}{\mathop{{}\bigtriangleup}\nolimits}
\newcommand{\messwert}{\textcolor{blue}{\square}}
\newcommand{\punkte}{\textcolor{white}{xxxxx}}
\newcommand{\tens}[1]{\boldsymbol{#1}}
\newcommand{\vnabla}{\vec \nabla}
\renewcommand{\vec}[1]{\boldsymbol{#1}}

\newcommand{\themodul}{math340}
\newcommand{\thegruppe}{Gruppe 16 -- Malte Lackmann}
\newcommand{\theuebung}{6}

\pagestyle{fancy}

\fancyfoot[C]{\footnotesize{\thegruppe}}
\fancyfoot[L]{\footnotesize{Ueding, Manz, Lemmer}}
\fancyfoot[R]{\footnotesize{Seite \thepage\ / \pageref{LastPage}}}
\fancyhead[L]{\themodul{} -- Übung \theuebung}

\def\thesection{\theuebung.\arabic{section}}
\def\thesubsection{\thesection\alph{subsection}}

\title{\themodul{} -- Übung \theuebung \\ \vspace{0.5cm} \large{\thegruppe}}

\author{
	Martin Ueding \\ \small{\href{mailto:mu@uni-bonn.de}{mu@uni-bonn.de}}
	\and
	Paul Manz
	\and
	Lino Lemmer
}

\begin{document}

\maketitle

\begin{table}[h]
	\centering
	\begin{tabular}{l|c|c|c|c}
		Aufgabe & 6.1 & 6.2 & \ref{1} & $\sum$   \\
		\hline
		Punkte & \punkte / 3 & \punkte / 3 & \punkte / 6 & \punkte / 12
	\end{tabular}
\end{table}


\setcounter{section}{3}

%%%%%%%%%%%%%%%%%%%%%%%%%%%%%%%%%%%%%%%%%%%%%%%%%%%%%%%%%%%%%%%%%%%%%%%%%%%%%%%
%                              Dirichletproblem                               %
%%%%%%%%%%%%%%%%%%%%%%%%%%%%%%%%%%%%%%%%%%%%%%%%%%%%%%%%%%%%%%%%%%%%%%%%%%%%%%%

\section{Dirichletproblem}
\label 1

In Polarkoordinaten ist die Laplacegleichung:
\[
	\laplace u
	= \frac 1\rho \dpd{}\rho \del{\rho \dpd u \rho} + \frac{1}{\rho^2} \dpd[2] u \phi
	= 0
\]

Zur Separation wählen wir:
\[
	u(\rho, \phi) = v(\rho) w(\phi)
\]

In die Differentialgleichung eingesetzt erhalten wir nach einigen Umformungen:
\[
	\frac{w''}{w} = - \alpha^2
	\quad \wedge \quad
	\rho^2 v'' + \rho v' - \alpha^2 v = 0
\]

Die Gleichung in $w$ können wir durch $w(\phi) = c_1 \cos(\alpha \phi) + c_2
\sin(\alpha \phi)$ lösen. Nun setzen wir wie auf dem Aufgabenzettel angegeben
$p = \ee^t$ für die Gleichung in $\rho$:
\[
	\ee^{2t} v''\del{\ee^t} + \ee^t v'\del{\ee^t} - \alpha^2 v\del{\ee^t} = 0
\]

Wir führen eine neue Funktion $a(t) := v\del{\ee^t}$ ein, deren zweite Ableitung $a''(t) = v''\del{\ee^t} \ee^{2t} + v'\del{\ee^t} \ee^t$ ist. Damit können wir die Differentialgleichung vereinfachen zu:
\[
	a'' - \alpha^2 a = 0
\]

Die Lösung davon sind Exponentialfunktionen, nach Rücksubstitution erhalten wir:
\[
	v(\rho) = c_3 \rho^\alpha + c_4 \rho^{-\alpha}
\]

Die Funktion $u$ setzen wir aus beidem zusammen:
\[
	u(\rho, \phi) = \del{c_3 \rho^\alpha + c_4 \rho^{-\alpha}} \del{c_1 \cos(\alpha \phi) + c_2 \sin(\alpha \phi)} + c_5
\]

Wir wählen $c_5 = 3$. Damit bei $\rho=1$ nur $3$ herauskommt sowie bei $\rho = 3$ der Wert $3 + 3 \cos\del\phi$, müssen wir $c_4=0$ wählen. Wir wählen $c_3 = 1$ und müssen nun noch folgende Gleochungen für $c_1$ und $c_2$ erfüllen:
\[
	c_1 + c_2 = 3
	, \quad
	3 c_1 + \frac 13 c_2 = 3
\]

Damit erhalten wir $c_1 = 9/8$ und $c_2 = - 9/8$. Die Lösung der Differentialgleichung ist also:
\[
	u(\rho, \phi) = \del{\frac 98 r - \frac 98 \frac 1r}\cos\del\phi + 3
\]

Wenn wir zur Kontrolle in diese Funktion $\rho = 1$ einsetzen, erhalten wir nur die $3$. Setzen wir $\rho = 3$ ein, erhalten wir $\del{27/8-3/8}\cos\del\phi = 3 \cos\del\phi$.

Einsetzen in die Laplacegleichung ergibt $0$.

%\bibliography{../../zentrale_BibTeX/Central}
%\bibliographystyle{plain}

\end{document}

.. vim: spell spelllang=de
